\documentclass[]{article}
%made from template named MathArticle
\usepackage{amsfonts}
\usepackage{amsmath}
\usepackage{amssymb}
\usepackage{hyperref}
\hypersetup{colorlinks=true}
\usepackage{graphics}

%Fix \eqref in section title
\pdfstringdefDisableCommands{\def\eqref#1{(\ref{#1})}}

\DeclareMathOperator{\es}{Es}
\DeclareMathOperator{\ez}{Ez}
\DeclareMathOperator{\gs}{gs}
\DeclareMathOperator{\md}{mod}
%Parenthesis, Braces, Brackets usepackage{physics}
\newcommand{\pqty}[1]{{\left(#1\right)}}
\newcommand{\Bqty}[1]{{\left\{#1\right\}}}
\newcommand{\bqty}[1]{{\left[#1\right]}}
\newcommand{\abs}[1]{{\left\lvert#1\right\rvert}}
%other stuff
\newcommand{\floor}[1]{{\left\lfloor#1\right\rfloor}}
\newcommand{\ceil}[1]{{\left\lceil#1\right\rceil}}
%Laplace transform and inverse
\newcommand{\laplace}[1]{\mathcal{L}\Bqty{#1}\pqty{s}}
\newcommand{\laplaceInv}[1]{\mathcal{L}^{-1}\Bqty{#1}\pqty{t}}
%Derivatives
\newcommand{\pdiff}[2]{\frac{\partial^{#2}}{\partial #1^{#2}}}

%\usepackage{minted}
%opening
\title{Exponential Sum Function related to Generalized Split-complex numbers}
\author{Nazgand: nazgand@gmail.com}

\begin{document}
	
	\maketitle
	
	\begin{abstract}
		The goal of this paper is to analyze a class of functions which are equal to their own \(n\)th derivative. Disclaimer: this is a draft.
	\end{abstract}
	
	\section{Exponential Sum Definition}
	For the fundamental definition where \(n\in\mathbb{Z}^+\), define:
	\begin{equation}
	\label{Es fundamenta definition}
	\es_n\pqty{x}=
	\sum_{k=0}^{\infty}\frac{x^{nk}}{\pqty{nk}!}
	\end{equation}
	
	\section{Laplace inverse transform form}
	\begin{equation}
	\es_n\pqty{t}=
	\laplaceInv{\frac{s^{n-1}}{s^n-1}}
	\end{equation}
	
	Equivalence is shown using the General derivative rule for Laplace transforms:
	\begin{equation}
	\label{Laplace General Derivatives}
	\laplace{f^{(n)}\pqty{t}}=
	s^n \laplace{f\pqty{t}} - \sum_{k=1}^{n} s^{n-k} f^{(k-1)}\pqty{0^{+}}
	\end{equation}
	
	Substitute \(f\) with \(\es_n^{(k)}\) in \eqref{Laplace General Derivatives}, use the derivatives provided by \eqref{Es fundamenta definition} to simplify noting \(\md\pqty{-m,n}=\md\pqty{k-1,n}\) is the only surviving summand:
	\begin{equation}
	\laplace{\es_n^{(m)}\pqty{t}}=
	\laplace{\es_n^{(m+n)}\pqty{t}}=
	s^n\laplace{\es_n^{(m)}\pqty{t}}-s^{\md\pqty{m-1,n}}
	\end{equation}
	
	Solve:
	\begin{equation}
	\laplace{\es_n^{(m)}\pqty{t}}=
	\frac{s^{\md\pqty{m-1,n}}}{s^n-1}
	\end{equation}
	
	
	
	\section{As a sum of exponential functions}
	The reason this function is named $\es$ is because it is an Exponential Sum function.
	\begin{equation}
	\label{Es Exponential sum form}
	\es_n\pqty{x}=
	\frac{1}{n}\sum _{k=1}^n \exp\pqty{xe^{2ki\pi/n}}
	\end{equation}
	Proof of equivalence to \eqref{Es fundamenta definition} via Taylor series:
	\begin{equation}
	\sum _{k=1}^n \exp\pqty{xe^{2ki\pi/n}}
	=
	\sum _{k=1}^n \sum _{j=0}^\infty e^{2jki\pi/n}\frac{x^j}{j!}
	=
	\sum _{j=0}^\infty \frac{x^j}{j!} \sum _{k=1}^n e^{2jki\pi/n}
	=
	\sum_{m=0}^{\infty}\frac{nx^{nm}}{\pqty{nm}!}
	\end{equation}
	
	
	\section{Real formulae derived from \eqref{Es Exponential sum form}}
	This section exploits this fact:
	\begin{equation}
	\exp\pqty{xe^{iy}}+\exp\pqty{xe^{-iy}}
	=
	2e^{x\cos\pqty{y}} \cos\pqty{x\sin\pqty{y}}
	\end{equation}
	Thus:
	\begin{equation}
	\label{Es trig sum form 1}
	\es_n\pqty{x}
	=
	\frac{1}{n}\pqty{
		e^x+e^{-x}\frac{1+\cos\pqty{\pi n}}{2} 
		+2\sum _{k=1}^{\lceil n/2\rceil-1}e^{x\cos\pqty{\frac{2k\pi}{n}}} \cos\pqty{x\sin\pqty{\frac{2k\pi}{n}}}
	}
	\end{equation}
	And:
	\begin{equation}
	\es_n\pqty{xe^{i\pi/n}}
	=
	\frac{1}{n}\pqty{
		e^{-x}\frac{1-\cos\pqty{\pi n}}{2}
		+2\sum _{k=1}^{\lfloor n/2\rfloor}e^{x\cos\pqty{\frac{2k-1}{n}\pi}} \cos\pqty{x\sin\pqty{\frac{2k-1}{n}\pi}}
	}
	\end{equation}
	
	
	
	\section{other}
	Notable values of \(n\):
	\begin{equation}
	\es_1\pqty{x}=e^x\land\es_2\pqty{x}=\cosh\pqty{x}\land\es_4\pqty{x}=\cosh\pqty{\frac{x}{1+i}}\cosh\pqty{\frac{x}{1-i}}
	\end{equation}
	
	Complex rotation property:
	\begin{equation}
	\es_n^{(k)}\pqty{x}=
	\es_n^{(k)}\pqty{x*e^{2i\pi/n}}e^{2ki\pi/n}
	\end{equation}
	
	Derivative sum rules:
	\begin{equation}
	e^x=\sum_{k=0}^{n-1}\es_n^{(k)}\pqty{x}\land\es_n\pqty{x}=\sum_{k=0}^{m-1}\es_{nm}^{(kn)}\pqty{x}
	\end{equation}
	
	\begin{equation}
	\es_m\pqty{x}=
	\frac{1}{m}
	\sum_{k=1}^{n}\sum_{j=1}^{m}\es_n^{(k)}\pqty{x\exp\pqty{\frac{2 i \pi j}{m}}}
	\end{equation}
	
	Argument sum rule:
	\begin{equation}
	e^{x+y}=\sum_{k=0}^{n-1}\sum_{j=0}^{n-1}\es_n^{(k)}\pqty{x}\es_n^{(j)}\pqty{y}
	\end{equation}
	
	\begin{equation}
	\es_n\pqty{x+y}=
	\frac{1}{n}\sum _{l=1}^n \sum_{k=0}^{n-1}\sum_{j=0}^{n-1}
	\es_n^{(k)}\pqty{x\exp\pqty{\frac{2 i \pi  l}{n}}}
	\es_n^{(j)}\pqty{y\exp\pqty{\frac{2 i \pi  l}{n}}}
	\end{equation}
	
	\begin{equation}
	\es_n\pqty{x+y}=
	\frac{1}{n}\sum _{l=1}^n \sum_{k=0}^{n-1}\sum_{j=0}^{n-1}
	\es_n^{(k)}\pqty{x}
	\es_n^{(j)}\pqty{y}
	\exp\pqty{\frac{-2(j+k) i \pi  l}{n}}
	\end{equation}
	
	The \(l\) index cancels out all the terms where \(j+k\neq 0\), so:
	\begin{equation}
	\es_n\pqty{x+y}=
	\sum_{k=0}^{n-1}
	\es_n^{(k)}\pqty{x}
	\es_n^{(n-k)}\pqty{y}
	\end{equation}
	
	Differentiation:
	\begin{equation}
	\es_n^{(m)}\pqty{x+y}=
	\sum_{k=0}^{n-1}
	\es_n^{(m+k)}\pqty{x}
	\es_n^{(n-k)}\pqty{y}
	\end{equation}

	Substituting $x=-y$ produces the equations analogous to $\sin\pqty{x}^2 + \cos\pqty{x}^2=1$.
	\begin{equation}
	1=
	\sum_{k=0}^{n-1}
	\es_n^{(k)}\pqty{x}
	\es_n^{(n-k)}\pqty{-x}
	\end{equation}
	
	More generally:
	\begin{equation}
	\delta\pqty{\md\pqty{m,n}}=
	\sum_{k=0}^{n-1}
	\es_n^{(m+k)}\pqty{x}
	\es_n^{(n-k)}\pqty{-x}
	\end{equation}
	
	
	\section{Real and imaginary parts}
	Using \eqref{Es Exponential sum form}
	\begin{equation}
	\Re\pqty{\es_n\pqty{z}}=
	\frac{1}{n}\sum _{k=1}^n \exp\pqty{ze^{2ki\pi/n}}\exp\pqty{ze^{2ki\pi/n}}
	\end{equation}
	
	For $n\geq 2$, arbitrary complex numbers many be decomposed as $z=\Re\pqty{z}+e^{i\pi/n}$
	
	\section{Weierstrass product conjecture}
	This conjecture states all zeroes of $\es_n^{(m)}$ not on the origin are of degree 1 and are located on the critical rays $e^{i\pi(1+2k)/n}, k\in\mathbb{Z}$.
	The absolute values of the zeroes are denoted
	\begin{equation}
	\ez\pqty{n,m,1}=\min\Bqty{\es_n^{(m)}\pqty{r*e^{i\pi/n}}\Big\vert r\in\mathbb{R}^+}
	\end{equation}
	\begin{equation}
	\ez\pqty{n,m,k+1}=\min\Bqty{\es_n^{(m)}\pqty{r*e^{i\pi/n}}\Big\vert r\in\mathbb{R}^+, r>\ez\pqty{n,m,k}}
	\end{equation}
	
	Moreover, for $n>1$,
	\begin{equation}
	\es_n^{(m)}\pqty{z}=z^{\md\pqty{-m,n}}\prod_{k=1}^{\infty}\pqty{1+\pqty{\frac{z}{\ez\pqty{n,m,k}}}^n}
	\end{equation}
	
	This conjecture is easily verifiable for $\pqty{n,m}\in\Bqty{\pqty{2,0}, \pqty{2,1}, \pqty{4,0}}$.
	
	Work has been done in German to show this works for \(\es_n\pqty{z}\).
	
	\url{https://math.stackexchange.com/questions/3221569/conjecture-all-complex-roots-of-sum-k-0-infty-fraczk-leftnk-right}
	
	\subsection{Bounds on Complex Zeroes \(\es_3\)}
	Lemma
	\begin{equation}
	\cos\pqty{a+bi}=\cos\pqty{a}\cosh\pqty{b}-i\sin\pqty{a}\sinh\pqty{b}
	\end{equation}
	\begin{equation}
	\abs{\cos\pqty{a+bi}} \leq \abs{\cos\pqty{a}\cosh\pqty{b}}+\abs{\sin\pqty{a}\sinh\pqty{b}}
	\end{equation}
	\begin{equation}
	\abs{\cos\pqty{a+bi}} \leq \abs{\cosh\pqty{b}}+\abs{\sinh\pqty{b}}
	= \cosh\pqty{\abs{b}}+{\sinh\pqty{\abs{b}}} = \exp\pqty{\abs{b}}
	\end{equation}
	\begin{equation}
	\label{cos exp bound lemma}
	\abs{\cos\pqty{a+bi}} \leq \exp\pqty{\abs{b}}
	\end{equation}
	A bound which supports this conjecture is attainable from \eqref{Es trig sum form 1}.
	\begin{equation}
	\es_3\pqty{x}=
	\frac{1}{3}\pqty{
		e^x
		+2e^{x\cos\pqty{\frac{2\pi}{3}}} \cos\pqty{x\sin\pqty{\frac{2\pi}{3}}}
	}
	\end{equation}
	\begin{equation}
	0=\es_3\pqty{a+bi}=
	\frac{1}{3}\pqty{
		e^\pqty{a+bi}
		+2e^{\pqty{a+bi}\cos\pqty{\frac{2\pi}{3}}} \cos\pqty{\pqty{a+bi}\sin\pqty{\frac{2\pi}{3}}}
	}
	\end{equation}
	\begin{equation}
	-e^\pqty{a+bi}=
	2e^{\pqty{a+bi}\cos\pqty{\frac{2\pi}{3}}} \cos\pqty{\pqty{a+bi}\sin\pqty{\frac{2\pi}{3}}}
	\end{equation}
	Absolute value:
	\begin{equation}
	e^{a}=
	2e^{{a}\cos\pqty{\frac{2\pi}{3}}} \abs{\cos\pqty{\pqty{a+bi}\sin\pqty{\frac{2\pi}{3}}}}
	\end{equation}
	\begin{equation}
	\exp\pqty{\frac{3a}{2}}=
	2 \abs{\cos\pqty{\pqty{a+bi}\sin\pqty{\frac{2\pi}{3}}}}
	\end{equation}
	Use lemma \eqref{cos exp bound lemma}:
	\begin{equation}
	\exp\pqty{\frac{3a}{2}} \leq
	2 {\exp\pqty{\abs{b}\sin\pqty{\frac{2\pi}{3}}}}
	\end{equation}
	\begin{equation}
	{\frac{3a}{2}} \leq
	\ln{2}+ {\abs{b}\frac{\sqrt{3}}{2}}
	\end{equation}
	The last inequality carves out a sector of the complex plane where zeroes are not allowed. The symmetric properties of \(\es_3\) restrict zeroes to appearing near the conjectured region.
	
	
	\subsection{Non-Rigorous Bounds on Complex Zeroes \(\es_n, n\geq 3\)}
	A bound which supports this conjecture is attainable from \eqref{Es trig sum form 1}.
	\begin{equation}
	\es_n\pqty{x}
	=
	\frac{1}{n}\pqty{
		e^x+e^{-x}\frac{1+\cos\pqty{\pi n}}{2} 
		+2\sum _{k=1}^{\lceil n/2\rceil-1}e^{x\cos\pqty{\frac{2k\pi}{n}}} \cos\pqty{x\sin\pqty{\frac{2k\pi}{n}}}
	}
	\end{equation}
	Note \(\es_n\pqty{x}\) has the same zeroes as \(\es_n\pqty{x}\exp\pqty{-x\cos\pqty{\frac{2\pi}{n}}}\), and most terms go to zero as \(\Re\pqty{x}\to\infty\).
	\begin{equation}
	\es_n\pqty{x}\exp\pqty{-x\cos\pqty{\frac{2\pi}{n}}}
	\approx\mathcal{O}\pqty{
	\frac{1}{n}\pqty{
		e^{x\pqty{1-\cos\pqty{\frac{2\pi}{n}}}}
		+2 \cos\pqty{x\sin\pqty{\frac{2\pi}{n}}}}
	}
	\end{equation}
	Zeroes of \(\es_n\pqty{x}\) with large real parts are approximately bound by this equation:
	\begin{equation}
		0=
		e^{x\pqty{1-\cos\pqty{\frac{2\pi}{n}}}}
		+2 \cos\pqty{x\sin\pqty{\frac{2\pi}{n}}}
	\end{equation}
	\begin{equation}
	-
	e^{\pqty{a+b*i}\pqty{1-\cos\pqty{\frac{2\pi}{n}}}}=
	2 \cos\pqty{\pqty{a+b*i}\sin\pqty{\frac{2\pi}{n}}}
	\end{equation}
	Absolute value:
	\begin{equation}
	\exp\pqty{a\pqty{1-\cos\pqty{\frac{2\pi}{n}}}}=
	2 \abs{\cos\pqty{\pqty{a+b*i}\sin\pqty{\frac{2\pi}{n}}}}
	\end{equation}
	Use lemma \eqref{cos exp bound lemma}:
	\begin{equation}
	\exp\pqty{a\pqty{1-\cos\pqty{\frac{2\pi}{n}}}}\leq
	2 \exp\pqty{\abs{b}\sin\pqty{\frac{2\pi}{n}}}
	\end{equation}
	Logarithm:
	\begin{equation}
	{a\pqty{1-\cos\pqty{\frac{2\pi}{n}}}}\leq
	\ln\pqty{2} + {\abs{b}\sin\pqty{\frac{2\pi}{n}}}
	\end{equation}
	Equation form with slope:
	\begin{equation}
	{a\leq
	\frac{\ln\pqty{2}}{{1-\cos\pqty{\frac{2\pi}{n}}}}} + {\abs{b}\cot\pqty{\frac{\pi}{n}}}
	\end{equation}
	The last inequality carves out a sector of the complex plane where zeroes are not allowed. The symmetric properties of \(\es_n\) restrict zeroes to appearing near the conjectured region.
	
	\section{Riemann Liouville Operator}
	TODO
	
	\section{Generalized split-complex numbers}
	\subsection{Introduction}
	Split complex numbers a ring extending the reals with an extra unit which is a square root of 1 which is not a complex number.
	\subsection{Generalization}
	This concept may be generalized to rings with $n$th roots of 1, and these rings are interrelated. Define the general split units as
	\begin{equation}
	\gs\pqty{r_0}*\gs\pqty{r_1}=\gs\pqty{\md\pqty{r_0+r_1,1}}
	\land
	\bqty{\gs\pqty{z}\in\mathbb{C}\Leftrightarrow z\in\mathbb{Z}}
	\end{equation}
	The general split units commute with complex numbers:
	\begin{equation}
	c\in\mathbb{C}\Rightarrow c*\gs\pqty{r_0}=\gs\pqty{r_0}*c
	\end{equation}
	
	Simple deduction from the Taylor series of the exponential function for $m\in\mathbb{Z}, c\in\mathbb{C}$.
	\begin{equation}
	\exp\pqty{c*\gs\pqty{\frac{m}{n}}}=\sum_{k=0}^{n-1}\gs\pqty{\frac{m}{n}}^k \es_n^{(n-k)}\pqty{c}
	=\sum_{k=0}^{n-1}\gs\pqty{\frac{km}{n}} \es_n^{(n-k)}\pqty{c}
	\end{equation}
	
	The $n$-split-complex number ring is
	\begin{equation}
	\mathbf{Gs}_n=\Bqty{\sum_{k=0}^{n-1} c_k*\gs\pqty{\frac{k}{n}} \Bigg\vert c_k\in\mathbb{C}}
	\end{equation}
	
	Sub-ring property:
	\begin{equation}
	m\in\mathbb{Z}^+\Rightarrow\mathbf{Gs}_n\subseteq\mathbf{Gs}_{m*n}
	\end{equation}
	
	This self-isomorphism of $f:\mathbf{Gs}_n\rightarrow\mathbf{Gs}_n$ exists for $m\in\mathbb{Z}\land\gcd\pqty{j,n}=1$.
	\begin{equation}
	f\pqty{\sum_{k=0}^{n-1} c_k*\gs\pqty{\frac{k}{n}}}=\sum_{k=0}^{n-1} c_k\exp\pqty{\frac{2kmi\pi}{n}}\gs\pqty{\frac{kj}{n}}
	\end{equation}
	
	The above is a homomorphism for arbitrary $j,m\in\mathbb{Z}$. $f:\mathbf{Gs}_n\rightarrow\mathbf{Gs}_{n/\gcd\pqty{j,n}}$.
	\begin{equation}
	\label{GS Homomorphism}
	f\pqty{\sum_{k=0}^{n-1} c_k*\gs\pqty{\frac{k}{n}}}=\sum_{k=0}^{n-1} c_k\exp\pqty{\frac{2kmi\pi}{n}}\gs\pqty{\frac{kj}{n}}
	\end{equation}
	
	Co-set isomorphism property, $f:\mathbf{Gs}_{m_0n}/\mathbf{Gs}_{m_0}\rightarrow \mathbf{Gs}_{m_1n}/\mathbf{Gs}_{m_1}$ for $n,m_0,m_1\in\mathbb{Z}^+$
	\begin{equation}
	f\pqty{\mathbf{Gs}_{m_0}*\sum_{k=0}^{n-1} c_k*\gs\pqty{\frac{k}{m_0n}}}=\mathbf{Gs}_{m_1}*\sum_{k=0}^{n-1} c_k*\gs\pqty{\frac{k}{m_1n}}
	\end{equation}
	
	Idempotent elements: Easily found:
	\begin{equation}
	\pqty{\sum_{k=0}^{n-1}\frac{1}{n}\gs\pqty{\frac{k}{n}}}^2
	=
	\sum_{k=0}^{n-1}\frac{1}{n}\gs\pqty{\frac{k}{n}}
	\end{equation}
	Because homomorphisms map idempotent elements to idempotent elements, using \eqref{GS Homomorphism} produces a set of idempotent elements
	\begin{equation}
	\Bqty{ \sum_{k=0}^{n-1} \frac{1}{n}\exp\pqty{\frac{2kmi\pi}{n}}\gs\pqty{\frac{kj}{n}} \Bigg\vert m,j\in\mathbb{Z} }
	\end{equation}
	The set of idempotent elements is closed under multiplication in abelian groups.
	\begin{equation}
	\bqty{n\in\mathbb{Z}^+\land\pqty{F_1F_2}^n=\pqty{F_2F_1}^n\land F_1^2=F_1\land F_2^2=F_2}\Rightarrow \pqty{\pqty{F_1F_2}^n}^2=\pqty{F_1F_2}^n
	\end{equation}.
	\begin{equation}
	\bqty{F_1F_2=F_2F_1\land F_1^2=F_1\land F_2^2=F_2}\Rightarrow \pqty{{F_1+F_2-2F_1F_2}}^2=\pqty{F_1+F_2-2F_1F_2}
	\end{equation}.
	
	\subsection{Possible applications}
	\url{http://www.math.usm.edu/lee/quantum.html} Traditional Split-complex numbers, a sub-ring of $\mathbf{Gs}_2$, are used in quantum mechanics, thus General Split-complex numbers likely also have application.
	
	\section{More general rings}
	\subsection{Extending the reals}
	The goal here is to extend the real numbers, $\mathbb{R}$, using an abelian multiplicative group, $G$, such that exponential functions still work. The extension is the abelian ring $\Bqty{\sum_{g\in G}g*x_g \bigg\vert x_g\in\mathbb{R}}$, denoted $\mathbb{R}\bqty{G}$.
	
	Defining
	\begin{equation}
	\exp(x)=\sum_{k=0}^{\infty}\frac{x^k}{k!}
	\end{equation}
	, an immediate consequence for all rings is
	\begin{equation}
	x_1*x_2=x_2*x_1\Rightarrow\exp(x_1)*\exp(x_2)=\exp(x_1+x_2)
	\end{equation}
	. The ring $\mathbb{R}\bqty{G}$ is abelian, thus $x_1,x_2\in\mathbb{R}\bqty{G}\Rightarrow\exp(x_1)*\exp(x_2)=\exp(x_1+x_2)$. The elements of finite order may be simplified:
	\begin{equation}
	\bqty{1=g^n\land n\in\mathbb{Z}^+\land g\in G\land x\in\mathbb{R}}\Rightarrow\exp\pqty{g*x}=\sum_{k=0}^{n-1}g^k*\es_n^{(n-k)}\pqty{x}
	\end{equation}
	Every homomorphism, $f\pqty{a*b}=f\pqty{a}*f\pqty{b}$, $f:G_1\rightarrow G_2$ may be extended to $f:\mathbb{R}\bqty{G_1}\rightarrow \mathbb{R}\bqty{G_2}$, $f\pqty{a+b}=f\pqty{a}+f\pqty{b}$, $f\pqty{a*b}=f\pqty{a}*f\pqty{b}$, $f\pqty{\exp\pqty{a}}=\exp\pqty{f\pqty{a}}$.
	
	By the fundamental theorem of finite abelian groups, every finite abelian group is the direct sum of cyclic groups. Every cyclic group is isomorphic to a subgroup of $\pqty{\mathbb{R}/\mathbb{Z},+}$. Thus every finite abelian group is isomorphic to a subgroup of the ring generated by $\pqty{\pqty{\mathbb{R}/\mathbb{Z}}^{\aleph_0},+}$.
	
	\subsection{Extending other rings}
	The goal here is to extend an abelian ring, $R_a$, with an exponential function, $x_1,x_2\in R_a\Rightarrow f(x_1+x_2)=f(x_1)*f(x_2)$, using an abelian multiplicative group, $G$.
	$
	f\pqty{0}=f\pqty{0}^2
	$, i.e. $f\pqty{0}$ is idempotent. $\Bqty{f(x)*a \vert a^2=a\land a\in R_a}$ are all valid exponential functions.
	
	
	
	\section{Bibliography}
	\url{https://math.stackexchange.com/questions/1542147/find-the-complex-or-real-roots-of-e-frac3-x22-cos-left-frac-sqrt}
	\url{https://en.wikipedia.org/wiki/Split-complex_number}
	
\end{document}
