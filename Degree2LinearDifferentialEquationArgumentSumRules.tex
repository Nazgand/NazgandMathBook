\documentclass[]{article}
%margins
\usepackage[a4paper,margin=0.15in]{geometry}
%document colors
\usepackage{xcolor}
\definecolor{favoritecolor1}{HTML}{607CB2}
\definecolor{favoritecolor2}{HTML}{303E59}
\makeatletter
\newcommand{\globalcolor}[1]{%
	\color{#1}\global\let\default@color\current@color
}
\makeatother

\AtBeginDocument{\globalcolor{favoritecolor2}}
\pagecolor{favoritecolor1}

%made from template named MathArticleTemplate
\usepackage{amsfonts}
\usepackage{amsmath}
\usepackage{amsthm}
\usepackage{amssymb}
\usepackage{hyperref}
\hypersetup{colorlinks=true}
\usepackage{graphics}

%Fix \eqref in section title
\pdfstringdefDisableCommands{\def\eqref#1{(\ref{#1})}}

\DeclareMathOperator{\es}{Es}
\DeclareMathOperator{\ez}{Ez}
\DeclareMathOperator{\gs}{gs}
\DeclareMathOperator{\md}{mod}
\DeclareMathOperator{\pow}{Pow}
%Parenthesis, Braces, Brackets usepackage{physics}
\newcommand{\pqty}[1]{{\left(#1\right)}}
\newcommand{\Bqty}[1]{{\left\{#1\right\}}}
\newcommand{\bqty}[1]{{\left[#1\right]}}
\newcommand{\abs}[1]{{\left\lvert#1\right\rvert}}
%other stuff
\newcommand{\floor}[1]{{\left\lfloor#1\right\rfloor}}
\newcommand{\ceil}[1]{{\left\lceil#1\right\rceil}}
%Laplace transform and inverse
\newcommand{\laplace}[1]{\mathcal{L}\Bqty{#1}\pqty{s}}
\newcommand{\laplaceInv}[1]{\mathcal{L}^{-1}\Bqty{#1}\pqty{t}}
%Derivatives
\newcommand{\pdiff}[2]{\frac{\partial^{#2}}{\partial #1^{#2}}}
%Kettenbruch
\newcommand{\ketten}[4]{\underset{#1}{\overset{#2}{\LARGE\mathrm K}}\frac{#3}{#4}}

%lemma,theorem, proof
\newtheorem{theorem}{Theorem}[section]
\newtheorem{lemma}[theorem]{Lemma}
\newtheorem{definition}[theorem]{Definition}
\newtheorem{corollary}[theorem]{Corollary}

\numberwithin{equation}{section}

%\usepackage{minted}
%opening
\author{Mark Andrew Gerads: \href{MailTo:Nazgand@Gmail.Com}{Nazgand@Gmail.Com}}

\title{
	Degree 2 Differential Equation Argument Sum Rules
	
	\href{https://github.com/Nazgand/nazgandMathBook}{https://github.com/Nazgand/nazgandMathBook}
}

\begin{document}
	
	\maketitle
	
	\section{Assumptions and definitions}
	A degree 2 [homogeneous linear differential equation of constant coefficients] has the form
	\begin{equation}
		\label{DifferentialEquation}
		0=\sum_{k=0}^{2}a_k f^\pqty{k}\pqty{z}
	\end{equation}
	where $\forall k, a_k\in\mathbb{C}, a_2 \neq 0$, and $f:\mathbb{C}\to\mathbb{C}$ is differentiable infinitely many times.
	
	Let 
	\begin{equation}
		\lambda_0=\frac{-a_1-\sqrt{a_1^2-4 a_0 a_2}}{2 a_2},
		\lambda_1=\frac{-a_1+\sqrt{a_1^2-4 a_0 a_2}}{2 a_2}
	\end{equation}
	
	\section{Case $\lambda_0\neq\lambda_1$}
	\begin{equation}
		f\pqty{z}=c_0 \exp\pqty{\lambda_0 z}+c_1 \exp\pqty{\lambda_1 z},
		c_0=\frac{f^\pqty{1}\pqty{0}-f\pqty{0}\lambda_1}{\lambda_0-\lambda_1},
		c_1=\frac{f^\pqty{1}\pqty{0}-f\pqty{0}\lambda_0}{\lambda_1-\lambda_0}
	\end{equation}
	
	We can choose the basis of the set of solutions to be $\Bqty{\exp\pqty{\lambda_0 z}, \exp\pqty{\lambda_1 z}}$.
	Let
	\begin{equation}
		v_0\pqty{z}=
		\begin{pmatrix}
			\exp\pqty{\lambda_0 z} \\
			\exp\pqty{\lambda_1 z}
		\end{pmatrix}
		,
		B_0=
		\begin{pmatrix}
			c_0 & 0 \\
			0 & c_1
		\end{pmatrix}
		,
		B_1=
		\begin{pmatrix}
			c_0 \lambda_0 & 0 \\
			0 & c_1 \lambda_1
		\end{pmatrix}
	\end{equation}
	
	Then
	\begin{equation}
		\begin{pmatrix}
			f\pqty{z_0+z_1}
		\end{pmatrix}
		=v_0\pqty{z_0}^\top B_0 v_0\pqty{z_1}
		,
		\begin{pmatrix}
			f^\pqty{1}\pqty{z_0+z_1}
		\end{pmatrix}
		=v_0\pqty{z_0}^\top B_1 v_0\pqty{z_1}
	\end{equation}
	
	These facts are a bit too obvious. We want $\Bqty{f\pqty{z},f^\pqty{1}\pqty{z}}$ to be a basis of the set of solutions to get an awesome equation. Thus we require $c_0\neq 0,c_1\neq 0$.
	
	Let
	\begin{equation}
		M=
		\begin{pmatrix}
			c_0 & c_1 \\
			c_0 \lambda_0 & c_1 \lambda_1
		\end{pmatrix}
		,
		v_1\pqty{z}=
		\begin{pmatrix}
			f\pqty{z} \\
			f^\pqty{1}\pqty{z}
		\end{pmatrix}
	\end{equation}
	
	Then
	\begin{equation}
		M v_0\pqty{z}=v_1\pqty{z}
		,
		M^{-1}=
		\frac{\begin{pmatrix}
			-c_1 \lambda_1 & c_1 \\
			c_0 \lambda_0 & -c_0
		\end{pmatrix}}{c_0 c_1 \pqty{\lambda_0-\lambda_1}}
		,
		v_0\pqty{z}=M^{-1} v_1\pqty{z}
	\end{equation}
	
	Thus, by substitution,
	\begin{equation}
		\begin{pmatrix}
			f\pqty{z_0+z_1}
		\end{pmatrix}
		=\pqty{M^{-1} v_1\pqty{z_0}}^\top B_0 M^{-1} v_1\pqty{z_1}
		,
		\begin{pmatrix}
			f^\pqty{1}\pqty{z_0+z_1}
		\end{pmatrix}
		=\pqty{M^{-1} v_1\pqty{z_0}}^\top B_1 M^{-1} v_1\pqty{z_1}
	\end{equation}
	
	Simplify:
	\begin{equation}
		\begin{pmatrix}
			f\pqty{z_0+z_1}
		\end{pmatrix}
		=\frac{
		\begin{pmatrix}
			f\pqty{z_0} \\
			f^\pqty{1}\pqty{z_0}
		\end{pmatrix}^\top
		\begin{pmatrix}
			c_0 \lambda_0^2+c_1 \lambda_1^2 & -c_0 \lambda_0-c_1 \lambda_1 \\
			-c_0 \lambda_0-c_1 \lambda_1 & c_0 +c_1
		\end{pmatrix}
		\begin{pmatrix}
			f\pqty{z_1} \\
			f^\pqty{1}\pqty{z_1}
		\end{pmatrix}}{c_0 c_1 \pqty{\lambda_0-\lambda_1}^2}
	\end{equation}
	\begin{equation}
		\begin{pmatrix}
			f^\pqty{1}\pqty{z_0+z_1}
		\end{pmatrix}
		=\frac{
		\begin{pmatrix}
			f\pqty{z_0} \\
			f^\pqty{1}\pqty{z_0}
		\end{pmatrix}^\top
		\begin{pmatrix}
			\lambda_0 \lambda_1\pqty{c_0 \lambda_0+c_1 \lambda_1} & -\lambda_0 \lambda_1 \pqty{c_0+c_1} \\
			-\lambda_0 \lambda_1 \pqty{c_0+c_1} & c_0 \lambda_1+c_1\lambda_0
		\end{pmatrix}
		\begin{pmatrix}
			f\pqty{z_1} \\
			f^\pqty{1}\pqty{z_1}
		\end{pmatrix}}{c_0 c_1 \pqty{\lambda_0-\lambda_1}^2}
	\end{equation}
	
	\section{Case $\lambda_0=\lambda_1=\lambda$}
	\begin{equation}
		f\pqty{z}=c_0 \exp\pqty{\lambda z}+c_1 z \exp\pqty{\lambda z},
		c_0=f\pqty{0},
		c_1=f^\pqty{1}\pqty{0}-f\pqty{0}\lambda
	\end{equation}
	
	We can choose the basis of the set of solutions to be $\Bqty{\exp\pqty{\lambda z}, z\exp\pqty{\lambda z}}$.
	Let
	\begin{equation}
		v_0\pqty{z}=
		\begin{pmatrix}
			\exp\pqty{\lambda z} \\
			z\exp\pqty{\lambda z}
		\end{pmatrix}
		,
		B_0=
		\begin{pmatrix}
			c_0 & c_1 \\
			c_1 & 0
		\end{pmatrix}
		,
		B_1=
		\begin{pmatrix}
			c_1+c_0\lambda & c_1 \lambda \\
			c_1 \lambda & 0
		\end{pmatrix}
	\end{equation}
	
	Then
	\begin{equation}
		\begin{pmatrix}
			f\pqty{z_0+z_1}
		\end{pmatrix}
		=v_0\pqty{z_0}^\top B_0 v_0\pqty{z_1}
		,
		\begin{pmatrix}
			f^\pqty{1}\pqty{z_0+z_1}
		\end{pmatrix}
		=v_0\pqty{z_0}^\top B_1 v_0\pqty{z_1}
	\end{equation}
	
	These facts are a bit too obvious. We want $\Bqty{f\pqty{z},f^\pqty{1}\pqty{z}}$ to be a basis of the set of solutions to get an awesome equation. Thus we require $c_1\neq 0$.
	
	Let
	\begin{equation}
		M=
		\begin{pmatrix}
			c_0 & c_1 \\
			c_1+c_0 \lambda & c_1 \lambda
		\end{pmatrix}
		,
		v_1\pqty{z}=
		\begin{pmatrix}
			f\pqty{z} \\
			f^\pqty{1}\pqty{z}
		\end{pmatrix}
	\end{equation}
	
	Then
	\begin{equation}
		M v_0\pqty{z}=v_1\pqty{z}
		,
		M^{-1}=
		\frac{\begin{pmatrix}
			-c_1 \lambda & c_1 \\
			c_1+c_0\lambda & -c_0
		\end{pmatrix}}{c_1^2}
		,
		v_0\pqty{z}=M^{-1} v_1\pqty{z}
	\end{equation}
	
	Thus, by substitution,
	\begin{equation}
		\begin{pmatrix}
			f\pqty{z_0+z_1}
		\end{pmatrix}
		=\pqty{M^{-1} v_1\pqty{z_0}}^\top B_0 M^{-1} v_1\pqty{z_1}
		,
		\begin{pmatrix}
			f^\pqty{1}\pqty{z_0+z_1}
		\end{pmatrix}
		=\pqty{M^{-1} v_1\pqty{z_0}}^\top B_1 M^{-1} v_1\pqty{z_1}
	\end{equation}
	
	Simplify:
	\begin{equation}
		\begin{pmatrix}
			f\pqty{z_0+z_1}
		\end{pmatrix}
		=\frac{
		\begin{pmatrix}
			f\pqty{z_0} \\
			f^\pqty{1}\pqty{z_0}
		\end{pmatrix}^\top
		\begin{pmatrix}
			-\lambda\pqty{2 c_1+c_0 \lambda} & c_1+c_0 \lambda \\
			c_1+c_0 \lambda & -c_0
		\end{pmatrix}
		\begin{pmatrix}
			f\pqty{z_1} \\
			f^\pqty{1}\pqty{z_1}
		\end{pmatrix}}{c_1^2}
	\end{equation}
	\begin{equation}
		\begin{pmatrix}
			f^\pqty{1}\pqty{z_0+z_1}
		\end{pmatrix}
		=\frac{
		\begin{pmatrix}
			f\pqty{z_0} \\
			f^\pqty{1}\pqty{z_0}
		\end{pmatrix}^\top
		\begin{pmatrix}
			-\lambda^2\pqty{c_1+c_0\lambda} & c_0 \lambda^2 \\
			c_0 \lambda^2 & c_1-c_0\lambda
		\end{pmatrix}
		\begin{pmatrix}
			f\pqty{z_1} \\
			f^\pqty{1}\pqty{z_1}
		\end{pmatrix}}{c_1^2}
	\end{equation}
	
\end{document}
