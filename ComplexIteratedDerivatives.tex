\documentclass[]{article}
%margins
\usepackage[a4paper,margin=0.15in]{geometry}
%document colors
\usepackage{xcolor}
\definecolor{favoritecolor1}{HTML}{607CB2}
\definecolor{favoritecolor2}{HTML}{303E59}
\makeatletter
\newcommand{\globalcolor}[1]{%
	\color{#1}\global\let\default@color\current@color
}
\makeatother

\AtBeginDocument{\globalcolor{favoritecolor2}}
\pagecolor{favoritecolor1}

%made from template named MathArticleTemplate
\usepackage{amsfonts}
\usepackage{amsmath}
\usepackage{amsthm}
\usepackage{amssymb}
\usepackage{hyperref}
\hypersetup{colorlinks=true}
\usepackage{graphics}

%Fix \eqref in section title
\pdfstringdefDisableCommands{\def\eqref#1{(\ref{#1})}}

\DeclareMathOperator{\rues}{rues}
\DeclareMathOperator{\cidf}{cidf}
\DeclareMathOperator{\cidzero}{cidzero}
\DeclareMathOperator{\Cid}{Cid}
\DeclareMathOperator{\cidrues}{cidrues}
\DeclareMathOperator{\ez}{Ez}
\DeclareMathOperator{\gs}{gs}
\DeclareMathOperator{\md}{mod}
\DeclareMathOperator{\pow}{Pow}
%Parenthesis, Braces, Brackets usepackage{physics}
\newcommand{\pqty}[1]{{\left(#1\right)}}
\newcommand{\Bqty}[1]{{\left\{#1\right\}}}
\newcommand{\bqty}[1]{{\left[#1\right]}}
\newcommand{\abs}[1]{{\left\lvert#1\right\rvert}}
%other stuff
\newcommand{\floor}[1]{{\left\lfloor#1\right\rfloor}}
\newcommand{\ceil}[1]{{\left\lceil#1\right\rceil}}
%Laplace transform and inverse
\newcommand{\laplace}[1]{\mathcal{L}\Bqty{#1}\pqty{s}}
\newcommand{\laplaceInv}[1]{\mathcal{L}^{-1}\Bqty{#1}\pqty{t}}
%Derivatives
\newcommand{\pdiff}[2]{\frac{\partial^{#2}}{\partial #1^{#2}}}

%lemma,theorem, proof
\newtheorem{theorem}{Theorem}[section]
\newtheorem{example}{Example}[section]
\newtheorem{lemma}[theorem]{Lemma}
\newtheorem{definition}[theorem]{Definition}
\newtheorem{corollary}[theorem]{Corollary}

\numberwithin{equation}{section}

%\usepackage{minted}
%opening
\author{Mark Andrew Gerads: \href{MailTo:Nazgand@Gmail.Com}{Nazgand@Gmail.Com}}

\title{
	Complex Iterated Derivatives
	
	\href{https://github.com/Nazgand/nazgandMathBook}{https://github.com/Nazgand/nazgandMathBook}
}

\begin{document}
	
	\maketitle
	
	\begin{abstract}
		The goal of this paper is to analyze the iterated derivative extended to a complex number of iterations.
	\end{abstract}
	
	\section{Complex Iterated Derivative For Functions, In General}
	Let us have a function, $\mathbb{C}^2\to\mathbb{C}$, $\cidf\pqty{m,z}$ such that $\cidf\pqty{0,z}=f\pqty{z}$ and $\pdiff{z}{k}\cidf\pqty{m,z}=\cidf\pqty{m+k,z}$. Let $\Cid\pqty{f\pqty{z}}$ be the set of functions with the properties of $\cidf$, so $\cidf\pqty{m,z}\in\Cid\pqty{f\pqty{z}}$.
	
	We can deduce $\infty$ more functions with the same properties as $\cidf$. 
	If we have an arbitrary wave function, $w\pqty{m}=w\pqty{m+1}$, then it produces another function with the properties of $\cidf$:
	\begin{equation}
		\cidf\pqty{m+w\pqty{m}-w\pqty{0},z}\in \Cid\pqty{f\pqty{z}}
	\end{equation}
	\begin{equation}
		\pqty{1+w\pqty{m}-w\pqty{0}}\cidf\pqty{m,z}\in \Cid\pqty{f\pqty{z}}
	\end{equation}
	The previous statement can be simplified:
	\begin{equation}
		\pqty{w\pqty{m}-w\pqty{0}}\cidf\pqty{m,z}\in \Cid\pqty{0}
	\end{equation}
	A simple thing to note is:
	\begin{equation}
		\label{Cid scale function argument}
		k\in\mathbb{Z}^{\geq 0}
		\Rightarrow
		\pdiff{z}{k}\cidf\pqty{m,bz}=b^k\cidf\pqty{m+k,bz}
	\end{equation}
	\eqref{Cid scale function argument} can be restated as
	\begin{equation}
		b^m\cidf\pqty{m,bz}\in\Cid\pqty{f\pqty{bz}}
	\end{equation}
	Note that [the weighted average of a set of functions all having the property of $\cidf$] also has the properties of $\cidf$. Let $\sum_{j\in\mathbb{Z}}c_j=1$ and $\cidf_j\pqty{m,z}\in\Cid\pqty{f\pqty{z}}$. Then
	\begin{equation}
		\sum_{j\in\mathbb{Z}}c_j \cidf_j\pqty{m,z}\in \Cid\pqty{f\pqty{z}}
	\end{equation}
	
	If we has solutions for the complex iterated derivative of 2 function, we can add the solutions to get a solution to the complex iterated derivative of the sum of the functions:
	\begin{equation}
		\bqty{\forall k,\cidf_k\pqty{m,z}\in\Cid\pqty{f_k\pqty{z}}}
		\Rightarrow
		\cidf_1\pqty{m,z}+\cidf_2\pqty{m,z}\in\Cid\pqty{f_1\pqty{z}+f_2\pqty{z}}
	\end{equation}
	
	The complications of multiple solutions can be shifted by adding the set $\Cid\pqty{0}$ to a single solution:
	\begin{equation}
		\Cid\pqty{f\pqty{z}}=
		\Bqty{\cidf\pqty{m,z}+\cidzero\pqty{m,z} \mid \cidzero\pqty{m,z}\in\Cid\pqty{0}}
	\end{equation}
	I thus declare the shorthand notation, reminiscent of the constant of integration:
	\begin{equation}
		\pdiff{z}{m}f\pqty{z}=\cidf\pqty{m,z}+\Cid\pqty{0}
	\end{equation}
	
	Another useful property is
	\begin{equation}
		\label{cid m+k}
		\cidf\pqty{m+k,z}\in\Cid\pqty{\cidf\pqty{k,z}}
	\end{equation}

	\section{Complex Iterated Derivative For Specific Functions}
	First, the simplest solution to the complex iterated derivative of the exponential function:
	\begin{equation}
		e^z\in \Cid\pqty{e^z}
	\end{equation}
	Thus, for an arbitrary exponential function, using \eqref{Cid scale function argument}, we can say:
	\begin{equation}
		\ln\pqty{t}^m t^z\in \Cid\pqty{t^z}
	\end{equation}
	Thus, with the linearity of the complex iterated derivative, we have a way to evaluate the complex iterated derivative of arbitrary linear combinations of exponential functions. Example, where the sums converge or are continued analytically:
	\begin{equation}
		\zeta\pqty{z}-1=\sum_{t=2}^{\infty}t^{-z}
		\Rightarrow
		\sum_{t=2}^{\infty}\ln\pqty{t^{-1}}^m t^{-z}\in\Cid\pqty{\zeta\pqty{z}-1}
	\end{equation}
	We might want to add $1$ to get $\zeta\pqty{z}$, so next is $\Cid\pqty{1}$:
	\begin{equation}
		\frac{z^{-m}}{\Gamma\pqty{1-m}}\in\Cid\pqty{1}
	\end{equation}
	For $k\in\mathbb{Z}^+$ and an arbitrary constant $c$,
	\begin{equation}
		\frac{cz^{-k-m}}{\Gamma\pqty{1-k-m}}\in\Cid\pqty{0}
	\end{equation}
	From \eqref{cid m+k} and $\Cid\pqty{1}$, the power functions are obtained:
	\begin{equation}
		\frac{z^{-m-k}}{\Gamma\pqty{1-m-k}}\in\Cid\pqty{\frac{z^{-k}}{\Gamma\pqty{1-k}}}
	\end{equation}
	As can be seen by dividing by $\Gamma\pqty{1-k}$ and substituting $k\to -k$, in the specific case where the left side of the following equation is not an indeterminate form:
	\begin{equation} \frac{\Gamma\pqty{1+k}z^{k-m}}{\Gamma\pqty{1+k-m}}\in\Cid\pqty{z^{k}}
	\end{equation}
	The linearity of complex iterated derivatives combines with the linearity of integrals so that for an arbitrary function $g$ and arbitrary constants $a,b$, where the integral converges:
	\begin{equation}
		\int_{a}^{b}g\pqty{t}\ln\pqty{t}^m t^z dt
		\in\Cid\pqty{\int_{a}^{b}g\pqty{t}t^z dt}
	\end{equation}
	
	\begin{example}
		Letting $g\pqty{t}=e^{-t},a=0,b=\infty$, the complex iterated derivative of the gamma function is found where $z\in\mathbb{C}\land -z\not\in\mathbb{Z}^+$:
		\begin{equation}
			\Gamma\pqty{z+1} = \int_{0}^{\infty}e^{-t}t^z dt
			\Rightarrow
			\int_{0}^{\infty}e^{-t}\ln\pqty{t}^m t^z dt\in\Cid\pqty{\Gamma\pqty{z+1}}
		\end{equation}
	\end{example}
	
	\section{Relation to earlier work}
	Where \(n\in\mathbb{Z}^+\land z\in\mathbb{C}\):
	\begin{equation}
		\cidrues_n\pqty{0,z}=
		\rues_n\pqty{z}=
		\sum_{k=0}^{\infty}\frac{z^{nk}}{\pqty{nk}!}=
		\frac{1}{n}\sum _{k=1}^n \exp\pqty{ze^{2ki\pi/n}}
	\end{equation}
	
	The reason these functions are named $\cidrues_n$ is because it is an acronym for Complex Iterated Derivative Root of Unity Exponential Sum function.
	\begin{definition}
		Where $n\in\mathbb{Z}^+,\Bqty{m,z}\subset\mathbb{C}$:
		\begin{equation}
			\label{cidrues Exponential sum form}
			\cidrues_n\pqty{m,z}=
			\frac{1}{n}\sum _{k=1}^n \exp\pqty{ze^{2ki\pi/n}+2mki\pi/n}
			\in\Cid\pqty{\rues_n\pqty{z}}
		\end{equation}
	\end{definition}
	The functions are periodic:
	\begin{equation}
		\cidrues_n\pqty{m,x}=\cidrues_n\pqty{m+n,x}
	\end{equation}

	Thus there exists some functions $c_{n,k}\pqty{m}$ relatively constant to $z$ such that
	\begin{equation}
		\cidrues_n\pqty{m,z}=\sum_{k=0}^{n-1}c_{n,k}\pqty{m}\cidrues_n\pqty{k,z}
	\end{equation}
	
\end{document}
