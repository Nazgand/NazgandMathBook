\documentclass[]{article}
%made from template named MathArticleTemplate
\usepackage{amsfonts}
\usepackage{amsmath}
\usepackage{amsthm}
\usepackage{amssymb}
\usepackage{hyperref}
\hypersetup{colorlinks=true}
\usepackage{graphics}

%Fix \eqref in section title
\pdfstringdefDisableCommands{\def\eqref#1{(\ref{#1})}}

\DeclareMathOperator{\es}{Es}
\DeclareMathOperator{\ez}{Ez}
\DeclareMathOperator{\gs}{gs}
\DeclareMathOperator{\md}{mod}
\DeclareMathOperator{\pow}{Pow}
%Parenthesis, Braces, Brackets usepackage{physics}
\newcommand{\pqty}[1]{{\left(#1\right)}}
\newcommand{\Bqty}[1]{{\left\{#1\right\}}}
\newcommand{\bqty}[1]{{\left[#1\right]}}
\newcommand{\abs}[1]{{\left\lvert#1\right\rvert}}
%other stuff
\newcommand{\floor}[1]{{\left\lfloor#1\right\rfloor}}
\newcommand{\ceil}[1]{{\left\lceil#1\right\rceil}}
%Laplace transform and inverse
\newcommand{\laplace}[1]{\mathcal{L}\Bqty{#1}\pqty{s}}
\newcommand{\laplaceInv}[1]{\mathcal{L}^{-1}\Bqty{#1}\pqty{t}}
%Derivatives
\newcommand{\pdiff}[2]{\frac{\partial^{#2}}{\partial #1^{#2}}}

%lemma,theorem, proof
\newtheorem{theorem}{Theorem}[section]
\newtheorem{lemma}[theorem]{Lemma}
\newtheorem{definition}[theorem]{Definition}
\newtheorem{corollary}[theorem]{Corollary}

\numberwithin{equation}{section}

%\usepackage{minted}
%opening
\author{Mark Andrew Gerads: MarkAndrewGerads.Nazgand@Gmail.Com}

\title{Notation}

\begin{document}
	
	\maketitle
	
	\begin{abstract}
		The goal of this paper is to clarify notation.
	\end{abstract}
	
	\section{Substitution}
	This works similarly to Mathematica's Replace function. Example:
	\begin{equation}
	\pqty{\pdiff{x}{}x^2:x\to a}=
	\pqty{2x:x\to a}=2a
	\end{equation}
	
	\url{https://reference.wolfram.com/language/ref/Replace.html}
	
	\section{Logic}
	
	And: Given statements \(A\) and \(B\). The statement \(A\land B\) means that both statements \(A\) and \(B\) are true.
	
	Or: Given statements \(A\) and \(B\). The statement \(A\lor B\) means that at least 1 of the statements \(A\) or \(B\) is not false.
	
	Implies: The statement \(A\Rightarrow B\) is a conditional statement stating that if A is true, then B is true. 
	
	\(A\Leftrightarrow B\) means \(\bqty{A\Rightarrow B\land B\Rightarrow A}\).
	
	Brackets: Normally, logic is read from left to right, yet sometimes brackets are used to change the order or add clarification. Example: \(A\lor\bqty{B\land C}\).
	
	Exists: \(\exists a\) means that some a exists.
	
	\(\exists a\land P\pqty{a}\) can be thought as \(\Bqty{}\neq\{a\mid P\pqty{a}\}\).
	
	\url{https://en.wikipedia.org/wiki/List_of_logic_symbols}
	
	Kronecker delta function:
	\begin{equation}
	\delta\pqty{0}=1
	\land
	\bqty{x\neq 0 \Rightarrow \delta\pqty{x}=0}
	\end{equation}
	\url{https://en.wikipedia.org/wiki/Kronecker_delta}
	
	\section{Order and Equality}
	Equals: \(a=b\) means that \(a\) is equal to \(b\).
	
	\url{https://en.wikipedia.org/wiki/Equality_(mathematics)}
	
	Greater than: \(a>b\) means that \(a\) is greater than \(b\).
	\(a\geq b\) means that \(a\) is greater than or equal to \(b\).
	
	Less than: \(a<b\) means that \(a\) is less than \(b\).
	\(a\leq b\) means that \(a\) is less than or equal to \(b\).
	
	\url{https://en.wikipedia.org/wiki/Inequality_(mathematics)}
	
	\section{Sets}
	Sets: Sets either have an element or they do not have the element; no element is listed more than 1 time. Example sets are: \(\Bqty{1,2,3}, \Bqty{1,\Bqty{1,a,b}}\).
	
	Element of: \(\in\) means "is an element of". Examples: \(1\in\Bqty{1,2,3}\) and \(\Bqty{1,a,b}\in\Bqty{1,\Bqty{1,a,b}}\).
	
	Subset: \(\subseteq\) means "is a subset of". Examples: \(\Bqty{1,2,3}\subseteq\Bqty{1,2,3}\) and \(\Bqty{3}\subseteq\Bqty{1,2,3}\) and \(\Bqty{}\subseteq\Bqty{1,2,3}\).
	
	Union: \(A\cup B\) is the minimal set which contains all elements either \(A\) or \(B\) contain.
	
	Intersection: \(A\cap B\) is the set which contains all elements both \(A\) and \(B\) contain.
	
	\url{https://en.wikipedia.org/wiki/Set_(mathematics)}
	
	Set builder notation:
	
	The set of all things \(a\) which satisfy the constraining statement \(P\pqty{a}\):
	\begin{equation}
	\Bqty{a\mid P\pqty{a}}
	\end{equation}
	
	The set of all things \(a\) in the set \(A\) which satisfy the statement \(P\pqty{a}\):
	\begin{equation}
	\Bqty{a\in A\mid P\pqty{a}}
	\end{equation}
	
	\url{https://en.wikipedia.org/wiki/Set-builder_notation}
	
	Integers:
	\begin{equation}
	0\in\mathbb{Z}\land\bqty{a\in\mathbb{Z}\Leftrightarrow \pqty{a+1}\in\mathbb{Z}}
	\end{equation}
	\begin{equation}
	\mathbb{Z}^+=\Bqty{n\in\mathbb{Z}\mid n>0}
	\end{equation}
	\begin{equation}
	\mathbb{Z}^{\geq 0}=\Bqty{n\in\mathbb{Z}\mid n\geq 0}
	\end{equation}
	
	\url{https://en.wikipedia.org/wiki/Integer}
	
	\url{https://en.wikipedia.org/wiki/Natural_number#Notation}
	
	Rational numbers:
	\begin{equation}
	\mathbb{Q}=\Bqty{\frac{a}{b}\mid a\in\mathbb{Z}\land b\in\mathbb{Z}^+}
	\end{equation}
	\begin{equation}
	\mathbb{Q}^+=\Bqty{q\in\mathbb{Q}\mid q>0}
	\end{equation}
	\begin{equation}
	\mathbb{Q}^{\geq 0}=\Bqty{q\in\mathbb{Q}\mid q\geq 0}
	\end{equation}
	
	\url{https://en.wikipedia.org/wiki/Rational_number}
	
	Real numbers:
	\begin{equation}
	\mathbb{R}^{\geq 0}=\Bqty{\inf A\mid A\subseteq\mathbb{Q}^+\land A\neq \Bqty{}}
	\end{equation}
	\begin{equation}
	\mathbb{R}=\Bqty{a-b\mid \Bqty{a,b}\subseteq\mathbb{R}^{\geq 0}}
	\end{equation}
	\begin{equation}
	\mathbb{R}^+=\Bqty{a\in\mathbb{R}^{\geq 0}\mid a>0}
	\end{equation}
	
	\url{https://en.wikipedia.org/wiki/Real_number}
	
	\section{Sigma summation}
	\begin{equation}
	\sum_{k\in\Bqty{}}f\pqty{k}=0
	\end{equation}
	\begin{equation}
	\sum_{k\in\Bqty{a}}f\pqty{k}=f\pqty{a}
	\end{equation}
	\begin{equation}
	\sum_{k\in\Bqty{S_1\cup S_2}}f\pqty{k}
	=
	\pqty{\sum_{k\in\Bqty{S_1}}f\pqty{k}}
	+
	\pqty{\sum_{k\in\Bqty{S_2}}f\pqty{k}}
	-
	\pqty{\sum_{k\in\Bqty{S_1\cap S_2}}f\pqty{k}}
	\end{equation}
	Shorthand
	\begin{equation}
	\sum_{k={a}}^{a-1} f\pqty{k}=0
	\end{equation}
	\begin{equation}
	\sum_{k={a}}^{b+1} f\pqty{k}=
	\pqty{\sum_{k={a}}^{b} f\pqty{k}}+f\pqty{b+1}=
	f\pqty{a}+{\sum_{k={a+1}}^{b} f\pqty{k}}
	\end{equation}
	
	\section{Calculus}
	A limit \(\lim\limits_{a\to b}f\pqty{a}\) is the value usually approaching \(f\pqty{b}\)
	
	\(n\)th derivative, not to be confused with \(f^{n}\pqty{x}\):
	\begin{equation}
	f^{(n)}\pqty{x}
	=
	\pdiff{x}{n}f\pqty{x}
	\end{equation}
	
	\url{https://en.wikipedia.org/wiki/Derivative#Leibniz's_notation}
	
	Integrals can be thought as the area under a curve. My notation differs slightly by using \(\partial\) instead of \(d\). The integral from \(a\) to \(b\) of function \(f\pqty{x}\) with respect to \(x\) is:
	
	
	\begin{equation}
	\int_{a}^{b} f\pqty{x}\partial x
	\end{equation}
	
	Laplace transform:
	\begin{equation}
	\laplace{f\pqty{t}}=
	\int_{t=0}^{\infty}f\pqty{t}e^{-st}\partial t
	\end{equation}
	
	Laplace transform inverse:
	\begin{equation}
	\laplaceInv{\laplace{f\pqty{t}}}=f\pqty{t}
	\end{equation}
	
	\url{https://en.wikipedia.org/wiki/Laplace_transform#Formal_definition}
	
	Modular arithmetic function, $\md:\pqty{\mathbb{C},\mathbb{C}}\rightarrow\mathbb{C}$
	\begin{equation}
	0\leq\Re\pqty{\frac{\md\pqty{a,b}}{b}}<1
	\land
	\md\pqty{a,b}=\md\pqty{a+b,b}
	\end{equation}
	
\end{document}
