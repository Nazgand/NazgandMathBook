\documentclass[]{article}
%margins
\usepackage[a4paper,margin=0.15in]{geometry}
%document colors
\usepackage{xcolor}
\definecolor{favoritecolor1}{HTML}{607CB2}
\definecolor{favoritecolor2}{HTML}{303E59}
\makeatletter
\newcommand{\globalcolor}[1]{%
	\color{#1}\global\let\default@color\current@color
}
\makeatother

\AtBeginDocument{\globalcolor{favoritecolor2}}
\pagecolor{favoritecolor1}

%made from template named MathArticleTemplate
\usepackage{amsfonts}
\usepackage{amsmath}
\usepackage{amsthm}
\usepackage{amssymb}
\usepackage{hyperref}
\hypersetup{colorlinks=true}
\usepackage{graphics}

%Fix \eqref in section title
\pdfstringdefDisableCommands{\def\eqref#1{(\ref{#1})}}

\DeclareMathOperator{\es}{Es}
\DeclareMathOperator{\ez}{Ez}
\DeclareMathOperator{\gs}{gs}
\DeclareMathOperator{\md}{mod}
\DeclareMathOperator{\pow}{Pow}
%Parenthesis, Braces, Brackets usepackage{physics}
\newcommand{\pqty}[1]{{\left(#1\right)}}
\newcommand{\Bqty}[1]{{\left\{#1\right\}}}
\newcommand{\bqty}[1]{{\left[#1\right]}}
\newcommand{\abs}[1]{{\left\lvert#1\right\rvert}}
%other stuff
\newcommand{\floor}[1]{{\left\lfloor#1\right\rfloor}}
\newcommand{\ceil}[1]{{\left\lceil#1\right\rceil}}
%Laplace transform and inverse
\newcommand{\laplace}[1]{\mathcal{L}\Bqty{#1}\pqty{s}}
\newcommand{\laplaceInv}[1]{\mathcal{L}^{-1}\Bqty{#1}\pqty{t}}
%Derivatives
\newcommand{\pdiff}[2]{\frac{\partial^{#2}}{\partial #1^{#2}}}
%Kettenbruch
\newcommand{\ketten}[4]{\underset{#1}{\overset{#2}{\LARGE\mathrm K}}\frac{#3}{#4}}

%lemma,theorem, proof
\newtheorem{theorem}{Theorem}[section]
\newtheorem{lemma}[theorem]{Lemma}
\newtheorem{definition}[theorem]{Definition}
\newtheorem{corollary}[theorem]{Corollary}

\numberwithin{equation}{section}

%\usepackage{minted}
%opening
\author{Mark Andrew Gerads: \href{MailTo:Nazgand@Gmail.Com}{Nazgand@Gmail.Com}}

\title{
	Miscellaneous Trivialities
	
	\href{https://github.com/Nazgand/nazgandMathBook}{https://github.com/Nazgand/nazgandMathBook}
}

\begin{document}
	
	\maketitle
	
	\begin{abstract}
		The goal of this paper is to have fun with trivial facts.
	\end{abstract}
	
	
	\begin{equation}
		\mathbb{R} = \Bqty{x_0-x_1 \mid k\in\mathbb{Z}\Rightarrow x_k\in\mathbb{R}^+}
	\end{equation}
	
	\begin{equation}
		\bqty{k\in\mathbb{Z}^+\Rightarrow v_k-1\in\mathbb{Z}^+}
		\Rightarrow
		\Bqty{x \mid x\in\mathbb{R},0\leq x,x\leq 1}=
		\Bqty{\sum_{k\in\mathbb{Z}^+} d_k \prod_{m=1}^{k} v_k^{-1} \mid \bqty{k\in\mathbb{Z}^+\Rightarrow d_k+1\in\mathbb{Z}^+},d_k<v_k}
	\end{equation}

	\begin{equation}
		\mathbb{C} = \Bqty{x_0\exp\pqty{ix_1} + x_2\exp\pqty{ix_3} \mid \bqty{k\in\mathbb{Z}\Rightarrow x_k\in\mathbb{R}}, \frac{x_1-x_3}{\pi}\notin\mathbb{Z}}
	\end{equation}
	
	\begin{equation}
		\Bqty{m_0 + m_2\exp\pqty{\frac{i\pi}{3}} \mid \bqty{k\in\mathbb{Z}\Rightarrow m_k\in\mathbb{Z}}}
		=
		\Bqty{m_0 + m_2\exp\pqty{\frac{i2\pi}{3}} \mid \bqty{k\in\mathbb{Z}\Rightarrow m_k\in\mathbb{Z}}}
	\end{equation}

\end{document}
