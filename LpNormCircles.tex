\documentclass[]{article}
%margins
\usepackage[a4paper,margin=0.15in]{geometry}
%document colors
\usepackage{xcolor}
\definecolor{favoritecolor1}{HTML}{607CB2}
\definecolor{favoritecolor2}{HTML}{303E59}
\makeatletter
\newcommand{\globalcolor}[1]{%
	\color{#1}\global\let\default@color\current@color
}
\makeatother

\AtBeginDocument{\globalcolor{favoritecolor2}}
\pagecolor{favoritecolor1}

\usepackage{amsfonts}
\usepackage{amsmath}
\usepackage{amssymb}
\usepackage{hyperref}
\hypersetup{colorlinks=true}
\usepackage{graphics}
\usepackage{mathrsfs}

\DeclareMathOperator{\md}{mod}
\DeclareMathOperator{\es}{Es}
\DeclareMathOperator{\dcc}{dcc}
%Parenthesis, Braces, Brackets
\newcommand{\paren}[1]{{\left(#1\right)}}
\newcommand{\Brace}[1]{{\left\{#1\right\}}}
\newcommand{\Brack}[1]{{\left[#1\right]}}
\newcommand{\abs}[1]{{\left\lvert#1\right\rvert}}
\newcommand{\floor}[1]{{\left\lfloor#1\right\rfloor}}
\newcommand{\ceil}[1]{{\left\lceil#1\right\rceil}}
%Laplace transform and inverse
\newcommand{\laplace}[1]{\mathscr{L}\Brace{#1}\paren{s}}
\newcommand{\laplaceInv}[1]{\mathscr{L}^{-1}\Brace{#1}\paren{t}}
%Derivatives
\newcommand{\pdiff}[2]{\frac{\partial^{#2}}{\partial #1^{#2}}}
%\usepackage{minted}
%opening
\author{Mark Andrew Gerads: \href{MailTo:Nazgand@Gmail.Com}{Nazgand@Gmail.Com}}

\title{
	$L^p$ Planar Algebraic Geometry
	
	\href{https://github.com/Nazgand/nazgandMathBook}{https://github.com/Nazgand/nazgandMathBook}
}

\begin{document}

\maketitle

\begin{abstract}

\end{abstract}

\section{The $L^p$ norm for $p\in\mathbb{R},p\geq 1$}
The distance between points $\paren{x_1,y_1}$ and $\paren{x_1,y_1}$ is:
\begin{equation}
\sqrt[p]{\abs{x_1-x_2}^p+\abs{y_1-y_2}^p}
\end{equation}

This has the triangle inequality:
\begin{equation}
\sqrt[p]{\abs{x_1-x_2}^p+\abs{y_1-y_2}^p}+
\sqrt[p]{\abs{x_2-x_3}^p+\abs{y_2-y_3}^p}\geq
\sqrt[p]{\abs{x_1-x_3}^p+\abs{y_1-y_3}^p}
\end{equation}

Euclidean translations and scalings of objects preserves shape in $L^p$ space. Euclidean rotations do not generally preserve shape.

\section{Lengths of curves}
A curve defined by a function $y=f\paren{x}$ between $a$ and $b$ has the length
\begin{equation}
\lim\limits_{\Delta x\rightarrow 0^+}\sum_{k=0}^{\floor{\paren{b-a}/\Delta x}}
\sqrt[p]{\paren{\Delta x}^p+\abs{f\paren{a+\paren{k+1}\Delta x}-f\paren{a+k\Delta x}}^p}
\end{equation}
where the limit exists. This is equivalent to
\begin{equation}
\int_{a}^{b}
\sqrt[p]{1+\abs{f'\paren{x}}^p} \,dx
\end{equation}

In polar coordinates,
\begin{equation}
\int_{a}^{b}
\sqrt[p]{\abs{\cos\paren{\theta}r'\paren{\theta}-\sin\paren{\theta}r\paren{\theta}}^p
	+\abs{\sin\paren{\theta}r'\paren{\theta}+\cos\paren{\theta}r\paren{\theta}}^p
} \,d\theta
\end{equation}
\begin{equation}
\label{CurveLengthPolar2}
\int_{a}^{b}
\abs{r\paren{\theta}}
\sqrt[p]{\abs{\cos\paren{\theta}\frac{r'\paren{\theta}}{r\paren{\theta}}-\sin\paren{\theta}}^p
	+\abs{\sin\paren{\theta}\frac{r'\paren{\theta}}{r\paren{\theta}}+\cos\paren{\theta}}^p
} \,d\theta
\end{equation}

\section{Area}
If for all disjoint regions $R_0,R_1$, if $\text{Area}\paren{R_0}+\text{Area}\paren{R_1}=\text{Area}\paren{R_0\cup R_1}$ and for all $R_2$ which is a Euclidean translation of $R_0$, $\text{Area}\paren{R_0}=\text{Area}\paren{R_2}$, then Area needs to be proportional to Euclidean area. Thus Euclidean area is used.

\section{Circles}
In the $L^p$ norm, a circle which is the set of points distance $r_p$ from the point $\paren{x_0,y_0}$ is.
\begin{equation}
\abs{x-x_0}^p+\abs{y-y_0}^p=r_p^p
\end{equation}

The shape of circles is scale invariant:
\begin{equation}
\abs{\frac{x}{r_p}}^p+\abs{\frac{y}{r_p}}^p=1
\end{equation}

Thus this work concentrates the unit circle in the first quadrant $x,y\in\mathbb{R}^+$, the other quadrants are reflections and not needing absolute value signs is nice during calculation. Thus what will be used is:
\begin{equation}
x^p+y^p=1
\end{equation}

An important point for symmetry: $x=y=2^\paren{-1/p}$.


Let $\tau_p$ be the circumference of a unit circle in $L^p$. $\theta_p\paren{\theta}$ is the function that translate the Euclidean angle to the $L^p$ angle. 
\begin{equation}
\theta_p\paren{\theta}=-\theta_p\paren{-\theta}=\theta_p\paren{\theta+\frac{\pi}{2}}-\frac{\tau_p}{4}
\end{equation}

In the first quadrant of the unit circle, 
\begin{equation}
1\geq x^p\Rightarrow y=\sqrt[p]{1-x^p}
\end{equation}

Area is important. Ratios between area and distance keep constant as do angles for scaling centered at the origin and rotations in $L^p$ space. For $p>0$,
\begin{equation}
\int_0^1 \paren{1-x^p}^{1/p} \, dx
=
\frac{\Gamma \left(1+\frac{1}{p}\right)^2}{\Gamma \left(1+\frac{2}{p}\right)}
\end{equation}

and polar coordinates with respect to the Euclidean angle and Euclidean norm can be found via substituting $x=r_p\cos\paren{\theta},y=r_p\sin\paren{\theta}$
\begin{equation}
1=r_p^p\paren{\abs{\cos\paren{\theta}}^p+\abs{\sin\paren{\theta}}^p}
\land
r_p=\paren{\abs{\cos\paren{\theta}}^p+\abs{\sin\paren{\theta}}^p}^{-1/p}
\end{equation}

In the first quadrant:
\begin{equation}
0\leq \theta \leq \frac{\pi}{2}
\Rightarrow
r_p\paren{\theta}=\paren{{\cos\paren{\theta}}^p+{\sin\paren{\theta}}^p}^{-1/p}
\end{equation}
\begin{equation}
\frac{r'\paren{\theta}}{r\paren{\theta}}=\frac{{\tan\paren{\theta}\cos\paren{\theta}^p-\cot\paren{\theta}\sin\paren{\theta}^p}}
{\cos\paren{\theta}^p+\sin\paren{\theta}^p}
\end{equation}
Useing \eqref{CurveLengthPolar2},
\begin{equation}
\tau_p=4\int_{0}^{\pi/2}
\sqrt[p]{ \paren{{\cos\paren{\theta}}^p+{\sin\paren{\theta}}^p}
	\paren{
		\abs{\cos\paren{\theta}\frac{{\tan\paren{\theta}\cos\paren{\theta}^p-\cot\paren{\theta}\sin\paren{\theta}^p}}
			{\cos\paren{\theta}^p+\sin\paren{\theta}^p}-\sin\paren{\theta}}^p
	+\abs{\sin\paren{\theta}\frac{{\tan\paren{\theta}\cos\paren{\theta}^p-\cot\paren{\theta}\sin\paren{\theta}^p}}
		{\cos\paren{\theta}^p+\sin\paren{\theta}^p}+\cos\paren{\theta}}^p
}} \,d\theta
\end{equation}


An important function is a scaling function based on the Euclidean angle.
\begin{equation}
s_p\paren{\theta}=\paren{\abs{\cos\paren{\theta}}^p+\abs{\sin\paren{\theta}}^p}^{1/p}
\end{equation}

In the $L^p$ norm, a formula for arc length is
\begin{equation}
\int_{a}^{b} s_p\paren{\arctan\paren{\frac{\partial y}{\partial x}}} \sqrt{1+\paren{\frac{\partial y}{\partial x}}^2}\,dx
\end{equation}

In the $L^p$ norm, a formula for arc length in Euclidean polar coordinates is
\begin{equation}
\int_{a}^{b} 
s_p\paren{\arctan\paren{\frac{\sin\paren{\theta}\frac{\partial r}{\partial \theta}+r\cos\paren{\theta}}{\cos\paren{\theta}\frac{\partial r}{\partial \theta}-r\sin\paren{\theta}}}}
\sqrt{r^2+\paren{\frac{\partial r}{\partial \theta}}^2}\,d\theta
\end{equation}

The circumference can be calculated:
\begin{equation}
\frac{\partial y}{\partial x}=\frac{\partial}{\partial x}\sqrt[p]{1-x^p}
=\frac{1}{p}\paren{1-x^p}^{\frac{1}{p}-1}*-px^{p-1}
=-\paren{1-x^p}^{\frac{1}{p}-1}x^{p-1}
\end{equation}
\begin{equation}
\tau_p=4\int_{0}^{1} \sqrt[p]{1+\abs{\paren{1-x^p}^{\frac{1}{p}-1}x^{p-1}}^p} \,dx
=
4\int_{0}^{1} \sqrt[p]{1+{\paren{1-x^p}^{1-p}x^{p^2-p}}} \,dx
\end{equation}




\begin{equation}
\tau_p=
4\int_{0}^{1} \frac{\sqrt[p]{\paren{1-x^p}\paren{\paren{1-x^p}^{p-1}+x^{p^2-p}}}}{1-x^p} \,dx
=
\frac{4}{p}\int_{0}^{1} \frac{\sqrt[p]{z\paren{1-z}\paren{\paren{1-z}^{p-1}+z^{p-1}}}}{\paren{1-z}z} \,dz
\end{equation}
\begin{equation}
\tau_p=
\frac{8}{p}\int_{0}^{1/2} \frac{\sqrt[p]{z\paren{1-z}\paren{\paren{1-z}^{p-1}+z^{p-1}}}}{\paren{1-z}z} \,dz
\end{equation}

Case $p=3$: Choose $\paren{1-z}^2+z^2=y$, $\frac{\partial y}{\partial z}=4z-2$, $z\paren{1-z}=\frac{1-y}{2}$
\begin{equation}
\tau_p=
\frac{16\sqrt[3]{4}}{3}
\int_{1/2}^{1} y \paren{y\paren{1-y}}^{-2/3} \sqrt{2y-1}\,dy
\end{equation}



Choose $z=\frac{1}{1+e^{-x}}$, $\frac{\partial x}{\partial z}=\frac{1}{z\paren{1-z}}$
\begin{equation}
\tau_p=
\frac{8}{p}\int_{0}^{\infty} \sqrt[p]{ \frac{e^{x}+e^{p {x}}}{\paren{1+e^{x}}^{p+1}} } \,dx
\end{equation}

Choose $z=e^{x}$, $\frac{\partial x}{\partial z}=\frac{1}{z}$
\begin{equation}
\tau_p=
\frac{8}{p}\int_{1}^{\infty} \frac{1}{z\paren{1+z}}\sqrt[p]{ \frac{z+z^{p}}{1+z} } \,dz
\end{equation}

Choose $x=1+z^{p-1}$, $\paren{x-1}^{1/\paren{p-1}}=z$, $\frac{\partial x}{\partial z}=\frac{1}{z}$
\begin{equation}
\tau_p=
\frac{8}{p}\int_{1}^{\infty} \frac{1}{z\paren{1+z}}\sqrt[p]{ \frac{z+z^{p}}{1+z} } \,dz
\end{equation}

Some exact values:
\begin{equation}
\tau_3=\frac{2}{3} \Gamma \paren{\frac{1}{3}} \paren{\frac{\Gamma \paren{\frac{1}{4}}}{\Gamma \paren{\frac{7}{12}}}+\frac{\Gamma \paren{\frac{3}{4}}}{\Gamma \paren{\frac{13}{12}}}}, \tau_2=2\pi,\tau_1=8
\end{equation}


\section{Regular polygons and circles around circles}
Given a point at Euclidean angle $\theta$ a function is desired which gives the angle for the point distance $0\leq d\leq 2$ counter-clockwise from the point at angle $\theta$, and nesting of this function is desired to construct regular polygons.
\begin{equation}
\theta\leq\dcc\paren{\theta,d,p,1}\leq\theta+\pi
\end{equation}
\begin{align}
\paren{\Delta x}= &&
r_p\paren{\dcc\paren{\theta,d,p,1}}\cos\paren{\dcc\paren{\theta,d,p,1}}-r_p\paren{\theta}\cos\paren{\theta}\\
\paren{\Delta y}= &&
r_p\paren{\dcc\paren{\theta,d,p,1}}\sin\paren{\dcc\paren{\theta,d,p,1}}-r_p\paren{\theta}\sin\paren{\theta}\\
d= &&
\sqrt[p]{\paren{\Delta x}^p+\paren{\Delta y}^p}\\
\dcc\paren{\theta,d,p,0}
&& =\theta
\\
\dcc\paren{\theta,d,p,n}
&& =\theta -\md\paren{\theta,\frac{\pi}{2}}+\dcc\paren{\md\paren{\theta,\frac{\pi}{2}},d,p,n}
\\
&& =\dcc\paren{\dcc\paren{\theta,d,p,n-m},d,p,m}
\\
&& =-\dcc\paren{-\theta,d,p,-n}
\end{align}

Interestingly, regular hexagons have a constant side regardless of angle and $p$.
\begin{align}
\dcc\paren{\theta,1,p,6}
&& =\theta + 2\pi
\\
\dcc\paren{\theta,2,p,2}
&& =\theta + 2\pi
\end{align}



\section{Parabola for $p>1$}
Each point on a parabola has the same distance from the focus as from the directrix. With a directrix of $y=0$ and a focus at $\paren{0,2a}, a\in\mathbb{R}^+$, we have $f\paren{x}=f\paren{-x},f\paren{0}=a,f\paren{2a}=2a$. Also,
\begin{equation}
\abs{x}^p+\abs{f\paren{x}-2a}^p=\abs{f\paren{x}}^p
\end{equation}

$x\geq 0$ WLOG. 2 cases exist. $f\paren{x}\geq 2a$:
\begin{equation}
x^p+\paren{f\paren{x}-2a}^p={f\paren{x}}^p
\end{equation}

and $f\paren{x}\leq 2a$:
\begin{equation}
x^p+\paren{2a-f\paren{x}}^p={f\paren{x}}^p
\end{equation}

Both cases are equivalent when $2 \vert p$.


\end{document}
