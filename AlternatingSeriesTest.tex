\documentclass[]{article}
%document colors
\usepackage{xcolor}
\definecolor{favoritecolor1}{HTML}{607CB2}
\definecolor{favoritecolor2}{HTML}{303E59}
\makeatletter
\newcommand{\globalcolor}[1]{%
	\color{#1}\global\let\default@color\current@color
}
\makeatother

\AtBeginDocument{\globalcolor{favoritecolor2}}
\pagecolor{favoritecolor1}

%made from template named MathArticleTemplate
\usepackage{amsfonts}
\usepackage{amsmath}
\usepackage{amsthm}
\usepackage{amssymb}
\usepackage{hyperref}
\hypersetup{colorlinks=true}
\usepackage{graphics}

%Fix \eqref in section title
\pdfstringdefDisableCommands{\def\eqref#1{(\ref{#1})}}

\DeclareMathOperator{\es}{Es}
\DeclareMathOperator{\ez}{Ez}
\DeclareMathOperator{\gs}{gs}
\DeclareMathOperator{\md}{mod}
\DeclareMathOperator{\pow}{Pow}
%Parenthesis, Braces, Brackets usepackage{physics}
\newcommand{\pqty}[1]{{\left(#1\right)}}
\newcommand{\Bqty}[1]{{\left\{#1\right\}}}
\newcommand{\bqty}[1]{{\left[#1\right]}}
\newcommand{\abs}[1]{{\left\lvert#1\right\rvert}}
%other stuff
\newcommand{\floor}[1]{{\left\lfloor#1\right\rfloor}}
\newcommand{\ceil}[1]{{\left\lceil#1\right\rceil}}
%Laplace transform and inverse
\newcommand{\laplace}[1]{\mathcal{L}\Bqty{#1}\pqty{s}}
\newcommand{\laplaceInv}[1]{\mathcal{L}^{-1}\Bqty{#1}\pqty{t}}
%Derivatives
\newcommand{\pdiff}[2]{\frac{\partial^{#2}}{\partial #1^{#2}}}

%lemma,theorem, proof
\newtheorem{theorem}{Theorem}[section]
\newtheorem{lemma}[theorem]{Lemma}
\newtheorem{definition}[theorem]{Definition}
\newtheorem{corollary}[theorem]{Corollary}

\numberwithin{equation}{section}

%\usepackage{minted}
%opening
\author{Mark Andrew Gerads: \href{MailTo:MarkAndrewGerads.Nazgand@Gmail.Com}{MarkAndrewGerads.Nazgand@Gmail.Com}}

\title{
	Alternative series test
	
	\href{https://github.com/Nazgand/nazgandMathBook}{https://github.com/Nazgand/nazgandMathBook}
}

\begin{document}
	
	\maketitle
	\section{Alternating series with monotone decreasing absolute value}
	
	\begin{theorem}
		Let a sequence \(a_n\) exist such that \(\abs{a_n}\geq\abs{a_{n+1}}\), \(a_{2n}\geq 0,a_{2n+1}\leq 0\), and \(\lim\limits_{n\to\infty}a_n=0\). Then the series \(\sum_{n=0}^{\infty}a_n\) converges.
	\end{theorem}
	\begin{proof}
		\begin{equation}
			\sum_{n=0}^{\infty}a_n
			=
			\sum_{n=0}^{2k}a_n+\sum_{n=2k+1}^{\infty}a_n
			=
			\sum_{n=0}^{2k}a_n+\sum_{m=0}^{\infty}\pqty{a_{2m+1}+a_{2m+2}}
		\end{equation}
		Because each pair \(\pqty{a_{2m+1}+a_{2m+2}}\) is negative, \(\sum_{m=0}^{\infty}\pqty{a_{2m+1}+a_{2m+2}}\) is negative, and thus \(\sum_{n=0}^{2k}a_n\geq \sum_{n=0}^{\infty}a_n\). Likewise,
		\begin{equation}
		\sum_{n=0}^{\infty}a_n
		=
		\sum_{n=0}^{2k+1}a_n+\sum_{n=2k+2}^{\infty}a_n
		=
		\sum_{n=0}^{2k+1}a_n+\sum_{n=m}^{\infty}\pqty{a_{2m+2}+a_{2m+3}}
		\end{equation}
		Because each pair \(\pqty{a_{2m+2}+a_{2m+3}}\) is positive, \(\sum_{m=0}^{\infty}\pqty{a_{2m+2}+a_{2m+3}}\) is positive, and thus \(\sum_{n=0}^{2k+1}a_n\leq \sum_{n=0}^{\infty}a_n\). Thus
		\begin{equation}
			\sum_{n=0}^{2k+1}a_n\leq
			\sum_{n=0}^{\infty}a_n\leq
			\sum_{n=0}^{2k}a_n
		\end{equation}
		The right and left sides differ only by \(a_{2k}\) and \(\lim\limits_{k\to\infty}a_{2k}=0\). Thus by the squeeze theorem, \(\sum_{n=0}^{\infty}a_n\) converges.
	\end{proof}
	
	\section{\(\int_{0}^{\infty}\sin\pqty{xf\pqty{x}}\partial x\) converges for positive monotone increasing unbounded continuous \(f\pqty{x}\)}
	\begin{definition}
		Let a speed function \(f:\mathbb{R}^+\to\mathbb{R}^+\) exist such that
		\begin{equation}
		\bqty{\Bqty{a,b}\subseteq\mathbb{R}^+\land a<b}
		\Rightarrow
		f\pqty{a}<f\pqty{b}
		\end{equation}
	\end{definition}

	As \(f\pqty{x}\to\infty\), the period of \(\sin\pqty{xf\pqty{x}}\) approaches zero, making the integral converge. This is valid by using the Alternating Series test on the positive and negative chunks of the integral which decrease in area monotonically.
	
	\url{https://en.wikipedia.org/wiki/Alternating_series_test}
	

\end{document}
