\documentclass[]{article}
%margins
\usepackage[a4paper,margin=0.15in]{geometry}
%document colors
\usepackage{xcolor}
\definecolor{favoritecolor1}{HTML}{607CB2}
\definecolor{favoritecolor2}{HTML}{303E59}
\makeatletter
\newcommand{\globalcolor}[1]{%
	\color{#1}\global\let\default@color\current@color
}
\makeatother

\AtBeginDocument{\globalcolor{favoritecolor2}}
\pagecolor{favoritecolor1}

%made from template named MathArticleTemplate
\usepackage{amsfonts}
\usepackage{amsmath}
\usepackage{amsthm}
\usepackage{amssymb}
\usepackage{hyperref}
\hypersetup{colorlinks=true}
\usepackage{graphics}

%Fix \eqref in section title
\pdfstringdefDisableCommands{\def\eqref#1{(\ref{#1})}}

\DeclareMathOperator{\es}{Es}
\DeclareMathOperator{\ez}{Ez}
\DeclareMathOperator{\gs}{gs}
\DeclareMathOperator{\md}{mod}
\DeclareMathOperator{\pow}{Pow}
%Parenthesis, Braces, Brackets usepackage{physics}
\newcommand{\pqty}[1]{{\left(#1\right)}}
\newcommand{\Bqty}[1]{{\left\{#1\right\}}}
\newcommand{\bqty}[1]{{\left[#1\right]}}
\newcommand{\abs}[1]{{\left\lvert#1\right\rvert}}
%other stuff
\newcommand{\floor}[1]{{\left\lfloor#1\right\rfloor}}
\newcommand{\ceil}[1]{{\left\lceil#1\right\rceil}}
%Laplace transform and inverse
\newcommand{\laplace}[1]{\mathcal{L}\Bqty{#1}\pqty{s}}
\newcommand{\laplaceInv}[1]{\mathcal{L}^{-1}\Bqty{#1}\pqty{t}}
%Derivatives
\newcommand{\pdiff}[2]{\frac{\partial^{#2}}{\partial #1^{#2}}}

%lemma,theorem, proof
\newtheorem{theorem}{Theorem}[section]
\newtheorem{lemma}[theorem]{Lemma}
\newtheorem{definition}[theorem]{Definition}
\newtheorem{corollary}[theorem]{Corollary}

\numberwithin{equation}{section}

%\usepackage{minted}
%opening
\author{Mark Andrew Gerads: \href{MailTo:Nazgand@Gmail.Com}{Nazgand@Gmail.Com}}

\title{
	1
	
	\href{https://github.com/Nazgand/nazgandMathBook}{https://github.com/Nazgand/nazgandMathBook}
}

\begin{document}
	
	\maketitle
	
	\begin{abstract}
		The goal of this paper is to have fun reviewing the number 1.
	\end{abstract}
	
	Let exist a sequence $a_k\in\mathbb{Z},a_k>1$. Then
	\begin{equation}
		1=\sum_{k\in\mathbb{Z}^+}\frac{a_k-1}{\prod_{m=1}^{k}a_m}
		=\sum_{k=1}^{\infty}\frac{a_k-1}{\prod_{m=1}^{k}a_m}
	\end{equation}


	This can be seen via the limit $j\to\infty$ in the following equation.
	\begin{equation}
	1-\prod_{m=1}^{j}a_m^{-1}=
	\sum_{k=1}^{j}\frac{a_k-1}{\prod_{m=1}^{k}a_m}
	\end{equation}
	
	Proof of previous equation by induction: For the base case, $j=0$, we have $1-1=0$.
	For the inductive step, we can simply prove
	\begin{equation}
		\pqty{1-\prod_{m=1}^{j+1}a_m^{-1}}-\pqty{1-\prod_{m=1}^{j}a_m^{-1}}=
		\pqty{\sum_{k=1}^{j+1}\frac{a_k-1}{\prod_{m=1}^{k}a_m}}-
		\pqty{\sum_{k=1}^{j}\frac{a_k-1}{\prod_{m=1}^{k}a_m}}
	\end{equation}
	
	Simplify both sides to get:
	\begin{equation}
		\prod_{m=1}^{j}a_m^{-1}-\prod_{m=1}^{j+1}a_m^{-1}
		=\sum_{k=j+1}^{j+1}\frac{a_k-1}{\prod_{m=1}^{k}a_m}
		=\frac{a_{j+1}-1}{\prod_{m=1}^{j+1}a_m}
	\end{equation}

	Multiply both sides by $\prod_{m=1}^{j+1}a_m$:
	\begin{equation}
		a_{j+1}-1=a_{j+1}-1
	\end{equation}
	\qed

\end{document}
