\documentclass[]{article}
%made from template named MathArticleTemplate
\usepackage{amsfonts}
\usepackage{amsmath}
\usepackage{amsthm}
\usepackage{amssymb}
\usepackage{hyperref}
\hypersetup{colorlinks=true}
\usepackage{graphics}

%Fix \eqref in section title
\pdfstringdefDisableCommands{\def\eqref#1{(\ref{#1})}}

\DeclareMathOperator{\es}{Es}
\DeclareMathOperator{\ez}{Ez}
\DeclareMathOperator{\gs}{gs}
\DeclareMathOperator{\md}{mod}
\DeclareMathOperator{\pow}{Pow}
%Parenthesis, Braces, Brackets usepackage{physics}
\newcommand{\pqty}[1]{{\left(#1\right)}}
\newcommand{\Bqty}[1]{{\left\{#1\right\}}}
\newcommand{\bqty}[1]{{\left[#1\right]}}
\newcommand{\abs}[1]{{\left\lvert#1\right\rvert}}
%other stuff
\newcommand{\floor}[1]{{\left\lfloor#1\right\rfloor}}
\newcommand{\ceil}[1]{{\left\lceil#1\right\rceil}}
%Laplace transform and inverse
\newcommand{\laplace}[1]{\mathcal{L}\Bqty{#1}\pqty{s}}
\newcommand{\laplaceInv}[1]{\mathcal{L}^{-1}\Bqty{#1}\pqty{t}}
%Derivatives
\newcommand{\pdiff}[2]{\frac{\partial^{#2}}{\partial #1^{#2}}}

%lemma,theorem, proof
\newtheorem{theorem}{Theorem}[section]
\newtheorem{lemma}[theorem]{Lemma}
\newtheorem{definition}[theorem]{Definition}
\newtheorem{corollary}[theorem]{Corollary}

\numberwithin{equation}{section}

%\usepackage{minted}
%opening
\author{Mark Andrew Gerads: \href{MailTo:MarkAndrewGerads.Nazgand@Gmail.Com}{MarkAndrewGerads.Nazgand@Gmail.Com}}

\title{
	Complex Number Law Of Sines
	
	\href{https://github.com/Nazgand/nazgandMathBook}{https://github.com/Nazgand/nazgandMathBook}
}

\begin{document}
	
	\maketitle
	
	\begin{abstract}
		The goal of this paper is to appreciate The Law of Sines rewritten in the form of complex numbers.
	\end{abstract}
	
	The obvious single variable case where $z\in\mathbb{C},\lnot\bqty{z\in\Bqty{0,1}}$:
	\begin{equation}
		\frac{\sin\pqty{\Im\pqty{\ln\pqty{z}}}}{\abs{1-z}}=
		\frac{\sin\pqty{\Im\pqty{\ln\pqty{1-z}}}}{\abs{z}}=
		\sin\pqty{\Im\pqty{\ln\pqty{z}+\ln\pqty{1-z}}}
	\end{equation}

	The distinct 3 variable case where $\Bqty{z_0,z_1,z_2}\subset\mathbb{C},z_0\neq z_1,z_1\neq z_2,z_2\neq z_0$:
	\begin{equation}
		\abs{\frac{\sin\pqty{\Im\pqty{\ln\pqty{\frac{z_2-z_0}{z_1-z_0}}}}}{z_2-z_1}}
	\end{equation}
	is a formula with a symmetry under any permutation of the 3 variables.

\end{document}
