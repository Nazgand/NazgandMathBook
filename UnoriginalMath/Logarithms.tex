\documentclass[]{article}
%margins
\usepackage[a4paper,margin=0.15in]{geometry}
%document colors
\usepackage{xcolor}
\definecolor{favoritecolor1}{HTML}{607CB2}
\definecolor{favoritecolor2}{HTML}{303E59}
\makeatletter
\newcommand{\globalcolor}[1]{%
	\color{#1}\global\let\default@color\current@color
}
\makeatother

\AtBeginDocument{\globalcolor{favoritecolor2}}
\pagecolor{favoritecolor1}

%made from template named MathArticleTemplate
\usepackage{amsfonts}
\usepackage{amsmath}
\usepackage{amsthm}
\usepackage{amssymb}
\usepackage{hyperref}
\hypersetup{colorlinks=true}
\usepackage{graphics}

%Fix \eqref in section title
\pdfstringdefDisableCommands{\def\eqref#1{(\ref{#1})}}

\DeclareMathOperator{\es}{Es}
\DeclareMathOperator{\ez}{Ez}
\DeclareMathOperator{\gs}{gs}
\DeclareMathOperator{\md}{mod}
\DeclareMathOperator{\pow}{Pow}
%Parenthesis, Braces, Brackets usepackage{physics}
\newcommand{\pqty}[1]{{\left(#1\right)}}
\newcommand{\Bqty}[1]{{\left\{#1\right\}}}
\newcommand{\bqty}[1]{{\left[#1\right]}}
\newcommand{\abs}[1]{{\left\lvert#1\right\rvert}}
%other stuff
\newcommand{\floor}[1]{{\left\lfloor#1\right\rfloor}}
\newcommand{\ceil}[1]{{\left\lceil#1\right\rceil}}
%Laplace transform and inverse
\newcommand{\laplace}[1]{\mathcal{L}\Bqty{#1}\pqty{s}}
\newcommand{\laplaceInv}[1]{\mathcal{L}^{-1}\Bqty{#1}\pqty{t}}
%Derivatives
\newcommand{\pdiff}[2]{\frac{\partial^{#2}}{\partial #1^{#2}}}

%lemma,theorem, proof
\newtheorem{theorem}{Theorem}[section]
\newtheorem{lemma}[theorem]{Lemma}
\newtheorem{definition}[theorem]{Definition}

\numberwithin{equation}{section}

\usepackage{xr}
\externaldocument{Differentiation}
\externaldocument{ExponentialFunction}

%\usepackage{minted}
%opening
\author{Mark Andrew Gerads: \href{MailTo:Nazgand@Gmail.Com}{Nazgand@Gmail.Com}}

\title{
	Logarithms
	
	\href{https://github.com/Nazgand/nazgandMathBook}{https://github.com/Nazgand/nazgandMathBook}
}

\begin{document}
	
	\maketitle
	
	\begin{abstract}
		The goal of this paper is to examine the logarithm functions.
	\end{abstract}
	
	\section{Logarithms}
	\begin{definition}
		The logarithm functions are the inverse functions of the exponentiation functions:
		\begin{equation}
			\label{LogarithmDefinition}
			b^{\log_b\pqty{x}}=x
			,
			\log_b\pqty{b^x}=x
		\end{equation}
		\begin{equation}
		\log_e\pqty{x}=\ln\pqty{x}
		\end{equation}
	\end{definition}

	\begin{theorem}[Power rule]
		\begin{equation}
		\log_b\pqty{x^p}=p*\log_b\pqty{x}
		\end{equation}
		\begin{proof}
			Start with \eqref{LogarithmDefinition}:
			\begin{equation}
			b^{\log_b\pqty{x}}=x
			\end{equation}
			Raise both sides to the power of \(p\):
			\begin{equation}
			b^{p*\log_b\pqty{x}}=x^p
			\end{equation}
			Apply the \(\log_b\) function to both sides:
			\begin{equation}
			p*\log_b\pqty{x}=\log_b\pqty{x^p}
			\end{equation}
		\end{proof}
	\end{theorem}

	\begin{theorem}[Product sum rule]
		\begin{equation}
		\log_b\pqty{x*y}=\log_b\pqty{x}+\log_b\pqty{y}
		\end{equation}
		\begin{proof}
			Use \eqref{LogarithmDefinition}
			\begin{equation}
			\log_b\pqty{x*y}=
			\log_b\pqty{b^{\log_b\pqty{x}}*b^{\log_b\pqty{x}}}
			\end{equation}
			Use the additive property of exponential functions
			\begin{equation}
			\log_b\pqty{x*y}=
			\log_b\pqty{b^{\log_b\pqty{x}+\log_b\pqty{x}}}
			\end{equation}
			Use \eqref{LogarithmDefinition}
			\begin{equation}
			\log_b\pqty{x*y}=
			\log_b\pqty{x}+\log_b\pqty{x}
			\end{equation}
		\end{proof}
	\end{theorem}

	\begin{theorem}[Change of base]
		\begin{equation}
			\log_b\pqty{x}=\frac{\log_k\pqty{x}}{\log_k\pqty{b}}
		\end{equation}
		\begin{proof}
			Start with \eqref{LogarithmDefinition}:
			\begin{equation}
				b^{\log_b\pqty{x}}=x
			\end{equation}
			Apply the \(\log_k\) function to both sides:
			\begin{equation}
				\log_k\pqty{b^{\log_b\pqty{x}}}=\log_k\pqty{x}
			\end{equation}
			\begin{equation}
				\log_b\pqty{x}\log_k\pqty{b}=\log_k\pqty{x}
			\end{equation}
			Divide both sides by \(\log_k\pqty{b}\)
		\end{proof}
	\end{theorem}

	\begin{theorem}[Integral form]
		\begin{equation}
		\label{LnIntegralForm}
		\ln\pqty{z}=\int_{1}^{z}\frac{1}{x}\partial x
		\end{equation}
	\begin{proof}
		Proof from \url{https://math.stackexchange.com/questions/1341958/proof-of-the-derivative-of-lnx/1341995#1341995}.
		
		The derivative of the logarithm functions exist because the exponential functions have derivatives.
		\url{https://en.wikipedia.org/wiki/Inverse_function_theorem}
		\begin{equation}
		\pdiff{x}{}x=1
		\end{equation}
		Substitute \eqref{LogarithmDefinition}:
		\begin{equation}
		\pdiff{x}{}e^{\ln\pqty{x}}=1
		\end{equation}
		The chain rule [Differentiation\eqref{DiffChainRule}]
		\begin{equation}
		\pqty{\pdiff{z}{}e^{z}:z\to \ln\pqty{x}}\pdiff{x}{}{\ln\pqty{x}}=1
		\end{equation}
		The derivative of the natural exponential function is itself: [ExponentialFunction\eqref{expOwnDeriv}]
		\begin{equation}
		\pqty{e^{z}:z\to \ln\pqty{x}}\pdiff{x}{}{\ln\pqty{x}}=1
		\end{equation}
		Substitute:
		\begin{equation}
		e^{\ln\pqty{x}}\pdiff{x}{}{\ln\pqty{x}}=1
		\end{equation}
		Simplify:
		\begin{equation}
		x\pdiff{x}{}{\ln\pqty{x}}=1
		\end{equation}
		Divide:
		\begin{equation}
		\pdiff{x}{}{\ln\pqty{x}}=\frac{1}{x}
		\end{equation}
		Integrate:
		\begin{equation}
		\ln\pqty{z}={\ln\pqty{z}-\ln\pqty{1}}=\int_{1}^{z}\frac{1}{x}\partial x
		\end{equation}
		
	\end{proof}
	\end{theorem}


	
	
	\url{https://en.wikipedia.org/wiki/Logarithm}
\end{document}
