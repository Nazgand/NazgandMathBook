\documentclass[]{article}
%margins
\usepackage[a4paper,margin=0.15in]{geometry}
%document colors
\usepackage{xcolor}
\definecolor{favoritecolor1}{HTML}{607CB2}
\definecolor{favoritecolor2}{HTML}{303E59}
\makeatletter
\newcommand{\globalcolor}[1]{%
	\color{#1}\global\let\default@color\current@color
}
\makeatother

\AtBeginDocument{\globalcolor{favoritecolor2}}
\pagecolor{favoritecolor1}

%made from template named MathArticleTemplate
\usepackage{amsfonts}
\usepackage{amsmath}
\usepackage{amsthm}
\usepackage{amssymb}
\usepackage{hyperref}
\hypersetup{colorlinks=true}
\usepackage{graphics}

%Fix \eqref in section title
\pdfstringdefDisableCommands{\def\eqref#1{(\ref{#1})}}

\DeclareMathOperator{\es}{Es}
\DeclareMathOperator{\ez}{Ez}
\DeclareMathOperator{\gs}{gs}
\DeclareMathOperator{\md}{mod}
\DeclareMathOperator{\pow}{Pow}
%Parenthesis, Braces, Brackets usepackage{physics}
\newcommand{\pqty}[1]{{\left(#1\right)}}
\newcommand{\Bqty}[1]{{\left\{#1\right\}}}
\newcommand{\bqty}[1]{{\left[#1\right]}}
\newcommand{\abs}[1]{{\left\lvert#1\right\rvert}}
%other stuff
\newcommand{\floor}[1]{{\left\lfloor#1\right\rfloor}}
\newcommand{\ceil}[1]{{\left\lceil#1\right\rceil}}
%Laplace transform and inverse
\newcommand{\laplace}[1]{\mathcal{L}\Bqty{#1}\pqty{s}}
\newcommand{\laplaceInv}[1]{\mathcal{L}^{-1}\Bqty{#1}\pqty{t}}
%Derivatives
\newcommand{\pdiff}[2]{\frac{\partial^{#2}}{\partial #1^{#2}}}

%lemma,theorem, proof
\newtheorem{theorem}{Theorem}[section]
\newtheorem{lemma}[theorem]{Lemma}
\newtheorem{definition}[theorem]{Definition}
\newtheorem{corollary}[theorem]{Corollary}

\numberwithin{equation}{section}

\usepackage{xr}
\externaldocument{TaylorSeries}

%\usepackage{minted}
%opening
\author{Mark Andrew Gerads: \href{MailTo:Nazgand@Gmail.Com}{Nazgand@Gmail.Com}}

\title{
	Trigonometry
	
	\href{https://github.com/Nazgand/nazgandMathBook}{https://github.com/Nazgand/nazgandMathBook}
}

\begin{document}
	
	\maketitle
	
	\begin{abstract}
		The goal of this paper is to review trigonometry.
	\end{abstract}
	
	\section{Geometric Foundations}
	In [Elementary Geometry and Trigonometry.kig] by moving the angle control point to the second quadrant and comparing, we can see
	\begin{equation}
	\label{SinCosPhaseShift}
	\sin\pqty{\theta+\frac{\pi}{2}}=\pqty{\cos\pqty{\theta}}
	,
	\cos\pqty{\theta+\frac{\pi}{2}}=-\pqty{\sin\pqty{\theta}}
	\end{equation}
	Moreover, we see as \(\theta\to 0\)
	\begin{equation}
	\label{SinXOverXLimit}
	\lim\limits_{\theta\to 0}\frac{\sin\pqty{\theta}}{\theta}=0
	\end{equation}
	Comparing with [PythagoreanTheorem.png], we find
	\begin{equation}
	\label{SinCosPythagoras}
	1=\sin\pqty{\theta}^2+\cos\pqty{\theta}^2
	\end{equation}
	From [SumOfAnglesSinCos.kig], we get
	\begin{equation}
		\label{SinSumOfAngle}
		\sin\pqty{\alpha+\beta}=\sin\pqty{\alpha}\cos\pqty{\beta}+\cos\pqty{\alpha}\sin\pqty{\beta}
	\end{equation}
	And
	\begin{equation}
	\label{CosSumOfAngle}
	\cos\pqty{\alpha+\beta}=\cos\pqty{\alpha}\cos\pqty{\beta}-\sin\pqty{\alpha}\sin\pqty{\beta}
	\end{equation}
	
	\section{Derivatives}
	\begin{theorem}
	\begin{equation}
	\pdiff{x}{}\sin\pqty{x}=\cos\pqty{x}
	\end{equation}
	\begin{proof}
		\begin{equation}
			\pdiff{x}{}\sin\pqty{x}=
			\lim\limits_{h\to 0}\frac{\sin\pqty{x+h}-\sin\pqty{x}}{h}
		\end{equation}
		Use \eqref{SinSumOfAngle}
		\begin{equation}
		\pdiff{x}{}\sin\pqty{x}=
		\lim\limits_{h\to 0}\frac{\sin\pqty{x}\cos\pqty{h}+\cos\pqty{x}\sin\pqty{h}-\sin\pqty{x}}{h}
		\end{equation}
		Split the limit:
		\begin{equation}
		\pdiff{x}{}\sin\pqty{x}=
		\lim\limits_{h\to 0}\frac{\cos\pqty{x}\sin\pqty{h}}{h}
		+
		\lim\limits_{h\to 0}\frac{\sin\pqty{x}\pqty{\cos\pqty{h}-1}}{h}
		\end{equation}
		Simplify
		\begin{equation}
		\pdiff{x}{}\sin\pqty{x}=
		\cos\pqty{x}\lim\limits_{h\to 0}\frac{\sin\pqty{h}}{h}
		+
		\sin\pqty{x}\lim\limits_{h\to 0}\frac{{\cos\pqty{h}-1}}{h}
		\end{equation}
		Use \eqref{SinXOverXLimit} on left limit and multiply right limit by \(\frac{\cos\pqty{h}+1}{\cos\pqty{h}+1}\)
		\begin{equation}
		\pdiff{x}{}\sin\pqty{x}=
		\cos\pqty{x}
		+
		\sin\pqty{x}\lim\limits_{h\to 0}\frac{{\cos\pqty{h}^2-1}}{h\pqty{\cos\pqty{h}+1}}
		\end{equation}
		Use \eqref{SinCosPythagoras}
		\begin{equation}
		\pdiff{x}{}\sin\pqty{x}=
		\cos\pqty{x}
		+
		\sin\pqty{x}\lim\limits_{h\to 0}\frac{{\sin\pqty{h}^2}}{h\pqty{\cos\pqty{h}+1}}
		\end{equation}
		Split the limit:
		\begin{equation}
		\pdiff{x}{}\sin\pqty{x}=
		\cos\pqty{x}
		+
		\sin\pqty{x}\lim\limits_{a\to 0}\frac{{\sin\pqty{a}}}{h}
		\lim\limits_{b\to 0}{{\sin\pqty{b}}}
		\lim\limits_{c\to 0}\frac{1}{\pqty{\cos\pqty{c}+1}}
		\end{equation}
		Evaluate using \eqref{SinXOverXLimit} again
		\begin{equation}
		\pdiff{x}{}\sin\pqty{x}=
		\cos\pqty{x}
		+
		1
		*0
		*\frac{1}{2}=
		\cos\pqty{x}
		\end{equation}
	\end{proof}
	\end{theorem}
	
	\begin{theorem}
		\begin{equation}
		\label{key}
		\pdiff{x}{}\cos\pqty{x}=-\sin\pqty{x}
		\end{equation}
		\begin{proof}
			Use \eqref{SinCosPhaseShift}
			\begin{equation}
			\pdiff{x}{}\cos\pqty{x}=\pdiff{x}{}\sin\pqty{x+\frac{\pi}{2}}
			\end{equation}
			Use the chain rule:
			\begin{equation}
			\pdiff{x}{}\cos\pqty{x}=
			\pqty{\pdiff{z}{}\sin\pqty{z}:z\to x+\frac{\pi}{2}}
			\pdiff{x}{}\pqty{x+\frac{\pi}{2}}
			\end{equation}
			Simplify
			\begin{equation}
			\pdiff{x}{}\cos\pqty{x}=
			\cos\pqty{x+\frac{\pi}{2}}
			\end{equation}
			Use 
			Use \eqref{SinCosPhaseShift}
			\begin{equation}
			\pdiff{x}{}\cos\pqty{x}=
			-\sin\pqty{x}
			\end{equation}
		\end{proof}
	\end{theorem}

	\section{Taylor Series}
	\begin{theorem}
		\begin{equation}
		\label{SinTaylor}
		\sin\pqty{x}=\sum_{k=0}^{\infty} \frac{\pqty{-1}^kx^{2k+1}}{\pqty{2k+1}!}
		\end{equation}
		\begin{proof}
			The multiple derivatives cycle through 4 functions. Let \(n\in\mathbb{Z}\):
			\begin{equation}
				\pdiff{x}{4n}\sin\pqty{x}=\sin\pqty{x}
			\end{equation}
			\begin{equation}
				\pdiff{x}{4n+1}\sin\pqty{x}=\cos\pqty{x}
			\end{equation}
			\begin{equation}
				\pdiff{x}{4n+2}\sin\pqty{x}=-\sin\pqty{x}
			\end{equation}
			\begin{equation}
				\pdiff{x}{4n+3}\sin\pqty{x}=-\cos\pqty{x}
			\end{equation}
			The derivatives at 0 thus cycle through \(0,1,0,-1\). Use [TaylorSeries\eqref{TaylorSeriesDef}] and simplify out the zeroes.
		\end{proof}
	\end{theorem}

	\begin{theorem}
		\begin{equation}
		\label{CosTaylor}
		\cos\pqty{x}=\sum_{k=0}^{\infty} \frac{\pqty{-1}^kx^{2k}}{\pqty{2k}!}
		\end{equation}
		\begin{proof}
			The multiple derivatives cycle through 4 functions. Let \(n\in\mathbb{Z}\):
			\begin{equation}
			\pdiff{x}{4n}\sin\pqty{x}=\cos\pqty{x}
			\end{equation}
			\begin{equation}
			\pdiff{x}{4n+1}\sin\pqty{x}=-\sin\pqty{x}
			\end{equation}
			\begin{equation}
			\pdiff{x}{4n+2}\sin\pqty{x}=-\cos\pqty{x}
			\end{equation}
			\begin{equation}
			\pdiff{x}{4n+3}\sin\pqty{x}=\sin\pqty{x}
			\end{equation}
			The derivatives at 0 thus cycle through \(1,0,-1,0\). Use [TaylorSeries\eqref{TaylorSeriesDef}] and simplify out the zeroes.
		\end{proof}
	\end{theorem}
\end{document}
