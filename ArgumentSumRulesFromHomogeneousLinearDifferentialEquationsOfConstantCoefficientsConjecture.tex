\documentclass[]{article}
%margins
\usepackage[a4paper,margin=0.15in]{geometry}
%document colors
\usepackage{xcolor}
\definecolor{favoritecolor1}{HTML}{607CB2}
\definecolor{favoritecolor2}{HTML}{303E59}
\makeatletter
\newcommand{\globalcolor}[1]{%
	\color{#1}\global\let\default@color\current@color
}
\makeatother

\AtBeginDocument{\globalcolor{favoritecolor2}}
\pagecolor{favoritecolor1}

%made from template named MathArticleTemplate
\usepackage{amsfonts}
\usepackage{amsmath}
\usepackage{amsthm}
\usepackage{amssymb}
\usepackage{hyperref}
\hypersetup{colorlinks=true}
\usepackage{graphics}

%Fix \eqref in section title
\pdfstringdefDisableCommands{\def\eqref#1{(\ref{#1})}}

\DeclareMathOperator{\es}{Es}
\DeclareMathOperator{\ez}{Ez}
\DeclareMathOperator{\gs}{gs}
\DeclareMathOperator{\md}{mod}
\DeclareMathOperator{\pow}{Pow}
%Parenthesis, Braces, Brackets usepackage{physics}
\newcommand{\pqty}[1]{{\left(#1\right)}}
\newcommand{\Bqty}[1]{{\left\{#1\right\}}}
\newcommand{\bqty}[1]{{\left[#1\right]}}
\newcommand{\abs}[1]{{\left\lvert#1\right\rvert}}
%other stuff
\newcommand{\floor}[1]{{\left\lfloor#1\right\rfloor}}
\newcommand{\ceil}[1]{{\left\lceil#1\right\rceil}}
%Laplace transform and inverse
\newcommand{\laplace}[1]{\mathcal{L}\Bqty{#1}\pqty{s}}
\newcommand{\laplaceInv}[1]{\mathcal{L}^{-1}\Bqty{#1}\pqty{t}}
%Derivatives
\newcommand{\pdiff}[2]{\frac{\partial^{#2}}{\partial #1^{#2}}}
%Kettenbruch
\newcommand{\ketten}[4]{\underset{#1}{\overset{#2}{\LARGE\mathrm K}}\frac{#3}{#4}}

%lemma,theorem, proof
\newtheorem{theorem}{Theorem}[section]
\newtheorem{lemma}[theorem]{Lemma}
\newtheorem{definition}[theorem]{Definition}
\newtheorem{corollary}[theorem]{Corollary}

\numberwithin{equation}{section}

%\usepackage{minted}
%opening
\author{Mark Andrew Gerads: \href{MailTo:Nazgand@Gmail.Com}{Nazgand@Gmail.Com}}

\title{
	Argument Sum Rules From Homogeneous Linear Differential Equations Of Constant Coefficients Conjecture
	
	\href{https://github.com/Nazgand/nazgandMathBook}{https://github.com/Nazgand/nazgandMathBook}
}

\begin{document}
	
	\maketitle
	
	\begin{abstract}
		The goal of this paper is to conjecture a seemingly fundamental calculus fact.
	\end{abstract}
	
	A [homogeneous linear differential equation of constant coefficients] has the form
	\begin{equation}
		0=\sum_{k=0}^{n}a_k\pdiff{z}{k}f\pqty{z}
		=\sum_{k=0}^{n}a_k f^\pqty{k}\pqty{z}
	\end{equation}
	where $\forall k, a_k\in\mathbb{C}, a_n \neq 0$, and $f:\mathbb{C}\to\mathbb{C}$ is differentiable everywhere.
	Let $g_0\pqty{z},\dots,g_{n-1}\pqty{z}$ be solutions that span the vector space of solutions of the [homogeneous linear differential equation of constant coefficients]. To be clear:
	\begin{equation}
		\text{Solutions}=
		\Bqty{h\pqty{z} \mid 0=\sum_{k=0}^{n}a_k h^\pqty{k}\pqty{z}}
		=\Bqty{\sum_{k=0}^{n-1}b_k g_k\pqty{z} \mid b_k\in\mathbb{C}}
	\end{equation}
	Let us define a column vector and clarify its transpose (a row vector):
	\begin{equation}
		v\pqty{z_0}=
		\begin{pmatrix}
			g_0\pqty{z_0} \\
			\vdots \\
			g_{n-1}\pqty{z_0} \\
		\end{pmatrix}
		,
		v\pqty{z_1}^T=
		\begin{pmatrix}
			g_0\pqty{z_1} &
			\dots &
			g_{n-1}\pqty{z_1}
		\end{pmatrix}
	\end{equation}
	I conjecture there exists a complex-valued constant $n$ by $n$ symmetric matrix $A$ ($A=A^T$) such that
	\begin{equation}
		f\pqty{z_0+z_1}=v\pqty{z_1}^T A v\pqty{z_0}=v\pqty{z_0}^T A v\pqty{z_1}
	\end{equation}

	\section{A reason $A$ is symmetric}
	Suppose instead of a symmetric matrix $A$, we find a matrix $B$ such that
	\begin{equation}
		f\pqty{z_0+z_1}=v\pqty{z_1}^T B v\pqty{z_0}
	\end{equation}
	Then take the transpose of the equation and substitute $z_0\to z_1,z_1\to z_0$, resulting in the following equation:
	\begin{equation}
		f\pqty{z_0+z_1}=v\pqty{z_1}^T B^T v\pqty{z_0}
	\end{equation}
	Average both equations:
	\begin{equation}
		f\pqty{z_0+z_1}=v\pqty{z_1}^T \frac{B+B^T}{2} v\pqty{z_0}
	\end{equation}
	Note that we can set $A=\frac{B+B^T}{2}$ because it is symmetric. \qed

	\section{Motivation}
	The first thing of note is
	\begin{equation}
		f\pqty{z}\in\text{Solutions}
	\end{equation}
	The next thing of note is that offsetting the function's argument by a constant results in the same differential equation, with a solution of the same form.
	\begin{equation}
		\pdiff{z}{}z_0=0 \Rightarrow f\pqty{z+z_0}\in\text{Solutions}
	\end{equation}	
	One thing to note is that all constants are replaced with functions of the constant \(z_0\). This allows us that change \(z_0\) from a constant to a variable and still have working math when we consider all possible constants \(z_0\).
	\begin{equation}
		\bqty{\pdiff{z}{}z_0=0 \land f\pqty{z+z_0}\in\text{Solutions}}
		\Rightarrow \exists b_{0,k} : \mathbb{C}\to\mathbb{C}, f\pqty{z+z_0} = \sum_{k=0}^{n-1}b_{0,k}\pqty{z_0} g_k\pqty{z}
	\end{equation}
	Because we were careful about making all constants a function of the constant \(z_0\), we can make \(z_0\) a variable with the conclusion
	\begin{equation}
		\label{fzz0}
		\exists b_{0,k} : \mathbb{C}\to\mathbb{C}, f\pqty{z+z_0} = \sum_{k=0}^{n-1}b_{0,k}\pqty{z_0} g_k\pqty{z}
	\end{equation}
	Follow the same logic with a constant named \(z_1\):
	\begin{equation}
		\pdiff{z}{}z_1=0 \Rightarrow f\pqty{z+z_1}\in\text{Solutions}
	\end{equation}	
	\begin{equation}
		\bqty{\pdiff{z}{}z_1=0 \land f\pqty{z+z_1}\in\text{Solutions}}
		\Rightarrow \exists b_{1,k} : \mathbb{C}\to\mathbb{C}, f\pqty{z+z_1} = \sum_{k=0}^{n-1}b_{1,k}\pqty{z_1} g_k\pqty{z}
	\end{equation}
	\begin{equation}
		\label{fzz1}
		\exists b_{1,k} : \mathbb{C}\to\mathbb{C}, f\pqty{z+z_1} = \sum_{k=0}^{n-1}b_{1,k}\pqty{z_1} g_k\pqty{z}
	\end{equation}
	Now we substitute \(z\to z_1\) in \eqref{fzz0}, and we substitute \(z\to z_0\) in \eqref{fzz1}, and combine them.
	\begin{equation}
		\exists b_{0,k} : \mathbb{C}\to\mathbb{C},
		\exists b_{1,k} : \mathbb{C}\to\mathbb{C},
		f\pqty{z_0+z_1}
		= \sum_{k=0}^{n-1}b_{0,k}\pqty{z_0} g_k\pqty{z_1}
		= \sum_{k=0}^{n-1}b_{1,k}\pqty{z_1} g_k\pqty{z_0}
	\end{equation}
	We now have 2 ways of viewing \(f\pqty{z_0+z_1}\), and we could imagine a sum of a line from the front, a sum of a line from the right, and 'deduce' that the true form is probably the sum of a square from above, which leads to the matrix form conjectured, but this is slightly sketchy logic that needs more rigor.

	\section{LEAN 4 code}
		This conjecture has been formalized in LEAN 4:
		
		\href{https://github.com/Nazgand/NazgandLean4/blob/master/NazgandLean4/ArgumentSumConjecture.lean}{https://github.com/Nazgand/NazgandLean4/blob/master/NazgandLean4/ArgumentSumConjecture.lean}
		
		For a given differential equation, if the conjecture is true for 1 basis, then it is true for all basises.
\end{document}
