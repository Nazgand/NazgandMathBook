\documentclass{article}
\usepackage{amsmath}
\usepackage{amssymb}
\usepackage{amsthm}

\title{Proof of $x^{\frac{1}{x}} \leq e^{\frac{1}{e}}$ for $x > 0$}
\author{Gemini 2.0 Flash Experimental}

\begin{document}
	
	\maketitle
	
	\begin{abstract}
		We provide a complete proof that for any positive real number $x$, the inequality $x^{\frac{1}{x}} \leq e^{\frac{1}{e}}$ holds. This is achieved by analyzing the natural logarithm of the function $g(x) = x^{\frac{1}{x}}$, demonstrating that its maximum occurs at $x=e$.
	\end{abstract}
	
	\section{Introduction}
	
	The inequality $x^{\frac{1}{x}} \leq e^{\frac{1}{e}}$ for $x > 0$ is a well-known result. Our approach involves examining the function $g(x) = x^{\frac{1}{x}}$. Since the natural logarithm is a strictly increasing function, finding the maximum of $\ln(g(x))$ is equivalent to finding the maximum of $g(x)$. This simplification makes the analysis considerably easier.
	
	\section{Proof}
	
	Let $g(x) = x^{\frac{1}{x}}$ for $x > 0$. We define $f(x) = \ln(g(x))$:
	
	$$f(x) = \ln(x^{\frac{1}{x}}) = \frac{1}{x} \ln x = \frac{\ln x}{x}$$
	
	Our goal is to find the maximum of $f(x)$. Since the natural logarithm is a monotonically increasing function, the $x$ value at which $f(x)$ attains its maximum will be the same $x$ value at which $g(x)$ attains its maximum.
	
	To find the maximum of $f(x)$, we compute its first derivative:
	
	$$f'(x) = \frac{\frac{1}{x} \cdot x - \ln x \cdot 1}{x^2} = \frac{1 - \ln x}{x^2}$$
	
	We find the critical points by setting $f'(x) = 0$:
	
	$$\frac{1 - \ln x}{x^2} = 0$$
	
	Since $x^2 > 0$ for $x > 0$, this is equivalent to:
	
	$$1 - \ln x = 0$$
	$$\ln x = 1$$
	$$x = e$$
	
	Now, we use the second derivative test to determine if this critical point corresponds to a maximum or minimum. We calculate the second derivative of $f(x)$:
	
	$$f''(x) = \frac{-\frac{1}{x} \cdot x^2 - (1 - \ln x) \cdot 2x}{x^4} = \frac{-x - 2x + 2x \ln x}{x^4} = \frac{-3x + 2x \ln x}{x^4} = \frac{-3 + 2 \ln x}{x^3}$$
	
	We evaluate the second derivative at $x = e$:
	
	$$f''(e) = \frac{-3 + 2 \ln e}{e^3} = \frac{-3 + 2}{e^3} = \frac{-1}{e^3}$$
	
	Since $f''(e) < 0$, the function $f(x)$ has a local maximum at $x = e$.
	
	Because $f'(x) > 0$ for $0 < x < e$ and $f'(x) < 0$ for $x > e$, $f(x)$ is increasing on $(0, e)$ and decreasing on $(e, \infty)$. Thus, the local maximum at $x=e$ is a global maximum.
	
	The maximum value of $f(x)$ is:
	
	$$f(e) = \frac{\ln e}{e} = \frac{1}{e}$$
	
	Since the maximum of $f(x) = \ln(g(x))$ occurs at $x=e$, the maximum of $g(x) = x^{\frac{1}{x}}$ also occurs at $x=e$. The maximum value of $g(x)$ is:
	
	$$g(e) = e^{\frac{1}{e}}$$
	
	Therefore, for all $x > 0$:
	
	$$g(x) = x^{\frac{1}{x}} \leq e^{\frac{1}{e}}$$
	
	This completes the proof.
	
\end{document}