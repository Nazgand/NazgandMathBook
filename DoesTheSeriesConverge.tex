\documentclass[]{article}
%margins
\usepackage[a4paper,margin=0.15in]{geometry}
%document colors
\usepackage{xcolor}
\definecolor{favoritecolor1}{HTML}{607CB2}
\definecolor{favoritecolor2}{HTML}{303E59}
\makeatletter
\newcommand{\globalcolor}[1]{%
	\color{#1}\global\let\default@color\current@color
}
\makeatother

\AtBeginDocument{\globalcolor{favoritecolor2}}
\pagecolor{favoritecolor1}

%made from template named MathArticleTemplate
\usepackage{amsfonts}
\usepackage{amsmath}
\usepackage{amsthm}
\usepackage{amssymb}
\usepackage{hyperref}
\hypersetup{colorlinks=true}
\usepackage{graphics}

%Fix \eqref in section title
\pdfstringdefDisableCommands{\def\eqref#1{(\ref{#1})}}

\DeclareMathOperator{\es}{Es}
\DeclareMathOperator{\ez}{Ez}
\DeclareMathOperator{\gs}{gs}
\DeclareMathOperator{\md}{mod}
\DeclareMathOperator{\pow}{Pow}
%Parenthesis, Braces, Brackets usepackage{physics}
\newcommand{\pqty}[1]{{\left(#1\right)}}
\newcommand{\Bqty}[1]{{\left\{#1\right\}}}
\newcommand{\bqty}[1]{{\left[#1\right]}}
\newcommand{\abs}[1]{{\left\lvert#1\right\rvert}}
%other stuff
\newcommand{\floor}[1]{{\left\lfloor#1\right\rfloor}}
\newcommand{\ceil}[1]{{\left\lceil#1\right\rceil}}
%Laplace transform and inverse
\newcommand{\laplace}[1]{\mathcal{L}\Bqty{#1}\pqty{s}}
\newcommand{\laplaceInv}[1]{\mathcal{L}^{-1}\Bqty{#1}\pqty{t}}
%Derivatives
\newcommand{\pdiff}[2]{\frac{\partial^{#2}}{\partial #1^{#2}}}

%lemma,theorem, proof
\newtheorem{theorem}{Theorem}[section]
\newtheorem{lemma}[theorem]{Lemma}
\newtheorem{definition}[theorem]{Definition}
\newtheorem{corollary}[theorem]{Corollary}

\numberwithin{equation}{section}

%\usepackage{minted}
%opening
\author{Mark Andrew Gerads: \href{MailTo:Nazgand@Gmail.Com}{Nazgand@Gmail.Com}}

\title{
	Does the Series Converge?
	
	\href{https://github.com/Nazgand/nazgandMathBook}{https://github.com/Nazgand/nazgandMathBook}
}

\begin{document}
	
	\maketitle
	
	\begin{abstract}
		The goal of this paper is to prove the convergence of certain series.
	\end{abstract}

	\section{Problem -2}
	Given [a function from [the natural numbers] to [the real numbers]] which is [inclusively bounded in magnitude by 1], i.e.
	\begin{equation}
		\exists f_1:\mathbb{Z}_{\geq 0}\to\mathbb{R},
		\bqty{\forall k\in\mathbb{Z}_{\geq 0}, \abs{f_1\pqty{k}}\leq 1}
	\end{equation}
	and [a positive real number smaller than 1], i.e.
	\begin{equation}
		\exists r\in\mathbb{R}^+,
		r<1
	\end{equation}
	, prove
	\begin{equation}
		\exists \bqty{\lim\limits_{m\to\infty}\sum_{k=0}^m{f_1\pqty{k}}r^k}
		\in\mathbb{R}
	\end{equation}
	Solution: Note
	\begin{equation}
		\forall n\in\mathbb{Z}_{\geq 0},\bqty{
			\exists \bqty{\lim\limits_{m\to\infty}\sum_{k=0}^m{f_1\pqty{k}}r^k}
			\in\mathbb{R}
			\Leftrightarrow
			\exists \bqty{\lim\limits_{m\to\infty}\sum_{k=n}^m{f_1\pqty{k}}r^k}
			\in\mathbb{R}
		}
	\end{equation}
	and
	\begin{equation}
		-\frac{r^n-r^{m+1}}{1-r}=-\sum_{k=n}^m r^k\leq\sum_{k=n}^m{f_1\pqty{k}}r^k\leq\sum_{k=n}^m r^k=\frac{r^n-r^{m+1}}{1-r}
	\end{equation}
	First, let $m\to\infty$.
	\begin{equation}
		-\frac{r^n}{1-r}\leq\lim\limits_{m\to\infty}\sum_{k=n}^m{f_1\pqty{k}}r^k\leq\frac{r^n}{1-r}
	\end{equation}
	Then let $n\to\infty$ to observe the Squeeze Theorem.
	\begin{equation}
		0\leq\lim\limits_{n\to\infty}\lim\limits_{m\to\infty}\sum_{k=n}^m{f_1\pqty{k}}r^k\leq 0
	\end{equation}

	\section{Problem -1}
	Given [a function from [the natural numbers] to [the complex numbers]] which is [inclusively bounded in magnitude by 1], i.e.
	\begin{equation}
		\exists f_1:\mathbb{Z}_{\geq 0}\to\mathbb{C},
		\bqty{\forall k\in\mathbb{Z}_{\geq 0}, \abs{f_1\pqty{k}}\leq 1}
	\end{equation}
	and [a complex number smaller than 1 in magnitude], i.e.
	\begin{equation}
		\exists r\in\mathbb{C},
		\abs{r}<1
	\end{equation}
	, prove
	\begin{equation}
		\exists \bqty{\lim\limits_{m\to\infty}\sum_{k=0}^m{f_1\pqty{k}}r^k}
		\in\mathbb{C}
	\end{equation}
	Without loss of generality, we can set $r=\abs{r}$, as seen by the substitutions
	$f_1\pqty{k}\to f_1\pqty{k}\frac{r^k}{\abs{r}^k}, r\to\abs{r}$.
	The case $r=0$ is trivial, so let $r\in\mathbb{R},0<r<1$.
	The relevant Cauchy series can be split into 2 Cauchy series for the imaginary and real parts; [the complex limit can be proved to exist] by [proving the existence of the \{real, imaginary\} parts]; thus, the problem  is reduced to [Problem -2]
	
	\section{Problem 0}
	Given [a function from [the natural numbers] to [the complex numbers]] which is [exponentially bounded in magnitude], i.e.
	\begin{equation}
		\exists f_a:\mathbb{Z}_{\geq 0}\to\mathbb{C},
		\exists a\in\mathbb{R},
		\bqty{\forall k\in\mathbb{Z}_{\geq 0},
		a^{k+1}\geq\abs{f_a\pqty{k}}}
	\end{equation}
	, and
	\begin{equation}
		\exists t\in\mathbb{R}^+
	\end{equation}
	, prove:
	\begin{equation}
		\exists \bqty{\lim\limits_{m\to\infty}\sum_{k=0}^m\frac{f_a\pqty{k}}{\pqty{k!}^t}}
		\in\mathbb{C}
	\end{equation}
	This sum 'obviously' converges via $\bqty{\forall b\in\mathbb{R}^+,\mathcal{O}\pqty{\pqty{k!}^t}>\mathcal{O}\pqty{b^k}}$, $\mathcal{O}\pqty{a^k}\geq\mathcal{O}\pqty{\abs{f_a\pqty{k}}}$. We must be more rigorous than that to build a solid foundation.
	A good first step is to notice
	\begin{equation}
		\forall K\in\mathbb{Z}_{\geq 0},
		\bqty{\bqty{\exists \bqty{\lim\limits_{m\to\infty}\sum_{k=0}^m\frac{f_a\pqty{k}}{\pqty{k!}^t}}
		\in\mathbb{C}}
		\Leftrightarrow
		\bqty{\exists \bqty{\lim\limits_{m\to\infty}\sum_{k=K}^m\frac{f_a\pqty{k}}{\pqty{k!}^t}}
		\in\mathbb{C}}}
	\end{equation}
	and let $K$ be sufficiently large so that $\frac{\pqty{\pqty{K+1}!}^t}{\pqty{K!}^t}>a$, i.e. $K>\sqrt[t]{a}-1$. Then this problem reduces to [Problem -1].

	\section{Problem 1}
	Given [a function from [the natural numbers] to [the complex numbers]] which is [exponentially bounded in magnitude], i.e.
	\begin{equation}
		\exists f_b:\mathbb{Z}_{\geq 0}\to\mathbb{C},
		\exists b\in\mathbb{R},
		\bqty{\forall k\in\mathbb{Z}_{\geq 0},
			b^{k+1}\geq\abs{f_b\pqty{k}}}
	\end{equation}
	, and
	\begin{equation}
		\exists t\in\mathbb{R}^+,
		\exists z\in\mathbb{C}
	\end{equation}
	, prove:
	\begin{equation}
		\exists \bqty{\lim\limits_{m\to\infty}\sum_{k=0}^m\frac{f_b\pqty{k}z^k}{\pqty{k!}^t}}
		\in\mathbb{C}
	\end{equation}
	Solution: Let $f_a\pqty{k}=f_b\pqty{k}z^k,a\geq b\abs{z}$, and note
	\begin{equation}
		\bqty{\forall k\in\mathbb{Z}_{\geq 0},
			b^{k+1}\geq\abs{f_b\pqty{k}}}
		\Rightarrow
		\bqty{\forall k\in\mathbb{Z}_{\geq 0},
			a^{k+1}\geq\abs{f_a\pqty{k}}}
	\end{equation}
	Thus [Problem 1] has been reduced to [Problem 0].

\end{document}
