\documentclass[]{article}
%made from template named MathArticleTemplate
\usepackage{amsfonts}
\usepackage{amsmath}
\usepackage{amsthm}
\usepackage{amssymb}
\usepackage{hyperref}
\hypersetup{colorlinks=true}
\usepackage{graphics}

%Fix \eqref in section title
\pdfstringdefDisableCommands{\def\eqref#1{(\ref{#1})}}

\DeclareMathOperator{\rues}{rues}
\DeclareMathOperator{\cdrues}{cdrues}
\DeclareMathOperator{\ez}{Ez}
\DeclareMathOperator{\gs}{gs}
\DeclareMathOperator{\md}{mod}
\DeclareMathOperator{\pow}{Pow}
%Parenthesis, Braces, Brackets usepackage{physics}
\newcommand{\pqty}[1]{{\left(#1\right)}}
\newcommand{\Bqty}[1]{{\left\{#1\right\}}}
\newcommand{\bqty}[1]{{\left[#1\right]}}
\newcommand{\abs}[1]{{\left\lvert#1\right\rvert}}
%other stuff
\newcommand{\floor}[1]{{\left\lfloor#1\right\rfloor}}
\newcommand{\ceil}[1]{{\left\lceil#1\right\rceil}}
%Laplace transform and inverse
\newcommand{\laplace}[1]{\mathcal{L}\Bqty{#1}\pqty{s}}
\newcommand{\laplaceInv}[1]{\mathcal{L}^{-1}\Bqty{#1}\pqty{t}}
%Derivatives
\newcommand{\pdiff}[2]{\frac{\partial^{#2}}{\partial #1^{#2}}}

%lemma,theorem, proof
\newtheorem{theorem}{Theorem}[section]
\newtheorem{lemma}[theorem]{Lemma}
\newtheorem{definition}[theorem]{Definition}
\newtheorem{corollary}[theorem]{Corollary}

\numberwithin{equation}{section}

%\usepackage{minted}
%opening
\author{Mark Andrew Gerads: nazgand@gmail.com}

\title{Complex Derivative Root of Unity Exponential Sum Function related to Generalized Split-complex numbers}

\begin{document}
	
	\maketitle
	
	\begin{abstract}
		The goal of this paper is to analyze a class of functions which are equal to their own \(n\)th derivative. In particular, applying the derivative a complex number of times to these functions will be examined.
	\end{abstract}
	
	\section{As a sum of exponential functions}
	The reason this function is named $\rues$ is because it is a Complex Derivative Root of Unity Exponential Sum function.
	\begin{definition}
		Where $n\in\mathbb{Z}^+,\Bqty{d,x}\subset\mathbb{C}$:
		\begin{equation}
		\label{cdrues Exponential sum form}
		\cdrues_n\pqty{d,x}=
		\frac{1}{n}\sum _{k=1}^n \exp\pqty{xe^{2ki\pi/n}+2dki\pi/n}
		\end{equation}
	\end{definition}
	Consider the $m$th derivative with respect to $x$. It is apparent that, for $m\in\mathbb{Z}^{\geq 0},d\in\mathbb{C}$:
	\begin{equation}
	\pdiff{x}{m}\cdrues_n\pqty{d,x}=\cdrues_n\pqty{d+m,x}
	\end{equation}
	We can thus extend the definition of iterated derivative to an arbitrary complex number of iterations, $m\in\mathbb{C}$:
	\begin{equation}
	\pdiff{x}{m}\cdrues_n\pqty{0,x}=\cdrues_n\pqty{m,x}
	\end{equation}
	The functions are periodic:
	\begin{equation}
	\cdrues_n\pqty{d,x}=\cdrues_n\pqty{d+n,x}
	\end{equation}
	
	\section{Relation to earlier work}
	Where \(n\in\mathbb{Z}^+\land x\in\mathbb{C}\):
	\begin{equation}
	\cdrues_n\pqty{0,x}=
	\rues_n\pqty{x}=
	\sum_{k=0}^{\infty}\frac{x^{nk}}{\pqty{nk}!}
	\end{equation}

\end{document}
