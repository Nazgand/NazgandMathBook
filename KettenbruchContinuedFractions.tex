\documentclass[]{article}
%margins
\usepackage[a4paper,margin=0.15in]{geometry}
%document colors
\usepackage{xcolor}
\definecolor{favoritecolor1}{HTML}{607CB2}
\definecolor{favoritecolor2}{HTML}{303E59}
\makeatletter
\newcommand{\globalcolor}[1]{%
	\color{#1}\global\let\default@color\current@color
}
\makeatother

\AtBeginDocument{\globalcolor{favoritecolor2}}
\pagecolor{favoritecolor1}

%made from template named MathArticleTemplate
\usepackage{amsfonts}
\usepackage{amsmath}
\usepackage{amsthm}
\usepackage{amssymb}
\usepackage{hyperref}
\hypersetup{colorlinks=true}
\usepackage{graphics}

%Fix \eqref in section title
\pdfstringdefDisableCommands{\def\eqref#1{(\ref{#1})}}

\DeclareMathOperator{\es}{Es}
\DeclareMathOperator{\ez}{Ez}
\DeclareMathOperator{\gs}{gs}
\DeclareMathOperator{\md}{mod}
\DeclareMathOperator{\pow}{Pow}
%Parenthesis, Braces, Brackets usepackage{physics}
\newcommand{\pqty}[1]{{\left(#1\right)}}
\newcommand{\Bqty}[1]{{\left\{#1\right\}}}
\newcommand{\bqty}[1]{{\left[#1\right]}}
\newcommand{\abs}[1]{{\left\lvert#1\right\rvert}}
%other stuff
\newcommand{\floor}[1]{{\left\lfloor#1\right\rfloor}}
\newcommand{\ceil}[1]{{\left\lceil#1\right\rceil}}
%Laplace transform and inverse
\newcommand{\laplace}[1]{\mathcal{L}\Bqty{#1}\pqty{s}}
\newcommand{\laplaceInv}[1]{\mathcal{L}^{-1}\Bqty{#1}\pqty{t}}
%Derivatives
\newcommand{\pdiff}[2]{\frac{\partial^{#2}}{\partial #1^{#2}}}
%Kettenbruch
\newcommand{\ketten}[4]{\underset{#1}{\overset{#2}{\LARGE\mathrm K}}\frac{#3}{#4}}

%lemma,theorem, proof
\newtheorem{theorem}{Theorem}[section]
\newtheorem{lemma}[theorem]{Lemma}
\newtheorem{definition}[theorem]{Definition}
\newtheorem{corollary}[theorem]{Corollary}

\numberwithin{equation}{section}

%\usepackage{minted}
%opening
\author{Mark Andrew Gerads: \href{MailTo:Nazgand@Gmail.Com}{Nazgand@Gmail.Com}}

\title{
	Kettenbruch Continued Fractions
	
	\href{https://github.com/Nazgand/nazgandMathBook}{https://github.com/Nazgand/nazgandMathBook}
}

\begin{document}
	
	\maketitle
	
	\begin{abstract}
		The goal of this paper is to have fun reviewing Kettenbruch notation for continued fractions.
	\end{abstract}
	
	\section{Introduction}
	The 2 defining equations are:
	\begin{equation}
		\ketten{k=n}{n}{a_k}{b_k}=\cfrac{a_n}{b_n}
	\end{equation}
	\begin{equation}
		\ketten{k=n}{m}{a_k}{b_k}=\cfrac{a_n}{b_n+\ketten{k=n+1}{m}{a_k}{b_k}}
	\end{equation}

	The most interesting continued fractions are infinite continued fractions, like so, yet careful consideration must be taken in this case, e.g. because the limit might not exist:
	\begin{equation}
		\ketten{k=1}{\infty}{a_k}{b_k}=
		\lim_{m\to\infty}\ketten{k=1}{m}{a_k}{b_k}=
		\cfrac{a_1}{b_1+\cfrac{a_2}{b_2+\cfrac{a_3}{b_3+\ddots}}}
	\end{equation}

	\section{Examples}
	One of the simplest cases to consider is $a_k=a\in\mathbb{C},b_k=b\in\mathbb{C},b\neq 0$.
	\begin{equation}
		x=\ketten{k=1}{\infty}{a}{b}=\frac{a}{b+\ketten{k=1}{\infty}{a}{b}}
		\Rightarrow
		x^2+bx-a=0
	\end{equation}
	Here we run into a problem. The limit exists, but there are 2 values it might be. The simplest way to get around this problem is to choose $\Bqty{a,b}\subset\mathbb{R}^+$ such that 1 of the solutions is not in $\mathbb{R}^+$, thus proving the limit is the other solution which would be in $\mathbb{R}^+$. E.g.:

	\begin{equation}
		x=\ketten{k=1}{\infty}{1}{1}
		\Rightarrow
		x^2+x-1=0
		\Rightarrow
		x=\frac{-1\pm\sqrt{5}}{2}
		\Rightarrow
		\ketten{k=1}{\infty}{1}{1}=\frac{-1+\sqrt{5}}{2}
	\end{equation}

	A continued fraction I discovered without proof is:
	\begin{equation}
		x+\ketten{k=1}{\infty}{k}{x+k}=\frac{1}{e \gamma (x+1,1)}
		=\frac{1}{\int_{0}^{1}t^x e^{1-t}dt}
	\end{equation}
	where $\gamma$ is the lower incomplete gamma function.
\end{document}
