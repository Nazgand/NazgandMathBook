\documentclass[]{article}
%document colors
\usepackage{xcolor}
\definecolor{favoritecolor1}{HTML}{607CB2}
\definecolor{favoritecolor2}{HTML}{303E59}
\makeatletter
\newcommand{\globalcolor}[1]{%
	\color{#1}\global\let\default@color\current@color
}
\makeatother

\AtBeginDocument{\globalcolor{favoritecolor2}}
\pagecolor{favoritecolor1}

%made from template named MathArticleTemplate
\usepackage{amsfonts}
\usepackage{amsmath}
\usepackage{amsthm}
\usepackage{amssymb}
\usepackage{hyperref}
\hypersetup{colorlinks=true}
\usepackage{graphics}

%Fix \eqref in section title
\pdfstringdefDisableCommands{\def\eqref#1{(\ref{#1})}}

\DeclareMathOperator{\es}{Es}
\DeclareMathOperator{\ez}{Ez}
\DeclareMathOperator{\gs}{gs}
\DeclareMathOperator{\md}{mod}
\DeclareMathOperator{\pow}{Pow}
%Parenthesis, Braces, Brackets usepackage{physics}
\newcommand{\pqty}[1]{{\left(#1\right)}}
\newcommand{\Bqty}[1]{{\left\{#1\right\}}}
\newcommand{\bqty}[1]{{\left[#1\right]}}
\newcommand{\abs}[1]{{\left\lvert#1\right\rvert}}
%other stuff
\newcommand{\floor}[1]{{\left\lfloor#1\right\rfloor}}
\newcommand{\ceil}[1]{{\left\lceil#1\right\rceil}}
%Laplace transform and inverse
\newcommand{\laplace}[1]{\mathcal{L}\Bqty{#1}\pqty{s}}
\newcommand{\laplaceInv}[1]{\mathcal{L}^{-1}\Bqty{#1}\pqty{t}}
%Derivatives
\newcommand{\pdiff}[2]{\frac{\partial^{#2}}{\partial #1^{#2}}}

%lemma,theorem, proof
\newtheorem{theorem}{Theorem}[section]
\newtheorem{lemma}[theorem]{Lemma}
\newtheorem{definition}[theorem]{Definition}
\newtheorem{corollary}[theorem]{Corollary}

\numberwithin{equation}{section}

%\usepackage{minted}
%opening
\author{Mark Andrew Gerads: \href{MailTo:Nazgand@Gmail.Com}{Nazgand@Gmail.Com}}

\title{
	Cauchy Sequence Permutations
	
	\href{https://github.com/Nazgand/nazgandMathBook}{https://github.com/Nazgand/nazgandMathBook}
}

\begin{document}
	
	\maketitle
	
	\begin{abstract}
		The goal of this paper is to notice that [every reordering of a Cauchy Sequence] is a Cauchy Sequence.
	\end{abstract}
	
	Let a bijection exist on $\mathbb{N}_0$ (The set of natural numbers including 0), $perm:\mathbb{N}_0\to\mathbb{N}_0$. $perm$ is short for 'permutation', because every bijection from a set to itself is a permutation.
	
	Let a Cauchy Sequence exist, $cs:\mathbb{N}_0\to\mathbb{C}$, with the limit
	\begin{equation}
		l=\lim\limits_{k\to\infty}cs\pqty{k}\in\mathbb{C}
	\end{equation}

	Then
	\begin{equation}
		l=\lim\limits_{k\to\infty}cs\pqty{perm\pqty{k}}\in\mathbb{C}
	\end{equation}
	Let $\epsilon\in\mathbb{R}^+$. Then consider the set
	\begin{equation}
		OutsideCs\pqty{\epsilon}=\Bqty{k\in\mathbb{N}_0|\abs{cs\pqty{k}-l}>\epsilon}
	\end{equation}
	Notice $OutsideCs\pqty{\epsilon}$ is finite [because if $OutsideCs\pqty{\epsilon}$ were infinite in cardinality, then the limit could not be $l$]. Maximums of finite non-empty sets are well defined, so...
	\begin{equation}
		csSkip\pqty{\epsilon}=\max\pqty{\Bqty{-1}\cup OutsideCs\pqty{\epsilon}}+1
	\end{equation}
	This function tells [how much of the Cauchy Series to skip] such that [the new Cauchy Sequence created by the skip] in a precision circle of size $\epsilon$.
	\begin{equation}
		cs_2\pqty{k,\epsilon}=cs\pqty{k+csSkip\pqty{\epsilon}}
	\end{equation}
	
	\begin{equation}
		\bqty{k\in\mathbb{N}_0,\epsilon\in\mathbb{R}^+}\Rightarrow \abs{cs_2\pqty{k,\epsilon}-l}\leq\epsilon
	\end{equation}

	We need to create an analogous $csPermSkip\pqty{\epsilon}$ function for the second series, requiring
	\begin{equation}
		cs_3\pqty{k,\epsilon}=cs\pqty{perm\pqty{k+csPermSkip\pqty{\epsilon}}}
	\end{equation}
	
	\begin{equation}
		\bqty{k\in\mathbb{N}_0,\epsilon\in\mathbb{R}^+}\Rightarrow \abs{cs_3\pqty{k,\epsilon}-l}\leq\epsilon
	\end{equation}

	\begin{equation}
		OutsideCsPerm\pqty{\epsilon}=\Bqty{k\in\mathbb{N}_0|\abs{cs\pqty{perm\pqty{k}}-l}>\epsilon}
	\end{equation}
	It should be noticed that
	\begin{equation}
		OutsideCsPerm\pqty{\epsilon}=\Bqty{k\in\mathbb{N}_0|perm\pqty{k}\in OutsideCs\pqty{\epsilon}}
	\end{equation}
	The finiteness of $OutsideCsPerm\pqty{\epsilon}$ is enough to show the limit is $l$, yet we explicitly define the skip function:
	\begin{equation}
		csPermSkip\pqty{\epsilon}=\max\pqty{\Bqty{-1}\cup OutsideCsPerm\pqty{\epsilon}}+1
	\end{equation}
	Equivalently, 
	\begin{equation}
		csPermSkip\pqty{\epsilon}=\max\pqty{\Bqty{-1}\cup \Bqty{perm^{-1}\pqty{k}|k\in OutsideCs\pqty{\epsilon}}}+1
	\end{equation}
	\qed
\end{document}
