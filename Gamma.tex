\documentclass[]{article}
%made from template named MathArticleTemplate
\usepackage{amsfonts}
\usepackage{amsmath}
\usepackage{amsthm}
\usepackage{amssymb}
\usepackage{hyperref}
\hypersetup{colorlinks=true}
\usepackage{graphics}

%Fix \eqref in section title
\pdfstringdefDisableCommands{\def\eqref#1{(\ref{#1})}}

\DeclareMathOperator{\es}{Es}
\DeclareMathOperator{\ez}{Ez}
\DeclareMathOperator{\gs}{gs}
\DeclareMathOperator{\md}{mod}
\DeclareMathOperator{\pow}{Pow}
%Parenthesis, Braces, Brackets usepackage{physics}
\newcommand{\pqty}[1]{{\left(#1\right)}}
\newcommand{\Bqty}[1]{{\left\{#1\right\}}}
\newcommand{\bqty}[1]{{\left[#1\right]}}
\newcommand{\abs}[1]{{\left\lvert#1\right\rvert}}
%other stuff
\newcommand{\floor}[1]{{\left\lfloor#1\right\rfloor}}
\newcommand{\ceil}[1]{{\left\lceil#1\right\rceil}}
%Laplace transform and inverse
\newcommand{\laplace}[1]{\mathcal{L}\Bqty{#1}\pqty{s}}
\newcommand{\laplaceInv}[1]{\mathcal{L}^{-1}\Bqty{#1}\pqty{t}}
%Derivatives
\newcommand{\pdiff}[2]{\frac{\partial^{#2}}{\partial #1^{#2}}}

%lemma,theorem, proof
\newtheorem{theorem}{Theorem}[section]
\newtheorem{lemma}[theorem]{Lemma}
\newtheorem{definition}[theorem]{Definition}
\newtheorem{corollary}[theorem]{Corollary}

\numberwithin{equation}{section}

%\usepackage{minted}
%opening
\author{Mark Andrew Gerads: \href{MailTo:MarkAndrewGerads.Nazgand@Gmail.Com}{MarkAndrewGerads.Nazgand@Gmail.Com}}

\title{
	Gamma Function
	
	\href{https://github.com/Nazgand/nazgandMathBook}{https://github.com/Nazgand/nazgandMathBook}
}

\begin{document}
	
	\maketitle
	
	\begin{abstract}
		The goal of this paper is to review the gamma function.
	\end{abstract}
	
	\section{Definition}
	\begin{definition}
		\begin{equation}
		\label{defGamma}
		\Gamma\pqty{a+1}=\int_{t=0}^\infty t^a e^{-t} \partial t
		\end{equation}
	\end{definition}
	
	\section{Convergence of Integral}
	For \(t^a e^{-t}\) shrinks exponentially quickly as \(t\to\infty\), thus the integral \(\int_{t=1}^\infty t^a e^{-t} \partial t\) converges.
	The integral \(\int_{t=0}^1 t^a e^{-t} \partial t\) only converges for \(a>-1\).
	
	\section{Recursive Property}
	From \eqref{defGamma}, integrate by parts with \(u=t^a, v=-e^{-t}\)
	\begin{equation}
	\Gamma\pqty{a+1}=\bqty{-t^ae^{-t}}_{t=0}^\infty - 
	\int_{t=0}^\infty -e^{-t}at^{a-1} \partial t
	\end{equation}
	Simplify
	\begin{equation}
	\Gamma\pqty{a+1}= 
	a\int_{t=0}^\infty e^{-t}t^{a-1} \partial t= 
	a\Gamma\pqty{a}
	\end{equation}

\end{document}
