\documentclass{article}
\usepackage{amsmath, amssymb, amsthm}

\newtheorem{theorem}{Theorem}
\newtheorem{lemma}[theorem]{Lemma}
\newtheorem{claim}[theorem]{Claim}

\title{Rigorous Proof of the 2-Argument Matrix Conjecture for Homogeneous Linear Differential Equations with Constant Coefficients}
\author{Assistant}

\begin{document}

\maketitle

\section{Introduction}

We consider a homogeneous linear differential equation with constant coefficients:

\[
0 = \sum_{k=0}^n a_k f^{(k)}(z), \quad a_k \in \mathbb{C}, a_n \neq 0
\]

Let $\{g_0, \ldots, g_{n-1}\}$ be a basis for the solution space of this equation.

\section{The Conjecture}

For any solution $f$ of the differential equation, there exists a unique symmetric matrix $A \in \mathbb{C}^{n \times n}$ such that:

\[
f(z_0 + z_1) = v(z_1)^T A v(z_0)
\]

where $v(z) = [g_0(z), \ldots, g_{n-1}(z)]^T$, and $z_0, z_1 \in \mathbb{C}$.

\section{Proof}

We will prove this conjecture through a series of lemmas and claims.

\begin{lemma}
For any solution $f$ and any fixed $z_1 \in \mathbb{C}$, the function $h(z_0) := f(z_0 + z_1)$ is also a solution.
\end{lemma}

\begin{proof}
This follows from the linearity and constant coefficients of the differential equation:
\[
\sum_{k=0}^n a_k h^{(k)}(z_0) = \sum_{k=0}^n a_k f^{(k)}(z_0 + z_1) = 0
\]
\end{proof}

\begin{lemma}
For any solution $f$, there exist continuous functions $c_k: \mathbb{C} \to \mathbb{C}$ such that:
\[
f(z_0 + z_1) = \sum_{k=0}^{n-1} c_k(z_1) g_k(z_0)
\]
for all $z_0, z_1 \in \mathbb{C}$.
\end{lemma}

\begin{proof}
Fix $z_1$. By Lemma 1, $h(z_0) = f(z_0 + z_1)$ is a solution. Since $\{g_k\}$ is a basis for the solution space, we can write:
\[
h(z_0) = \sum_{k=0}^{n-1} c_k(z_1) g_k(z_0)
\]
where $c_k(z_1)$ are constants with respect to $z_0$. As $z_1$ varies, these constants become functions of $z_1$. The continuity of $c_k(z_1)$ follows from the continuity of solutions to linear differential equations with constant coefficients.
\end{proof}

\begin{claim}
The functions $c_k(z_1)$ satisfy the same differential equation as $f$, i.e.,
\[
\sum_{j=0}^n a_j c_k^{(j)}(z_1) = 0
\]
for all $k = 0, \ldots, n-1$ and all $z_1 \in \mathbb{C}$.
\end{claim}

\begin{proof}
Apply the differential operator $\sum_{j=0}^n a_j \frac{\partial^j}{\partial z_1^j}$ to both sides of the equation in Lemma 2:
\[
0 = \sum_{j=0}^n a_j f^{(j)}(z_0 + z_1) = \sum_{k=0}^{n-1} \left(\sum_{j=0}^n a_j c_k^{(j)}(z_1)\right) g_k(z_0)
\]

Let $h_k(z_1) = \sum_{j=0}^n a_j c_k^{(j)}(z_1)$. We have:

\[
\sum_{k=0}^{n-1} h_k(z_1) g_k(z_0) = 0 \quad \forall z_0, z_1 \in \mathbb{C}
\]

Since $\{g_k\}$ form a basis for the solution space, they are linearly independent. Therefore, the only way for this equation to hold for all $z_0$ is if all $h_k(z_1)$ are identically zero. Thus, $\sum_{j=0}^n a_j c_k^{(j)}(z_1) = 0$ for all $z_1$ and all $k$.
\end{proof}

\begin{lemma}
There exist constants $d_{km} \in \mathbb{C}$ such that:
\[
c_k(z_1) = \sum_{m=0}^{n-1} d_{km} g_m(z_1)
\]
for all $k = 0, \ldots, n-1$ and all $z_1 \in \mathbb{C}$.
\end{lemma}

\begin{proof}
This follows directly from the fact that $c_k(z_1)$ satisfies the same differential equation as $f$ (proven in the previous claim) and that $\{g_m\}$ is a basis for the solution space of this equation.
\end{proof}

\begin{theorem}
The 2-argument matrix conjecture holds.
\end{theorem}

\begin{proof}
1) From Lemma 2 and Lemma 3, we can write:
\[
f(z_0 + z_1) = \sum_{k=0}^{n-1} \sum_{m=0}^{n-1} d_{km} g_m(z_1) g_k(z_0)
\]

2) Define matrix $A$ with entries $A_{km} = d_{km}$. Then:
\[
f(z_0 + z_1) = v(z_1)^T A v(z_0)
\]

3) To prove $A$ is symmetric, note that $f(z_0 + z_1) = f(z_1 + z_0)$, which implies:
\[
v(z_1)^T A v(z_0) = v(z_0)^T A v(z_1) = v(z_1)^T A^T v(z_0)
\]
As this holds for all $z_0, z_1$, and $\{g_k\}$ form a basis, we must have $A = A^T$.

4) Uniqueness follows from the linear independence of $\{g_k\}$. If there were two such matrices $A$ and $B$, then:
\[
v(z_1)^T (A-B) v(z_0) = 0 \quad \forall z_0, z_1
\]
This implies $A-B = 0$, hence $A = B$.
\end{proof}

\section{Conclusion}

This proof rigorously establishes the 2-argument matrix conjecture for homogeneous linear differential equations with constant coefficients. The key steps involve showing that translated solutions and coefficient functions satisfy the original differential equation, leveraging the linear independence of the basis functions, and carefully constructing the symmetric matrix $A$.

\section{Implications and Applications}

This result has several important implications:

1. It provides a compact representation of solutions to linear differential equations, potentially simplifying their analysis and manipulation.

2. The symmetric matrix $A$ encodes all the information about a particular solution, which could be useful in classification or comparison of solutions.

3. This representation might lead to new numerical methods for solving or approximating solutions to linear differential equations.

4. In the context of linear systems theory, this result could provide insights into the structure of impulse responses and transfer functions.

5. The proof technique used here might be adaptable to other classes of differential equations or functional equations.

Further research could explore generalizations to non-homogeneous equations, partial differential equations, or equations with non-constant coefficients.

\end{document}
