\documentclass[]{article}
%made from template named MathArticleTemplate
\usepackage{amsfonts}
\usepackage{amsmath}
\usepackage{amsthm}
\usepackage{amssymb}
\usepackage{hyperref}
\hypersetup{colorlinks=true}
\usepackage{graphics}

%Fix \eqref in section title
\pdfstringdefDisableCommands{\def\eqref#1{(\ref{#1})}}

\DeclareMathOperator{\es}{Es}
\DeclareMathOperator{\ez}{Ez}
\DeclareMathOperator{\gs}{gs}
\DeclareMathOperator{\md}{mod}
\DeclareMathOperator{\pow}{Pow}
%Parenthesis, Braces, Brackets usepackage{physics}
\newcommand{\pqty}[1]{{\left(#1\right)}}
\newcommand{\Bqty}[1]{{\left\{#1\right\}}}
\newcommand{\bqty}[1]{{\left[#1\right]}}
\newcommand{\abs}[1]{{\left\lvert#1\right\rvert}}
%other stuff
\newcommand{\floor}[1]{{\left\lfloor#1\right\rfloor}}
\newcommand{\ceil}[1]{{\left\lceil#1\right\rceil}}
%Laplace transform and inverse
\newcommand{\laplace}[1]{\mathcal{L}\Bqty{#1}\pqty{s}}
\newcommand{\laplaceInv}[1]{\mathcal{L}^{-1}\Bqty{#1}\pqty{t}}
%Derivatives
\newcommand{\pdiff}[2]{\frac{\partial^{#2}}{\partial #1^{#2}}}

%lemma,theorem, proof
\newtheorem{theorem}{Theorem}[section]
\newtheorem{lemma}[theorem]{Lemma}
\newtheorem{definition}[theorem]{Definition}
\newtheorem{corollary}[theorem]{Corollary}

\numberwithin{equation}{section}

\usepackage{xr}
\externaldocument{GeometricSeries}
\externaldocument{Logarithms}
\externaldocument{Trigonometry}

%\usepackage{minted}
%opening
\author{Mark Andrew Gerads: \href{MailTo:MarkAndrewGerads.Nazgand@Gmail.Com}{MarkAndrewGerads.Nazgand@Gmail.Com}}

\title{
	Exponential function
	
	\href{https://github.com/Nazgand/nazgandMathBook}{https://github.com/Nazgand/nazgandMathBook}
}

\begin{document}
	
	\maketitle
	
	\begin{abstract}
		The goal of this paper is to have fun reviewing basic calculus.
	\end{abstract}
	
	\section{Power function definition}
	\begin{definition}
			Here I formally define the power function.
		\begin{equation}
			\pow\pqty{x,y}=x^y
		\end{equation}
		\begin{equation}
			\pow\pqty{x,0}=1
		\end{equation}
		\begin{equation}
			\exists\pow\pqty{x,y-1}\implies\pow\pqty{x,y}=\pow\pqty{x,y-1}*x
		\end{equation}
		\begin{equation}
			{\exists\pow\pqty{x,b}^{-1}}\implies\pow\pqty{x,-b}=\pow\pqty{x,b}^{-1}
		\end{equation}
		\begin{equation}
			\bqty{\exists\pow\pqty{x,a}\land\exists\pow\pqty{x,b}}\implies\pow\pqty{x,a+b}=\pow\pqty{x,a}\pow\pqty{x,b}
		\end{equation}
		\begin{equation}
			\bqty{x\in\mathbb{R}^+ \land y\in\mathbb{R}}\implies\exists\pow\pqty{x,y}\in\mathbb{R}
		\end{equation}
		\begin{equation}
			\bqty{x\in\mathbb{R}^+\land y\in\mathbb{R}}
			\implies\exists\pqty{\pdiff{z}{}\pow\pqty{x,z}:z\to y}
		\end{equation}
	\end{definition}
	
	\section{Basic results and definition of \(e\)}
	\begin{theorem}
		The derivative of an exponential function is a multiple of the same exponential function.
		\begin{equation}
		\pdiff{y}{}x^y=\pqty{\pdiff{z}{}x^z:z\to 0}x^y
		\end{equation}
	\end{theorem}
	\begin{proof}
		Substitute \(y\to z+a\) in the left side of the equation to get the right side
		\begin{equation}
		\pdiff{y}{}x^y=\pqty{\pdiff{z}{}x^{z+a}:z\to y-a}
		\end{equation}
		Substitute \(x^{z+a}\to x^zx^a\)
		\begin{equation}
		\pdiff{y}{}x^y=\pqty{\pdiff{z}{}x^zx^a:z\to y-a}
		\end{equation}
		Bring \(x^a\) out:
		\begin{equation}
		\pdiff{y}{}x^y=x^a\pqty{\pdiff{z}{}x^z:z\to y-a}
		\end{equation}
		Substitute \(a\to y\)
		\begin{equation}
		\label{expDiffMultOfExp}
		\pdiff{y}{}x^y=x^y\pqty{\pdiff{z}{}x^z:z\to 0}
		\end{equation}
	\end{proof}

	\begin{definition}
		Define \(e\) to be the base of the exponential function which has a derivative of \(1\) at \(0\).
		\begin{equation}
		\label{defineE}
			1=\pqty{\pdiff{z}{}e^z:z\to 0}
		\end{equation}
	\end{definition}

	\begin{lemma}
		\begin{equation}
		\label{expOwnDeriv}
		\pdiff{y}{}e^y=e^y
		\end{equation}
	\end{lemma}
	\begin{proof}
		Substitute \(x\to e\) in \eqref{expDiffMultOfExp} and simplify with \eqref{defineE}.
	\end{proof}

	\begin{corollary}
		The derivative of an exponential function is the natural logarithm of the base of the exponential function times the exponential function.
		\begin{equation}
			\pdiff{y}{}x^y
			={\ln\pqty{x}}x^y
		\end{equation}
	\end{corollary}
	\begin{proof}
		\begin{equation}
		\pqty{\pdiff{z}{}x^z:z\to 0}=\ln\pqty{x}
		\end{equation}
		\begin{equation}
		\pdiff{y}{}x^y=x^y\pqty{\pdiff{z}{}x^z:z\to 0}
		\end{equation}
		Substitute \(x\to e^{\ln\pqty{x}}\) in the left to get the right
		\begin{equation}
		\pdiff{y}{}x^y=\pdiff{y}{}e^{\ln\pqty{x}y}
		\end{equation}
		Apply the chain rule:
		\begin{equation}
		\pdiff{y}{}x^y
		=\pqty{\pdiff{z}{}e^{z}:z\to \ln\pqty{x}y}*\pdiff{y}{}{\ln\pqty{x}y}
		\end{equation}
		Simplify
		\begin{equation}
		\pdiff{y}{}x^y
		=\pqty{e^{z}:z\to \ln\pqty{x}y}*{\ln\pqty{x}}
		=e^{\ln\pqty{x}y}*{\ln\pqty{x}}
		={\ln\pqty{x}}x^y
		\end{equation}
		
	\end{proof}

	\section{Exponential Function Derivative}
	\begin{theorem}
		\begin{equation}
			e^x=\sum_{k=0}^{\infty}\frac{x^k}{k!}
		\end{equation}
	\end{theorem}
	\begin{proof}
		We have a formula which is it's own derivative \eqref{expOwnDeriv}.
		Another formula which is it's own derivative is
		\begin{equation}
			\exp\pqty{x}
			=\sum_{k=0}^{\infty}\frac{x^k}{k!}
			=\pdiff{x}{}\exp\pqty{x}
		\end{equation}
		The differential equation \(f(x)=f'(x)\) has 1 degree of freedom which is filled by \(f(0)=1\) by both formulae. Thus both formulae express the same function; \(e^x=\exp(x)\).
	\end{proof}

	\section{Convergence of \(\exp(x)\)}
	\begin{theorem}
		\(\exp\pqty{x}\) converges for all \(x\in\mathbb{C}\)
	\end{theorem}
	\begin{proof}
		By the triangle inequality, an upper bound and a lower bound exist for all complex numbers.
		\begin{equation}
		-\frac{\abs{x}^k}{k!}\leq
		\frac{x^k}{k!}\leq
		\frac{\abs{x}^k}{k!}
		\end{equation}
		\begin{equation}
		-\exp\pqty{\abs{x}}\leq
		\exp\pqty{x}\leq
		\exp\pqty{\abs{x}}
		\end{equation}
		Thus convergence for \(x\in\mathbb{R}^+\) implies convergence for \(x\in\mathbb{C}\). Let \(x\in\mathbb{R}^+\). Bound part of the sum by a geometric series:
		\begin{equation}
		\exp\pqty{x}
		=\sum_{k=0}^{n-1}\frac{x^k}{k!}
		+\sum_{k=n}^{\infty}\frac{x^k}{k!}
		<\sum_{k=0}^{n-1}\frac{x^k}{k!}
		+\sum_{k=n}^{\infty}\frac{x^k}{n^{k-n}\pqty{n}!}
		\end{equation}
		Simplify
		\begin{equation}
		\sum_{k=n}^{\infty}\frac{x^k}{n^{k-n}\pqty{n}!}
		=\frac{n^{n}}{\pqty{n}!}\sum_{k=n}^{\infty}\pqty{\frac{x}{n}}^k
		=\frac{x^{n}}{\pqty{n}!}\sum_{m=0}^{\infty}\pqty{\frac{x}{n}}^{m}
		\end{equation}
		Find where the bounding geometric series converges. [GeometricSeries\eqref{GeometricSeries}]
		\begin{equation}
		\frac{x}{n}<1
		\Rightarrow
		\sum_{m=0}^{\infty}\pqty{\frac{x}{n}}^{m}
		=\frac{1}{1-\frac{x}{n}}
		\end{equation}
		
		Every specific \(x\) has an integer larger than it and is bounded by a circle of convergence from a corresponding geometric series. Let \(n\to\infty\) and \(\exp\pqty{x}\) converges for all \(x\in\mathbb{C}\).
	\end{proof}

	\section{Limit Form of E}
	\begin{theorem}
		\begin{equation}
		\label{limitE}
		e=\lim\limits_{n\to\infty}\pqty{1+\frac{1}{n}}^n
		=\lim\limits_{n\to\infty}\pqty{\frac{1+n}{n}}^n
		\end{equation}
	\end{theorem}
	\begin{proof}
		Proof from \url{https://mathcs.clarku.edu/~djoyce/ma122/elimit.pdf}
		Note for
		\begin{equation}
			1\leq t\leq \frac{1+n}{n}
			\Rightarrow
			1\geq \frac{1}{t}\geq \frac{n}{1+n}
		\end{equation}
		Integrate over the inequality:
		\begin{equation}
			\int_{1}^{\frac{1+n}{n}} 1\partial t
			\geq \int_{1}^{\frac{1+n}{n}} \frac{1}{t}\partial t
			\geq \int_{1}^{\frac{1+n}{n}} \frac{n}{1+n}\partial t
		\end{equation}
		Simplify using [Logarithms\eqref{LnIntegralForm}]
		\begin{equation}
			\frac{1}{n}
			\geq \ln\pqty{\frac{1+n}{n}}
			\geq \frac{1}{n+1}
		\end{equation}
		Apply the exponential function:
		\begin{equation}
			e^{\frac{1}{n}}
			\geq \frac{1+n}{n}
			\geq e^{\frac{1}{n+1}}
		\end{equation}
		Raise to the power of \(n\) and \(n+1\)
		\begin{equation}
			e
			\geq \pqty{\frac{1+n}{n}}^n
			\land
			\pqty{\frac{1+n}{n}}^{n+1}
			\geq e
		\end{equation}
		Divide
		\begin{equation}
			e
			\geq \pqty{\frac{1+n}{n}}^n
			\land
			\pqty{\frac{1+n}{n}}^n
			\geq \frac{en}{1+n}
		\end{equation}
		Let \(n\to\infty\) using the squeeze theorem, simplify.
		\begin{equation}
			e
			\geq \lim\limits_{n\to\infty}\pqty{\frac{1+n}{n}}^n
			\geq e
		\end{equation}
		
	\end{proof}
	
	\section{Exponential Function Limit Form}
	\begin{theorem}
		\begin{equation}
			\label{expLim2}
			e^x=\lim\limits_{m\to\infty}\pqty{1+\frac{x}{m}}^m
		\end{equation}
	\end{theorem}
	\begin{proof}
		Raise \eqref{limitE} to the power of \(x\)
		\begin{equation}
			e^x=\lim\limits_{n\to\infty}\pqty{1+\frac{1}{n}}^{xn}
		\end{equation}
		For \(x\in\mathbb{R}^+\), a substitution \(n\to \frac{m}{x}\) can be made to obtain a limit known to exist.
		\begin{equation}
			e^x=\lim\limits_{m\to\infty}\pqty{1+\frac{x}{m}}^m
		\end{equation}
		The existence of the limit for \(x\in\mathbb{R}^+\) extends analytically to \(x\in\mathbb{C}\) because the new formula fulfills the differential equation \(f(x)=f'(x)\) and has \(f(0)=1\).
		Chain rule used:
		\begin{equation}
		\pdiff{x}{}\lim\limits_{m\to\infty}\pqty{1+\frac{x}{m}}^m
		=
		\lim\limits_{m\to\infty}\pqty{\pdiff{z}{}z^m:z\to \pqty{1+\frac{x}{m}}}*\pdiff{x}{}\pqty{1+\frac{x}{m}}
		\end{equation}
		Simplify:
		\begin{equation}
		\pdiff{x}{}\lim\limits_{m\to\infty}\pqty{1+\frac{x}{m}}^m
		=
		\lim\limits_{m\to\infty}m\pqty{1+\frac{x}{m}}^{m-1}*\frac{1}{m}
		\end{equation}
		Split the limit
		\begin{equation}
		\pdiff{x}{}\lim\limits_{m\to\infty}\pqty{1+\frac{x}{m}}^m
		=
		\lim\limits_{m\to\infty}\pqty{1+\frac{x}{m}}^{m}*\pqty{1+\frac{x}{m}}^{-1}
		\end{equation}
		\begin{equation}
		\pdiff{x}{}\lim\limits_{m\to\infty}\pqty{1+\frac{x}{m}}^m
		=
		\lim\limits_{m\to\infty}\pqty{1+\frac{x}{m}}^{m}*
		\lim\limits_{p\to\infty}\pqty{1+\frac{x}{p}}^{-1}
		\end{equation}
		Simplify to see \(f(x)=f'(x)\).
		\begin{equation}
		\pdiff{x}{}\lim\limits_{m\to\infty}\pqty{1+\frac{x}{m}}^m
		=
		\lim\limits_{m\to\infty}\pqty{1+\frac{x}{m}}^{m}
		\end{equation}
	\end{proof}
	
	\section{Euler's Identity}
	\begin{theorem}
		\begin{equation}
		e^{ix}=\cos\pqty{x}+i\sin\pqty{x}
		\end{equation}
	\end{theorem}
	\begin{proof}
		\begin{equation}
			\pdiff{x}{n}e^{ix}=i^ne^{ix}
		\end{equation}
		The derivatives at 0 thus cycle through \(1,i,-1,-i\). Use [TaylorSeries\eqref{TaylorSeriesDef}] and compare to [Trigonometry\eqref{SinTaylor}] and [Trigonometry\eqref{CosTaylor}]
	\end{proof}
	
	\section{Bibliography}
	\url{https://en.wikipedia.org/wiki/Exponential_function}
	
\end{document}
