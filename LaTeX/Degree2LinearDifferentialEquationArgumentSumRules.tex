%!TEX engine = lualatex
\documentclass{article}

% --- CORE MATH & LOGIC ---
\usepackage{amsmath}
\usepackage{amsthm}
\usepackage{amssymb}
\usepackage{amsfonts}
\usepackage{physics}
\usepackage{mathtools}

% --- TYPOGRAPHY & LAYOUT ---
\usepackage[a4paper,margin=0.0931547in]{geometry}
\usepackage{fontspec}
\usepackage{microtype}
\usepackage[skip=0.5\baselineskip]{parskip}

% --- COLORS ---
\usepackage{xcolor}
% Define Palette
\definecolor{FavoriteColor1}{HTML}{607CB2} % Background (Medium Blue)
\definecolor{FavoriteColor2}{HTML}{303E59} % Text/Frame (Dark Blue)
\definecolor{LinkColor}{HTML}{FFFF00}	  % Links (Bright Yellow)

% Apply Background to Page
\pagecolor{FavoriteColor1}

% Apply Default Text Color (Global)
\makeatletter
\newcommand{\globalcolor}{\color{FavoriteColor2}\global\let\default@color\current@color}
\makeatother
\AtBeginDocument{\globalcolor}

% --- THEOREM STYLING ---
\usepackage[many]{tcolorbox}

% Define the box style
\tcbset{
	nazgandbox/.style={
		enhanced,
		% Background matches page color exactly (looks transparent)
		colback=FavoriteColor1,
		% Frame matches text color
		colframe=FavoriteColor2,
		% Text inside box matches global text color
		coltext=FavoriteColor2,
		% Title text matches background (high contrast against dark frame)
		coltitle=FavoriteColor1,
		fonttitle=\bfseries\sffamily,
		boxrule=1pt,
		arc=2pt,
		left=5pt, right=5pt, top=5pt, bottom=5pt,
		boxsep=1pt,
		sharp corners=downhill,
		breakable
	}
}

% Standard Definitions
\theoremstyle{definition}
\newtheorem{theorem}{Theorem}[section]
\newtheorem{lemma}[theorem]{Lemma}
\newtheorem{definition}[theorem]{Definition}
\newtheorem{corollary}[theorem]{Corollary}

% Wrap standard theorems in the custom box
\tcolorboxenvironment{theorem}{nazgandbox}
\tcolorboxenvironment{lemma}{nazgandbox}
\tcolorboxenvironment{definition}{nazgandbox}
\tcolorboxenvironment{corollary}{nazgandbox}

% --- UTILITIES ---
\usepackage{datetime2}
\usepackage{minted}

% --- FONTS ---
\directlua{luaotfload.add_fallback("EmojiFallback",{"NotoColorEmoji:mode=harf;"})}
\setmainfont{Noto Serif}[RawFeature={fallback=EmojiFallback}, Ligatures=TeX]
\setsansfont{Noto Sans}[RawFeature={fallback=EmojiFallback}]
\setmonofont{Noto Sans Mono}[RawFeature={fallback=EmojiFallback}]

% --- REFERENCES & LINKS ---
\usepackage{xr}
\usepackage{hyperref}
\hypersetup{
	colorlinks=true,
	linkcolor=LinkColor, % Internal links (equations, sections)
	urlcolor=LinkColor,  % Web URLs
	citecolor=LinkColor, % Bibliography citations
	filecolor=LinkColor, % Links to local files
	pdfnewwindow=true
}
\usepackage{cleveref}

% --- CUSTOM MACROS ---
\pdfstringdefDisableCommands{\def\eqref#1{(\ref{#1})}}
\newcommand{\inlineeqnum}{\refstepcounter{equation}~~\mbox{(\theequation)}}
\numberwithin{equation}{section}

% Operators
\DeclareMathOperator{\arsinh}{arsinh}
\DeclareMathOperator{\arcosh}{arcosh}
\DeclareMathOperator{\artanh}{artanh}
\DeclareMathOperator{\rues}{Rues}
\DeclareMathOperator{\md}{Mod}
\DeclareMathOperator{\pow}{Pow}

% Helpers
\newcommand{\floor}[1]{{\left\lfloor#1\right\rfloor}}
\newcommand{\ceil}[1]{{\left\lceil#1\right\rceil}}
\newcommand{\transpose}{^\intercal}
\newcommand{\laplace}[1]{\mathcal{L}\Bqty{#1}\pqty{s}}
\newcommand{\laplaceInv}[1]{\mathcal{L}^{-1}\Bqty{#1}\pqty{t}}
\newcommand{\pdiff}[2]{\frac{\partial^{#2}}{\partial #1^{#2}}}
\newcommand{\dint}[4]{\int_{#1}^{#2}#3\,\mathrm{d}#4}
\newcommand{\ketten}[4]{\underset{#1}{\overset{#2}{\mathop{\vcenter{\hbox{\huge\(\mathrm{K}\)}}}}}\frac{#3}{#4}}
\newcommand{\replace}[2]{\Big\vert_{#1\to{#2}}}

% --- METADATA & TITLE ---
\author{Mark Andrew Gerads \(<\)\href{MailTo:Nazgand@Gmail.Com}{Nazgand@Gmail.Com}\(>\)}
\let\oldauthor\author
\renewcommand{\author}[1]{
	\oldauthor{
		Author: #1
		\\
		Editor: Mark Andrew Gerads \(<\)\href{MailTo:Nazgand@Gmail.Com}{Nazgand@Gmail.Com}\(>\)
	}
}
\date{\DTMnow}
\let\oldtitle\title
\renewcommand{\title}[1]{
	\oldtitle{
		\vspace{-1.5cm}
		\url{https://GitHub.Com/Nazgand/NazgandMathBook}
		\\
		#1
	}
}

% --- LUA AUTO-REFERENCER ---
\usepackage{luacode}
\begin{luacode*}
	local lfs = require("lfs")

	local function split_path(path)
		local t = {}
		for part in path:gmatch("[^/]+") do table.insert(t, part) end
		return t
	end

	local function get_relative_path(base, target)
		local b_parts = split_path(base)
		local t_parts = split_path(target)
		local i = 1
		while b_parts[i] and t_parts[i] and b_parts[i] == t_parts[i] do
			i = i + 1
		end
		local rel = {}
		for j = i, #b_parts do table.insert(rel, "..") end
		for j = i, #t_parts do table.insert(rel, t_parts[j]) end
		return table.concat(rel, "/")
	end

	local handle = io.popen("git rev-parse --show-toplevel 2>/dev/null")
	local git_root = handle:read("*a"):gsub("%s+$", "")
	handle:close()

	if git_root ~= "" then
		local project_root = git_root .. "/LaTeX"
		local cwd = lfs.currentdir()
		
		local function scan_recursive(path)
			if not lfs.attributes(path) then return end
			for file in lfs.dir(path) do
				if file ~= "." and file ~= ".." then
					local full_path = path .. "/" .. file
					local f_attr = lfs.attributes(full_path)
					
					if f_attr and f_attr.mode == "directory" then
						scan_recursive(full_path)
					elseif file:match("%.tex$") then
						local job = file:gsub("%.tex$", "")
						if job ~= tex.jobname and job ~= "NazgandStyle" then
							local target_base = full_path:gsub("%.tex$", "")
							local rel_path = get_relative_path(cwd, target_base)
							
							tex.sprint("\\externaldocument{" .. rel_path .. "}[" .. rel_path .. ".pdf]")
						end
					end
				end
			end
		end

		texio.write_nl("--- [XR] Auto-Linking ---")
		scan_recursive(project_root)
	end
\end{luacode*}
\title{
	Degree 2 Differential Equation Argument Sum Rules
}
\begin{document}
	
	\maketitle
	
	\section{Assumptions and definitions}
	A degree 2 [homogeneous linear differential equation of constant coefficients] has the form
	\begin{equation}
		\label{DifferentialEquation}
		0=\sum_{k=0}^{2}a_k f^\pqty{k}\pqty{z}
	\end{equation}
	where \(\forall k, a_k\in\mathbb{C}, a_2 \neq 0\), and \(f:\mathbb{C}\to\mathbb{C}\) is differentiable infinitely many times.
	
	Let 
	\begin{equation}
		\lambda_0=\frac{-a_1-\sqrt{a_1^2-4 a_0 a_2}}{2 a_2},
		\lambda_1=\frac{-a_1+\sqrt{a_1^2-4 a_0 a_2}}{2 a_2}
	\end{equation}
	
	\section{Case \(\lambda_0\neq\lambda_1\)}
	\begin{equation}
		f\pqty{z}=c_0 \exp\pqty{\lambda_0 z}+c_1 \exp\pqty{\lambda_1 z},
		c_0=\frac{f^\pqty{1}\pqty{0}-f\pqty{0}\lambda_1}{\lambda_0-\lambda_1},
		c_1=\frac{f^\pqty{1}\pqty{0}-f\pqty{0}\lambda_0}{\lambda_1-\lambda_0}
	\end{equation}
	
	We can choose the basis of the set of solutions to be \(\Bqty{\exp\pqty{\lambda_0 z}, \exp\pqty{\lambda_1 z}}\).
	Let
	\begin{equation}
		v_0\pqty{z}=
		\begin{pmatrix}
			\exp\pqty{\lambda_0 z} \\
			\exp\pqty{\lambda_1 z}
		\end{pmatrix}
		,
		B_0=
		\begin{pmatrix}
			c_0 & 0 \\
			0 & c_1
		\end{pmatrix}
		,
		B_1=
		\begin{pmatrix}
			c_0 \lambda_0 & 0 \\
			0 & c_1 \lambda_1
		\end{pmatrix}
	\end{equation}
	
	Then
	\begin{equation}
		\begin{pmatrix}
			f\pqty{z_0+z_1}
		\end{pmatrix}
		=v_0\pqty{z_0}\transpose B_0 v_0\pqty{z_1}
		,
		\begin{pmatrix}
			f^\pqty{1}\pqty{z_0+z_1}
		\end{pmatrix}
		=v_0\pqty{z_0}\transpose B_1 v_0\pqty{z_1}
	\end{equation}
	
	These facts are a bit too obvious. We want \(\Bqty{f\pqty{z},f^\pqty{1}\pqty{z}}\) to be a basis of the set of solutions to get an awesome equation. Thus we require \(c_0\neq 0,c_1\neq 0\).
	
	Let
	\begin{equation}
		M=
		\begin{pmatrix}
			c_0 & c_1 \\
			c_0 \lambda_0 & c_1 \lambda_1
		\end{pmatrix}
		,
		v_1\pqty{z}=
		\begin{pmatrix}
			f\pqty{z} \\
			f^\pqty{1}\pqty{z}
		\end{pmatrix}
	\end{equation}
	
	Then
	\begin{equation}
		M v_0\pqty{z}=v_1\pqty{z}
		,
		M^{-1}=
		\frac{\begin{pmatrix}
			-c_1 \lambda_1 & c_1 \\
			c_0 \lambda_0 & -c_0
		\end{pmatrix}}{c_0 c_1 \pqty{\lambda_0-\lambda_1}}
		,
		v_0\pqty{z}=M^{-1} v_1\pqty{z}
	\end{equation}
	
	Thus, by substitution,
	\begin{equation}
		\begin{pmatrix}
			f\pqty{z_0+z_1}
		\end{pmatrix}
		=\pqty{M^{-1} v_1\pqty{z_0}}\transpose B_0 M^{-1} v_1\pqty{z_1}
		,
		\begin{pmatrix}
			f^\pqty{1}\pqty{z_0+z_1}
		\end{pmatrix}
		=\pqty{M^{-1} v_1\pqty{z_0}}\transpose B_1 M^{-1} v_1\pqty{z_1}
	\end{equation}
	
	Simplify:
	\begin{equation}
		\begin{pmatrix}
			f\pqty{z_0+z_1}
		\end{pmatrix}
		=\frac{
		\begin{pmatrix}
			f\pqty{z_0} \\
			f^\pqty{1}\pqty{z_0}
		\end{pmatrix}\transpose
		\begin{pmatrix}
			c_0 \lambda_0^2+c_1 \lambda_1^2 & -c_0 \lambda_0-c_1 \lambda_1 \\
			-c_0 \lambda_0-c_1 \lambda_1 & c_0 +c_1
		\end{pmatrix}
		\begin{pmatrix}
			f\pqty{z_1} \\
			f^\pqty{1}\pqty{z_1}
		\end{pmatrix}}{c_0 c_1 \pqty{\lambda_0-\lambda_1}^2}
	\end{equation}
	\begin{equation}
		\begin{pmatrix}
			f^\pqty{1}\pqty{z_0+z_1}
		\end{pmatrix}
		=\frac{
		\begin{pmatrix}
			f\pqty{z_0} \\
			f^\pqty{1}\pqty{z_0}
		\end{pmatrix}\transpose
		\begin{pmatrix}
			\lambda_0 \lambda_1\pqty{c_0 \lambda_0+c_1 \lambda_1} & -\lambda_0 \lambda_1 \pqty{c_0+c_1} \\
			-\lambda_0 \lambda_1 \pqty{c_0+c_1} & c_0 \lambda_1+c_1\lambda_0
		\end{pmatrix}
		\begin{pmatrix}
			f\pqty{z_1} \\
			f^\pqty{1}\pqty{z_1}
		\end{pmatrix}}{c_0 c_1 \pqty{\lambda_0-\lambda_1}^2}
	\end{equation}
	
	\section{Case \(\lambda_0=\lambda_1=\lambda\)}
	\begin{equation}
		f\pqty{z}=c_0 \exp\pqty{\lambda z}+c_1 z \exp\pqty{\lambda z},
		c_0=f\pqty{0},
		c_1=f^\pqty{1}\pqty{0}-f\pqty{0}\lambda
	\end{equation}
	
	We can choose the basis of the set of solutions to be \(\Bqty{\exp\pqty{\lambda z}, z\exp\pqty{\lambda z}}\).
	Let
	\begin{equation}
		v_0\pqty{z}=
		\begin{pmatrix}
			\exp\pqty{\lambda z} \\
			z\exp\pqty{\lambda z}
		\end{pmatrix}
		,
		B_0=
		\begin{pmatrix}
			c_0 & c_1 \\
			c_1 & 0
		\end{pmatrix}
		,
		B_1=
		\begin{pmatrix}
			c_1+c_0\lambda & c_1 \lambda \\
			c_1 \lambda & 0
		\end{pmatrix}
	\end{equation}
	
	Then
	\begin{equation}
		\begin{pmatrix}
			f\pqty{z_0+z_1}
		\end{pmatrix}
		=v_0\pqty{z_0}\transpose B_0 v_0\pqty{z_1}
		,
		\begin{pmatrix}
			f^\pqty{1}\pqty{z_0+z_1}
		\end{pmatrix}
		=v_0\pqty{z_0}\transpose B_1 v_0\pqty{z_1}
	\end{equation}
	
	These facts are a bit too obvious. We want \(\Bqty{f\pqty{z},f^\pqty{1}\pqty{z}}\) to be a basis of the set of solutions to get an awesome equation. Thus we require \(c_1\neq 0\).
	
	Let
	\begin{equation}
		M=
		\begin{pmatrix}
			c_0 & c_1 \\
			c_1+c_0 \lambda & c_1 \lambda
		\end{pmatrix}
		,
		v_1\pqty{z}=
		\begin{pmatrix}
			f\pqty{z} \\
			f^\pqty{1}\pqty{z}
		\end{pmatrix}
	\end{equation}
	
	Then
	\begin{equation}
		M v_0\pqty{z}=v_1\pqty{z}
		,
		M^{-1}=
		\frac{\begin{pmatrix}
			-c_1 \lambda & c_1 \\
			c_1+c_0\lambda & -c_0
		\end{pmatrix}}{c_1^2}
		,
		v_0\pqty{z}=M^{-1} v_1\pqty{z}
	\end{equation}
	
	Thus, by substitution,
	\begin{equation}
		\begin{pmatrix}
			f\pqty{z_0+z_1}
		\end{pmatrix}
		=\pqty{M^{-1} v_1\pqty{z_0}}\transpose B_0 M^{-1} v_1\pqty{z_1}
		,
		\begin{pmatrix}
			f^\pqty{1}\pqty{z_0+z_1}
		\end{pmatrix}
		=\pqty{M^{-1} v_1\pqty{z_0}}\transpose B_1 M^{-1} v_1\pqty{z_1}
	\end{equation}
	
	Simplify:
	\begin{equation}
		\begin{pmatrix}
			f\pqty{z_0+z_1}
		\end{pmatrix}
		=\frac{
		\begin{pmatrix}
			f\pqty{z_0} \\
			f^\pqty{1}\pqty{z_0}
		\end{pmatrix}\transpose
		\begin{pmatrix}
			-\lambda\pqty{2 c_1+c_0 \lambda} & c_1+c_0 \lambda \\
			c_1+c_0 \lambda & -c_0
		\end{pmatrix}
		\begin{pmatrix}
			f\pqty{z_1} \\
			f^\pqty{1}\pqty{z_1}
		\end{pmatrix}}{c_1^2}
	\end{equation}
	\begin{equation}
		\begin{pmatrix}
			f^\pqty{1}\pqty{z_0+z_1}
		\end{pmatrix}
		=\frac{
		\begin{pmatrix}
			f\pqty{z_0} \\
			f^\pqty{1}\pqty{z_0}
		\end{pmatrix}\transpose
		\begin{pmatrix}
			-\lambda^2\pqty{c_1+c_0\lambda} & c_0 \lambda^2 \\
			c_0 \lambda^2 & c_1-c_0\lambda
		\end{pmatrix}
		\begin{pmatrix}
			f\pqty{z_1} \\
			f^\pqty{1}\pqty{z_1}
		\end{pmatrix}}{c_1^2}
	\end{equation}
	
\end{document}
