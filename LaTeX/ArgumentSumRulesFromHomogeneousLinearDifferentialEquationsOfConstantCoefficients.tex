%!TEX engine = lualatex
\documentclass{article}

% --- CORE MATH & LOGIC ---
\usepackage{amsmath}
\usepackage{amsthm}
\usepackage{amssymb}
\usepackage{amsfonts}
\usepackage{physics}
\usepackage{mathtools}

% --- TYPOGRAPHY & LAYOUT ---
\usepackage[a4paper,margin=0.0931547in]{geometry}
\usepackage{fontspec}
\usepackage{microtype}
\usepackage[skip=0.5\baselineskip]{parskip}

% --- COLORS ---
\usepackage{xcolor}
% Define Palette
\definecolor{FavoriteColor1}{HTML}{607CB2} % Background (Medium Blue)
\definecolor{FavoriteColor2}{HTML}{303E59} % Text/Frame (Dark Blue)
\definecolor{LinkColor}{HTML}{FFFF00}	  % Links (Bright Yellow)

% Apply Background to Page
\pagecolor{FavoriteColor1}

% Apply Default Text Color (Global)
\makeatletter
\newcommand{\globalcolor}{\color{FavoriteColor2}\global\let\default@color\current@color}
\makeatother
\AtBeginDocument{\globalcolor}

% --- THEOREM STYLING ---
\usepackage[many]{tcolorbox}

% Define the box style
\tcbset{
	nazgandbox/.style={
		enhanced,
		% Background matches page color exactly (looks transparent)
		colback=FavoriteColor1,
		% Frame matches text color
		colframe=FavoriteColor2,
		% Text inside box matches global text color
		coltext=FavoriteColor2,
		% Title text matches background (high contrast against dark frame)
		coltitle=FavoriteColor1,
		fonttitle=\bfseries\sffamily,
		boxrule=1pt,
		arc=2pt,
		left=5pt, right=5pt, top=5pt, bottom=5pt,
		boxsep=1pt,
		sharp corners=downhill,
		breakable
	}
}

% Standard Definitions
\theoremstyle{definition}
\newtheorem{theorem}{Theorem}[section]
\newtheorem{lemma}[theorem]{Lemma}
\newtheorem{definition}[theorem]{Definition}
\newtheorem{corollary}[theorem]{Corollary}

% Wrap standard theorems in the custom box
\tcolorboxenvironment{theorem}{nazgandbox}
\tcolorboxenvironment{lemma}{nazgandbox}
\tcolorboxenvironment{definition}{nazgandbox}
\tcolorboxenvironment{corollary}{nazgandbox}

% --- UTILITIES ---
\usepackage{datetime2}
\usepackage{minted}

% --- FONTS ---
\directlua{luaotfload.add_fallback("EmojiFallback",{"NotoColorEmoji:mode=harf;"})}
\setmainfont{Noto Serif}[RawFeature={fallback=EmojiFallback}, Ligatures=TeX]
\setsansfont{Noto Sans}[RawFeature={fallback=EmojiFallback}]
\setmonofont{Noto Sans Mono}[RawFeature={fallback=EmojiFallback}]

% --- REFERENCES & LINKS ---
\usepackage{xr}
\usepackage{hyperref}
\hypersetup{
	colorlinks=true,
	linkcolor=LinkColor, % Internal links (equations, sections)
	urlcolor=LinkColor,  % Web URLs
	citecolor=LinkColor, % Bibliography citations
	filecolor=LinkColor, % Links to local files
	pdfnewwindow=true
}
\usepackage{cleveref}

% --- CUSTOM MACROS ---
\pdfstringdefDisableCommands{\def\eqref#1{(\ref{#1})}}
\newcommand{\inlineeqnum}{\refstepcounter{equation}~~\mbox{(\theequation)}}
\numberwithin{equation}{section}

% Operators
\DeclareMathOperator{\arsinh}{arsinh}
\DeclareMathOperator{\arcosh}{arcosh}
\DeclareMathOperator{\artanh}{artanh}
\DeclareMathOperator{\rues}{Rues}
\DeclareMathOperator{\md}{Mod}
\DeclareMathOperator{\pow}{Pow}

% Helpers
\newcommand{\floor}[1]{{\left\lfloor#1\right\rfloor}}
\newcommand{\ceil}[1]{{\left\lceil#1\right\rceil}}
\newcommand{\transpose}{^\intercal}
\newcommand{\laplace}[1]{\mathcal{L}\Bqty{#1}\pqty{s}}
\newcommand{\laplaceInv}[1]{\mathcal{L}^{-1}\Bqty{#1}\pqty{t}}
\newcommand{\pdiff}[2]{\frac{\partial^{#2}}{\partial #1^{#2}}}
\newcommand{\dint}[4]{\int_{#1}^{#2}#3\,\mathrm{d}#4}
\newcommand{\ketten}[4]{\underset{#1}{\overset{#2}{\mathop{\vcenter{\hbox{\huge\(\mathrm{K}\)}}}}}\frac{#3}{#4}}
\newcommand{\replace}[2]{\Big\vert_{#1\to{#2}}}

% --- METADATA & TITLE ---
\author{Mark Andrew Gerads \(<\)\href{MailTo:Nazgand@Gmail.Com}{Nazgand@Gmail.Com}\(>\)}
\let\oldauthor\author
\renewcommand{\author}[1]{
	\oldauthor{
		Author: #1
		\\
		Editor: Mark Andrew Gerads \(<\)\href{MailTo:Nazgand@Gmail.Com}{Nazgand@Gmail.Com}\(>\)
	}
}
\date{\DTMnow}
\let\oldtitle\title
\renewcommand{\title}[1]{
	\oldtitle{
		\vspace{-1.5cm}
		\url{https://GitHub.Com/Nazgand/NazgandMathBook}
		\\
		#1
	}
}

% --- LUA AUTO-REFERENCER ---
\usepackage{luacode}
\begin{luacode*}
	local lfs = require("lfs")

	local function split_path(path)
		local t = {}
		for part in path:gmatch("[^/]+") do table.insert(t, part) end
		return t
	end

	local function get_relative_path(base, target)
		local b_parts = split_path(base)
		local t_parts = split_path(target)
		local i = 1
		while b_parts[i] and t_parts[i] and b_parts[i] == t_parts[i] do
			i = i + 1
		end
		local rel = {}
		for j = i, #b_parts do table.insert(rel, "..") end
		for j = i, #t_parts do table.insert(rel, t_parts[j]) end
		return table.concat(rel, "/")
	end

	local handle = io.popen("git rev-parse --show-toplevel 2>/dev/null")
	local git_root = handle:read("*a"):gsub("%s+$", "")
	handle:close()

	if git_root ~= "" then
		local project_root = git_root .. "/LaTeX"
		local cwd = lfs.currentdir()
		
		local function scan_recursive(path)
			if not lfs.attributes(path) then return end
			for file in lfs.dir(path) do
				if file ~= "." and file ~= ".." then
					local full_path = path .. "/" .. file
					local f_attr = lfs.attributes(full_path)
					
					if f_attr and f_attr.mode == "directory" then
						scan_recursive(full_path)
					elseif file:match("%.tex$") then
						local job = file:gsub("%.tex$", "")
						if job ~= tex.jobname and job ~= "NazgandStyle" then
							local target_base = full_path:gsub("%.tex$", "")
							local rel_path = get_relative_path(cwd, target_base)
							
							tex.sprint("\\externaldocument{" .. rel_path .. "}[" .. rel_path .. ".pdf]")
						end
					end
				end
			end
		end

		texio.write_nl("--- [XR] Auto-Linking ---")
		scan_recursive(project_root)
	end
\end{luacode*}
\title{
	Argument Sum Rules From
	
	Homogeneous Linear Differential Equations Of Constant Coefficients
}
\begin{document}
	
	\maketitle
	
	\begin{abstract}
		The goal of this paper is to describe a theorem (previously a conjecture).
	\end{abstract}
	
	\section{Assumptions and definitions}
	A [homogeneous linear differential equation of constant coefficients] has the form
	\begin{equation}
		0=\sum_{k=0}^{n}a_k\pdiff{z}{k}f\pqty{z}
		=\sum_{k=0}^{n}a_k f^\pqty{k}\pqty{z}
	\end{equation}
	where \(\forall k, a_k\in\mathbb{C}\inlineeqnum\), \(n\in\mathbb{Z}_{\geq 0}\inlineeqnum\), \(a_n \neq 0\inlineeqnum\). Let \(f : \mathbb{C}\to\mathbb{C}\inlineeqnum\) be differentiable at least \(n\) times everywhere.
	
	Every [homogeneous linear differential equation of constant coefficients] has at least 1 basis for the vector space of its solutions.
	
	Let \(g_0\pqty{z},\dots,g_{n-1}\pqty{z}\) be such a basis. To be clear:
	\begin{equation}
		\label{BasisExistsForSolutionSet}
		\text{Solutions}=
		\Bqty{h\pqty{z} \mid 0=\sum_{k=0}^{n}a_k h^\pqty{k}\pqty{z}}
		=\Bqty{\sum_{k=0}^{n-1}b_k g_k\pqty{z} \mid b_k\in\mathbb{C}}
	\end{equation}
	Let us define a column vector and clarify its transpose (a row vector):
	\begin{equation}
		v\pqty{z_0}=
		\begin{pmatrix}
			g_0\pqty{z_0} \\
			\vdots \\
			g_{n-1}\pqty{z_0} \\
		\end{pmatrix}
		,
		v\pqty{z_1}\transpose=
		\begin{pmatrix}
			g_0\pqty{z_1} &
			\dots &
			g_{n-1}\pqty{z_1}
		\end{pmatrix}
	\end{equation}
	
	\section{Main theorem with 2 arguments}
	There exists a unique symmetric \(n\) by \(n\) matrix \(A\) (\(A=A\transpose\)) such that
	\begin{equation}
		\label{ArgumentSumRuleMatrix}
		f\pqty{z_0+z_1}=v\pqty{z_1}\transpose A v\pqty{z_0}=v\pqty{z_0}\transpose A v\pqty{z_1}
	\end{equation}
	
	\subsection{A reason \(A\) is symmetric}
	Suppose instead of a symmetric matrix \(A\), we find a matrix \(B\) such that
	\begin{equation}
		f\pqty{z_0+z_1}=v\pqty{z_1}\transpose B v\pqty{z_0}
	\end{equation}
	Then take the transpose of the equation and substitute \(z_0\to z_1,z_1\to z_0\), resulting in the following equation:
	\begin{equation}
		f\pqty{z_0+z_1}=v\pqty{z_1}\transpose B\transpose v\pqty{z_0}
	\end{equation}
	Average both equations:
	\begin{equation}
		f\pqty{z_0+z_1}=v\pqty{z_1}\transpose \frac{B+B\transpose}{2} v\pqty{z_0}
	\end{equation}
	Note that we can set \(A=\frac{B+B\transpose}{2}\) because it is symmetric. \qed
	
	\section{Theorem generalized to positive integer amounts of arguments in addition to 2}
	The following statements use the same assumptions and definitions as the main theorem.
	\subsection{1 argument}
	The following statement, which I call \(\text{ArgSumCon}\pqty{1}\), is trivial by assumption the \(\eqref{BasisExistsForSolutionSet}\).
	\begin{equation}
		\exists! c\pqty{k_0}\in\mathbb{C}, f\pqty{z_0} = \sum_{k_0=0}^{n-1}c\pqty{k_0} g_{k_0}\pqty{z_0}
	\end{equation}
	
	\subsection{2 arguments}
	The following statement, which I call \(\text{ArgSumCon}\pqty{2}\), is equivalent to \(\eqref{ArgumentSumRuleMatrix}\).
	\begin{equation}
		\exists! c\pqty{k_0,k_1}\in\mathbb{C}, f\pqty{z_0+z_1} = \sum_{k_0=0}^{n-1}\sum_{k_1=0}^{n-1}c\pqty{k_0,k_1} g_{k_0}\pqty{z_0} g_{k_1}\pqty{z_1}
	\end{equation}
	
	\subsection{3 arguments}
	The following statement, I call \(\text{ArgSumCon}\pqty{3}\).
	\begin{equation}
		\exists! c\pqty{k_0,k_1,k_2}\in\mathbb{C}, f\pqty{z_0+z_1+z_2} =
		\sum_{k_0=0}^{n-1}\sum_{k_1=0}^{n-1}\sum_{k_2=0}^{n-1}c\pqty{k_0,k_1,k_2} g_{k_0}\pqty{z_0} g_{k_1}\pqty{z_1} g_{k_2}\pqty{z_2}
	\end{equation}
	
	\subsection{\(m\) arguments for \(m\in\mathbb{Z}_{>0}\)}
	The following statement, I call \(\text{ArgSumCon}\pqty{m}\).
	\begin{equation}
		\exists! c\pqty{k_0,k_1,\dots,k_{m-1}}\in\mathbb{C}, f\pqty{\sum_{j=0}^{m-1}z_j} =
		\sum_{k_0=0}^{n-1}\sum_{k_1=0}^{n-1}\dots\sum_{k_{m-1}=0}^{n-1}c\pqty{k_0,k_1,\dots,k_{m-1}} \prod_{j=0}^{m-1}g_{k_j}\pqty{z_j}
	\end{equation}
	Note \(\text{ArgSumCon}\pqty{m+1}\Rightarrow\text{ArgSumCon}\pqty{m}\), as can be seen by replacing 1 of the variables with 0.
	
	\section{Lean 4 code}
	\(\text{ArgSumCon}\pqty{m}\) has been formally proved in and verified by Lean 4:
	
	\href{https://GitHub.Com/Nazgand/NazgandLean4/blob/master/NazgandLean4/HomogeneousLinearDifferentialEquationsOfConstantCoefficients/ArgumentSumRule.lean}
	{https://GitHub.Com/Nazgand/NazgandLean4/blob/master/NazgandLean4/\\HomogeneousLinearDifferentialEquationsOfConstantCoefficients/ArgumentSumRule.lean}
	
	Also verified by Lean 4: the relevant tensor is unique and symmetric.
\end{document}
