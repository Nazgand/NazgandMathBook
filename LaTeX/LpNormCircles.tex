%!TEX engine = lualatex
\documentclass{article}

% --- CORE MATH & LOGIC ---
\usepackage{amsmath}
\usepackage{amsthm}
\usepackage{amssymb}
\usepackage{amsfonts}
\usepackage{physics}
\usepackage{mathtools}

% --- TYPOGRAPHY & LAYOUT ---
\usepackage[a4paper,margin=0.0931547in]{geometry}
\usepackage{fontspec}
\usepackage{microtype}
\usepackage[skip=0.5\baselineskip]{parskip}

% --- COLORS ---
\usepackage{xcolor}
% Define Palette
\definecolor{FavoriteColor1}{HTML}{607CB2} % Background (Medium Blue)
\definecolor{FavoriteColor2}{HTML}{303E59} % Text/Frame (Dark Blue)
\definecolor{LinkColor}{HTML}{FFFF00}	  % Links (Bright Yellow)

% Apply Background to Page
\pagecolor{FavoriteColor1}

% Apply Default Text Color (Global)
\makeatletter
\newcommand{\globalcolor}{\color{FavoriteColor2}\global\let\default@color\current@color}
\makeatother
\AtBeginDocument{\globalcolor}

% --- THEOREM STYLING ---
\usepackage[many]{tcolorbox}

% Define the box style
\tcbset{
	nazgandbox/.style={
		enhanced,
		% Background matches page color exactly (looks transparent)
		colback=FavoriteColor1,
		% Frame matches text color
		colframe=FavoriteColor2,
		% Text inside box matches global text color
		coltext=FavoriteColor2,
		% Title text matches background (high contrast against dark frame)
		coltitle=FavoriteColor1,
		fonttitle=\bfseries\sffamily,
		boxrule=1pt,
		arc=2pt,
		left=5pt, right=5pt, top=5pt, bottom=5pt,
		boxsep=1pt,
		sharp corners=downhill,
		breakable
	}
}

% Standard Definitions
\theoremstyle{definition}
\newtheorem{theorem}{Theorem}[section]
\newtheorem{lemma}[theorem]{Lemma}
\newtheorem{definition}[theorem]{Definition}
\newtheorem{corollary}[theorem]{Corollary}

% Wrap standard theorems in the custom box
\tcolorboxenvironment{theorem}{nazgandbox}
\tcolorboxenvironment{lemma}{nazgandbox}
\tcolorboxenvironment{definition}{nazgandbox}
\tcolorboxenvironment{corollary}{nazgandbox}

% --- UTILITIES ---
\usepackage{datetime2}
\usepackage{minted}

% --- FONTS ---
\directlua{luaotfload.add_fallback("EmojiFallback",{"NotoColorEmoji:mode=harf;"})}
\setmainfont{Noto Serif}[RawFeature={fallback=EmojiFallback}, Ligatures=TeX]
\setsansfont{Noto Sans}[RawFeature={fallback=EmojiFallback}]
\setmonofont{Noto Sans Mono}[RawFeature={fallback=EmojiFallback}]

% --- REFERENCES & LINKS ---
\usepackage{xr}
\usepackage{hyperref}
\hypersetup{
	colorlinks=true,
	linkcolor=LinkColor, % Internal links (equations, sections)
	urlcolor=LinkColor,  % Web URLs
	citecolor=LinkColor, % Bibliography citations
	filecolor=LinkColor, % Links to local files
	pdfnewwindow=true
}
\usepackage{cleveref}

% --- CUSTOM MACROS ---
\pdfstringdefDisableCommands{\def\eqref#1{(\ref{#1})}}
\newcommand{\inlineeqnum}{\refstepcounter{equation}~~\mbox{(\theequation)}}
\numberwithin{equation}{section}

% Operators
\DeclareMathOperator{\arsinh}{arsinh}
\DeclareMathOperator{\arcosh}{arcosh}
\DeclareMathOperator{\artanh}{artanh}
\DeclareMathOperator{\rues}{Rues}
\DeclareMathOperator{\md}{Mod}
\DeclareMathOperator{\pow}{Pow}

% Helpers
\newcommand{\floor}[1]{{\left\lfloor#1\right\rfloor}}
\newcommand{\ceil}[1]{{\left\lceil#1\right\rceil}}
\newcommand{\transpose}{^\intercal}
\newcommand{\laplace}[1]{\mathcal{L}\Bqty{#1}\pqty{s}}
\newcommand{\laplaceInv}[1]{\mathcal{L}^{-1}\Bqty{#1}\pqty{t}}
\newcommand{\pdiff}[2]{\frac{\partial^{#2}}{\partial #1^{#2}}}
\newcommand{\dint}[4]{\int_{#1}^{#2}#3\,\mathrm{d}#4}
\newcommand{\ketten}[4]{\underset{#1}{\overset{#2}{\mathop{\vcenter{\hbox{\huge\(\mathrm{K}\)}}}}}\frac{#3}{#4}}
\newcommand{\replace}[2]{\Big\vert_{#1\to{#2}}}

% --- METADATA & TITLE ---
\author{Mark Andrew Gerads \(<\)\href{MailTo:Nazgand@Gmail.Com}{Nazgand@Gmail.Com}\(>\)}
\let\oldauthor\author
\renewcommand{\author}[1]{
	\oldauthor{
		Author: #1
		\\
		Editor: Mark Andrew Gerads \(<\)\href{MailTo:Nazgand@Gmail.Com}{Nazgand@Gmail.Com}\(>\)
	}
}
\date{\DTMnow}
\let\oldtitle\title
\renewcommand{\title}[1]{
	\oldtitle{
		\vspace{-1.5cm}
		\url{https://GitHub.Com/Nazgand/NazgandMathBook}
		\\
		#1
	}
}

% --- LUA AUTO-REFERENCER ---
\usepackage{luacode}
\begin{luacode*}
	local lfs = require("lfs")

	local function split_path(path)
		local t = {}
		for part in path:gmatch("[^/]+") do table.insert(t, part) end
		return t
	end

	local function get_relative_path(base, target)
		local b_parts = split_path(base)
		local t_parts = split_path(target)
		local i = 1
		while b_parts[i] and t_parts[i] and b_parts[i] == t_parts[i] do
			i = i + 1
		end
		local rel = {}
		for j = i, #b_parts do table.insert(rel, "..") end
		for j = i, #t_parts do table.insert(rel, t_parts[j]) end
		return table.concat(rel, "/")
	end

	local handle = io.popen("git rev-parse --show-toplevel 2>/dev/null")
	local git_root = handle:read("*a"):gsub("%s+$", "")
	handle:close()

	if git_root ~= "" then
		local project_root = git_root .. "/LaTeX"
		local cwd = lfs.currentdir()
		
		local function scan_recursive(path)
			if not lfs.attributes(path) then return end
			for file in lfs.dir(path) do
				if file ~= "." and file ~= ".." then
					local full_path = path .. "/" .. file
					local f_attr = lfs.attributes(full_path)
					
					if f_attr and f_attr.mode == "directory" then
						scan_recursive(full_path)
					elseif file:match("%.tex$") then
						local job = file:gsub("%.tex$", "")
						if job ~= tex.jobname and job ~= "NazgandStyle" then
							local target_base = full_path:gsub("%.tex$", "")
							local rel_path = get_relative_path(cwd, target_base)
							
							tex.sprint("\\externaldocument{" .. rel_path .. "}[" .. rel_path .. ".pdf]")
						end
					end
				end
			end
		end

		texio.write_nl("--- [XR] Auto-Linking ---")
		scan_recursive(project_root)
	end
\end{luacode*}
\DeclareMathOperator{\dcc}{dcc}
\title{
	\(L^p\) Planar Algebraic Geometry
}
\begin{document}

\maketitle

\section{\texorpdfstring{The \(L^p\) norm for \(p\in\mathbb{R},p\geq 1\)}{The L-p norm for real p, p≥1}}
The distance between points \(\pqty{x_1,y_1}\) and \(\pqty{x_1,y_1}\) is:
\begin{equation}
\sqrt[p]{\abs{x_1-x_2}^p+\abs{y_1-y_2}^p}
\end{equation}

This has the triangle inequality:
\begin{equation}
\sqrt[p]{\abs{x_1-x_2}^p+\abs{y_1-y_2}^p}+
\sqrt[p]{\abs{x_2-x_3}^p+\abs{y_2-y_3}^p}\geq
\sqrt[p]{\abs{x_1-x_3}^p+\abs{y_1-y_3}^p}
\end{equation}

Euclidean translations and scalings of objects preserves shape in \(L^p\) space. Euclidean rotations do not generally preserve shape.

\section{Lengths of curves}
A curve defined by a function \(y=f\pqty{x}\) between \(a\) and \(b\) has the length
\begin{equation}
\lim\limits_{\Delta x\to 0^+}\sum_{k=0}^{\floor{\pqty{b-a}/\Delta x}}
\sqrt[p]{\pqty{\Delta x}^p+\abs{f\pqty{a+\pqty{k+1}\Delta x}-f\pqty{a+k\Delta x}}^p}
\end{equation}
where the limit exists. This is equivalent to
\begin{equation}
\dint{a}{b}
{\sqrt[p]{1+\abs{f'\pqty{x}}^p}}{x}
\end{equation}

In polar coordinates,
\begin{equation}
\dint{a}{b}
{\sqrt[p]{\abs{\cos\pqty{\theta}r'\pqty{\theta}-\sin\pqty{\theta}r\pqty{\theta}}^p
	+\abs{\sin\pqty{\theta}r'\pqty{\theta}+\cos\pqty{\theta}r\pqty{\theta}}^p
}}{\theta}
\end{equation}
\begin{equation}
\label{CurveLengthPolar2}
\dint{a}{b}
{\abs{r\pqty{\theta}}
\sqrt[p]{\abs{\cos\pqty{\theta}\frac{r'\pqty{\theta}}{r\pqty{\theta}}-\sin\pqty{\theta}}^p
	+\abs{\sin\pqty{\theta}\frac{r'\pqty{\theta}}{r\pqty{\theta}}+\cos\pqty{\theta}}^p
}}{\theta}
\end{equation}

\section{Area}
If for all disjoint regions \(R_0,R_1\), if \(\text{Area}\pqty{R_0}+\text{Area}\pqty{R_1}=\text{Area}\pqty{R_0\cup R_1}\) and for all \(R_2\) which is a Euclidean translation of \(R_0\), \(\text{Area}\pqty{R_0}=\text{Area}\pqty{R_2}\), then Area needs to be proportional to Euclidean area. Thus Euclidean area is used.

\section{Circles}
In the \(L^p\) norm, a circle which is the set of points distance \(r_p\) from the point \(\pqty{x_0,y_0}\) is.
\begin{equation}
\abs{x-x_0}^p+\abs{y-y_0}^p=r_p^p
\end{equation}

The shape of circles is scale invariant:
\begin{equation}
\abs{\frac{x}{r_p}}^p+\abs{\frac{y}{r_p}}^p=1
\end{equation}

Thus this work concentrates the unit circle in the first quadrant \(x,y\in\mathbb{R}^+\), the other quadrants are reflections and not needing absolute value signs is nice during calculation. Thus what will be used is:
\begin{equation}
x^p+y^p=1
\end{equation}

An important point for symmetry: \(x=y=2^\pqty{-1/p}\).


Let \(\tau_p\) be the circumference of a unit circle in \(L^p\). \(\theta_p\pqty{\theta}\) is the function that translate the Euclidean angle to the \(L^p\) angle. 
\begin{equation}
\theta_p\pqty{\theta}=-\theta_p\pqty{-\theta}=\theta_p\pqty{\theta+\frac{\pi}{2}}-\frac{\tau_p}{4}
\end{equation}

In the first quadrant of the unit circle, 
\begin{equation}
1\geq x^p\Rightarrow y=\sqrt[p]{1-x^p}
\end{equation}

Area is important. Ratios between area and distance keep constant as do angles for scaling centered at the origin and rotations in \(L^p\) space. For \(p>0\),
\begin{equation}
\dint{0}{1}{\pqty{1-x^p}^{1/p}}{x}
=
\frac{\Gamma\pqty{1+\frac{1}{p}}^2}{\Gamma\pqty{1+\frac{2}{p}}}
\end{equation}

and polar coordinates with respect to the Euclidean angle and Euclidean norm can be found via substituting \(x=r_p\cos\pqty{\theta},y=r_p\sin\pqty{\theta}\)
\begin{equation}
1=r_p^p\pqty{\abs{\cos\pqty{\theta}}^p+\abs{\sin\pqty{\theta}}^p}
\land
r_p=\pqty{\abs{\cos\pqty{\theta}}^p+\abs{\sin\pqty{\theta}}^p}^{-1/p}
\end{equation}

In the first quadrant:
\begin{equation}
0\leq \theta \leq \frac{\pi}{2}
\Rightarrow
r_p\pqty{\theta}=\pqty{{\cos\pqty{\theta}}^p+{\sin\pqty{\theta}}^p}^{-1/p}
\end{equation}
\begin{equation}
\frac{r'\pqty{\theta}}{r\pqty{\theta}}=\frac{{\tan\pqty{\theta}\cos\pqty{\theta}^p-\cot\pqty{\theta}\sin\pqty{\theta}^p}}
{\cos\pqty{\theta}^p+\sin\pqty{\theta}^p}
\end{equation}

An important function is a scaling function based on the Euclidean angle.
\begin{equation}
s_p\pqty{\theta}=\pqty{\abs{\cos\pqty{\theta}}^p+\abs{\sin\pqty{\theta}}^p}^{1/p}
\end{equation}

In the \(L^p\) norm, a formula for arc length is
\begin{equation}
\dint{a}{b}{s_p\pqty{\arctan\pqty{\frac{\partial y}{\partial x}}} \sqrt{1+\pqty{\frac{\partial y}{\partial x}}^2}}{x}
\end{equation}

In the \(L^p\) norm, a formula for arc length in Euclidean polar coordinates is
\begin{equation}
\dint{a}{b}
{s_p\pqty{\arctan\pqty{\frac{\sin\pqty{\theta}\frac{\partial r}{\partial \theta}+r\cos\pqty{\theta}}{\cos\pqty{\theta}\frac{\partial r}{\partial \theta}-r\sin\pqty{\theta}}}}
\sqrt{r^2+\pqty{\frac{\partial r}{\partial \theta}}^2}}{\theta}
\end{equation}

The circumference can be calculated:
\begin{equation}
\frac{\partial y}{\partial x}=\pdiff{x}{}\sqrt[p]{1-x^p}
=\frac{1}{p}\pqty{1-x^p}^{\frac{1}{p}-1}*-px^{p-1}
=-\pqty{1-x^p}^{\frac{1}{p}-1}x^{p-1}
\end{equation}
\begin{equation}
\tau_p=4\dint{0}{1}{\sqrt[p]{1+\abs{\pqty{1-x^p}^{\frac{1}{p}-1}x^{p-1}}^p}}{x}
=
4\dint{0}{1}{\sqrt[p]{1+{\pqty{1-x^p}^{1-p}x^{p^2-p}}}}{x}
\end{equation}

\begin{equation}
\tau_p=
4\dint{0}{1}{\frac{\sqrt[p]{\pqty{1-x^p}\pqty{\pqty{1-x^p}^{p-1}+x^{p^2-p}}}}{1-x^p}}{x}
=
\frac{4}{p}\dint{0}{1}{\frac{\sqrt[p]{z\pqty{1-z}\pqty{\pqty{1-z}^{p-1}+z^{p-1}}}}{\pqty{1-z}z}}{z}
\end{equation}
\begin{equation}
\tau_p=
\frac{8}{p}\dint{0}{1/2}{\frac{\sqrt[p]{z\pqty{1-z}\pqty{\pqty{1-z}^{p-1}+z^{p-1}}}}{\pqty{1-z}z}}{z}
\end{equation}

Case \(p=3\): Choose \(\pqty{1-z}^2+z^2=y\), \(\pdiff{z}{}y=4z-2\), \(z\pqty{1-z}=\frac{1-y}{2}\)
\begin{equation}
\tau_3=
\frac{16\sqrt[3]{4}}{3}
\dint{1/2}{1}{y \pqty{y\pqty{1-y}}^{-2/3} \sqrt{2y-1}}{y}
\end{equation}

Choose \(z=\frac{1}{1+e^{-x}}\), \(\frac{\partial x}{\partial z}=\frac{1}{z\pqty{1-z}}\)
\begin{equation}
\tau_p=
\frac{8}{p}\dint{0}{\infty}{\sqrt[p]{ \frac{e^{x}+e^{p {x}}}{\pqty{1+e^{x}}^{p+1}} }}{x}
\end{equation}

Choose \(z=e^{x}\), \(\frac{\partial x}{\partial z}=\frac{1}{z}\)
\begin{equation}
\tau_p=
\frac{8}{p}\dint{1}{\infty}{\frac{1}{z\pqty{1+z}}\sqrt[p]{ \frac{z+z^{p}}{1+z} }}{z}
\end{equation}

Choose \(x=1+z^{p-1}\), \(\pqty{x-1}^{1/\pqty{p-1}}=z\), \(\frac{\partial x}{\partial z}=\frac{1}{z}\)
\begin{equation}
\tau_p=
\frac{8}{p}\dint{1}{\infty}{\frac{1}{z\pqty{1+z}}\sqrt[p]{ \frac{z+z^{p}}{1+z} }}{z}
\end{equation}

Some exact values:
\begin{equation}
\tau_3=\frac{2}{3} \Gamma \pqty{\frac{1}{3}} \pqty{\frac{\Gamma \pqty{\frac{1}{4}}}{\Gamma \pqty{\frac{7}{12}}}+\frac{\Gamma \pqty{\frac{3}{4}}}{\Gamma \pqty{\frac{13}{12}}}}, \tau_2=2\pi,\tau_1=8
\end{equation}


\section{Regular polygons and circles around circles}
Given a point at Euclidean angle \(\theta\) a function is desired which gives the angle for the point distance \(0\leq d\leq 2\) counter-clockwise from the point at angle \(\theta\), and nesting of this function is desired to construct regular polygons.
\begin{equation}
\theta\leq\dcc\pqty{\theta,d,p,1}\leq\theta+\pi
\end{equation}
\begin{align}
\pqty{\Delta x}= &&
r_p\pqty{\dcc\pqty{\theta,d,p,1}}\cos\pqty{\dcc\pqty{\theta,d,p,1}}-r_p\pqty{\theta}\cos\pqty{\theta}\\
\pqty{\Delta y}= &&
r_p\pqty{\dcc\pqty{\theta,d,p,1}}\sin\pqty{\dcc\pqty{\theta,d,p,1}}-r_p\pqty{\theta}\sin\pqty{\theta}\\
d= &&
\sqrt[p]{\pqty{\Delta x}^p+\pqty{\Delta y}^p}\\
\dcc\pqty{\theta,d,p,0}
&& =\theta
\\
\dcc\pqty{\theta,d,p,n}
&& =\theta -\md\pqty{\theta,\frac{\pi}{2}}+\dcc\pqty{\md\pqty{\theta,\frac{\pi}{2}},d,p,n}
\\
&& =\dcc\pqty{\dcc\pqty{\theta,d,p,n-m},d,p,m}
\\
&& =-\dcc\pqty{-\theta,d,p,-n}
\end{align}

Interestingly, regular hexagons have a constant side regardless of angle and \(p\).
\begin{align}
\dcc\pqty{\theta,1,p,6}
&& =\theta + 2\pi
\\
\dcc\pqty{\theta,2,p,2}
&& =\theta + 2\pi
\end{align}


\section{\texorpdfstring{Parabola for \(p>1\)}{Parabola for p>1}}
Each point on a parabola has the same distance from the focus as from the directrix. With a directrix of \(y=0\) and a focus at \(\pqty{0,2a}, a\in\mathbb{R}^+\), we have \(f\pqty{x}=f\pqty{-x},f\pqty{0}=a,f\pqty{2a}=2a\). Also,
\begin{equation}
\abs{x}^p+\abs{f\pqty{x}-2a}^p=\abs{f\pqty{x}}^p
\end{equation}

\(x\geq 0\) WLOG. 2 cases exist. \(f\pqty{x}\geq 2a\):
\begin{equation}
x^p+\pqty{f\pqty{x}-2a}^p={f\pqty{x}}^p
\end{equation}

and \(f\pqty{x}\leq 2a\):
\begin{equation}
x^p+\pqty{2a-f\pqty{x}}^p={f\pqty{x}}^p
\end{equation}

Both cases are equivalent when \(\exists k\in\mathbb{Z}, 2k = p\).

\end{document}
