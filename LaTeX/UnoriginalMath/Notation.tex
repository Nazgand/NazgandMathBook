%!TEX engine = lualatex
\documentclass{article}

% --- CORE MATH & LOGIC ---
\usepackage{amsmath}
\usepackage{amsthm}
\usepackage{amssymb}
\usepackage{amsfonts}
\usepackage{physics}
\usepackage{mathtools}

% --- TYPOGRAPHY & LAYOUT ---
\usepackage[a4paper,margin=0.0931547in]{geometry}
\usepackage{fontspec}
\usepackage{microtype}
\usepackage[skip=0.5\baselineskip]{parskip}

% --- COLORS ---
\usepackage{xcolor}
% Define Palette
\definecolor{FavoriteColor1}{HTML}{607CB2} % Background (Medium Blue)
\definecolor{FavoriteColor2}{HTML}{303E59} % Text/Frame (Dark Blue)
\definecolor{LinkColor}{HTML}{FFFF00}	  % Links (Bright Yellow)

% Apply Background to Page
\pagecolor{FavoriteColor1}

% Apply Default Text Color (Global)
\makeatletter
\newcommand{\globalcolor}{\color{FavoriteColor2}\global\let\default@color\current@color}
\makeatother
\AtBeginDocument{\globalcolor}

% --- THEOREM STYLING ---
\usepackage[many]{tcolorbox}

% Define the box style
\tcbset{
	nazgandbox/.style={
		enhanced,
		% Background matches page color exactly (looks transparent)
		colback=FavoriteColor1,
		% Frame matches text color
		colframe=FavoriteColor2,
		% Text inside box matches global text color
		coltext=FavoriteColor2,
		% Title text matches background (high contrast against dark frame)
		coltitle=FavoriteColor1,
		fonttitle=\bfseries\sffamily,
		boxrule=1pt,
		arc=2pt,
		left=5pt, right=5pt, top=5pt, bottom=5pt,
		boxsep=1pt,
		sharp corners=downhill,
		breakable
	}
}

% Standard Definitions
\theoremstyle{definition}
\newtheorem{theorem}{Theorem}[section]
\newtheorem{lemma}[theorem]{Lemma}
\newtheorem{definition}[theorem]{Definition}
\newtheorem{corollary}[theorem]{Corollary}

% Wrap standard theorems in the custom box
\tcolorboxenvironment{theorem}{nazgandbox}
\tcolorboxenvironment{lemma}{nazgandbox}
\tcolorboxenvironment{definition}{nazgandbox}
\tcolorboxenvironment{corollary}{nazgandbox}

% --- UTILITIES ---
\usepackage{datetime2}
\usepackage{minted}

% --- FONTS ---
\directlua{luaotfload.add_fallback("EmojiFallback",{"NotoColorEmoji:mode=harf;"})}
\setmainfont{Noto Serif}[RawFeature={fallback=EmojiFallback}, Ligatures=TeX]
\setsansfont{Noto Sans}[RawFeature={fallback=EmojiFallback}]
\setmonofont{Noto Sans Mono}[RawFeature={fallback=EmojiFallback}]

% --- REFERENCES & LINKS ---
\usepackage{xr}
\usepackage{hyperref}
\hypersetup{
	colorlinks=true,
	linkcolor=LinkColor, % Internal links (equations, sections)
	urlcolor=LinkColor,  % Web URLs
	citecolor=LinkColor, % Bibliography citations
	filecolor=LinkColor, % Links to local files
	pdfnewwindow=true
}
\usepackage{cleveref}

% --- CUSTOM MACROS ---
\pdfstringdefDisableCommands{\def\eqref#1{(\ref{#1})}}
\newcommand{\inlineeqnum}{\refstepcounter{equation}~~\mbox{(\theequation)}}
\numberwithin{equation}{section}

% Operators
\DeclareMathOperator{\arsinh}{arsinh}
\DeclareMathOperator{\arcosh}{arcosh}
\DeclareMathOperator{\artanh}{artanh}
\DeclareMathOperator{\rues}{Rues}
\DeclareMathOperator{\md}{Mod}
\DeclareMathOperator{\pow}{Pow}

% Helpers
\newcommand{\floor}[1]{{\left\lfloor#1\right\rfloor}}
\newcommand{\ceil}[1]{{\left\lceil#1\right\rceil}}
\newcommand{\transpose}{^\intercal}
\newcommand{\laplace}[1]{\mathcal{L}\Bqty{#1}\pqty{s}}
\newcommand{\laplaceInv}[1]{\mathcal{L}^{-1}\Bqty{#1}\pqty{t}}
\newcommand{\pdiff}[2]{\frac{\partial^{#2}}{\partial #1^{#2}}}
\newcommand{\dint}[4]{\int_{#1}^{#2}#3\,\mathrm{d}#4}
\newcommand{\ketten}[4]{\underset{#1}{\overset{#2}{\mathop{\vcenter{\hbox{\huge\(\mathrm{K}\)}}}}}\frac{#3}{#4}}
\newcommand{\replace}[2]{\Big\vert_{#1\to{#2}}}

% --- METADATA & TITLE ---
\author{Mark Andrew Gerads \(<\)\href{MailTo:Nazgand@Gmail.Com}{Nazgand@Gmail.Com}\(>\)}
\let\oldauthor\author
\renewcommand{\author}[1]{
	\oldauthor{
		Author: #1
		\\
		Editor: Mark Andrew Gerads \(<\)\href{MailTo:Nazgand@Gmail.Com}{Nazgand@Gmail.Com}\(>\)
	}
}
\date{\DTMnow}
\let\oldtitle\title
\renewcommand{\title}[1]{
	\oldtitle{
		\vspace{-1.5cm}
		\url{https://GitHub.Com/Nazgand/NazgandMathBook}
		\\
		#1
	}
}

% --- LUA AUTO-REFERENCER ---
\usepackage{luacode}
\begin{luacode*}
	local lfs = require("lfs")

	local function split_path(path)
		local t = {}
		for part in path:gmatch("[^/]+") do table.insert(t, part) end
		return t
	end

	local function get_relative_path(base, target)
		local b_parts = split_path(base)
		local t_parts = split_path(target)
		local i = 1
		while b_parts[i] and t_parts[i] and b_parts[i] == t_parts[i] do
			i = i + 1
		end
		local rel = {}
		for j = i, #b_parts do table.insert(rel, "..") end
		for j = i, #t_parts do table.insert(rel, t_parts[j]) end
		return table.concat(rel, "/")
	end

	local handle = io.popen("git rev-parse --show-toplevel 2>/dev/null")
	local git_root = handle:read("*a"):gsub("%s+$", "")
	handle:close()

	if git_root ~= "" then
		local project_root = git_root .. "/LaTeX"
		local cwd = lfs.currentdir()
		
		local function scan_recursive(path)
			if not lfs.attributes(path) then return end
			for file in lfs.dir(path) do
				if file ~= "." and file ~= ".." then
					local full_path = path .. "/" .. file
					local f_attr = lfs.attributes(full_path)
					
					if f_attr and f_attr.mode == "directory" then
						scan_recursive(full_path)
					elseif file:match("%.tex$") then
						local job = file:gsub("%.tex$", "")
						if job ~= tex.jobname and job ~= "NazgandStyle" then
							local target_base = full_path:gsub("%.tex$", "")
							local rel_path = get_relative_path(cwd, target_base)
							
							tex.sprint("\\externaldocument{" .. rel_path .. "}[" .. rel_path .. ".pdf]")
						end
					end
				end
			end
		end

		texio.write_nl("--- [XR] Auto-Linking ---")
		scan_recursive(project_root)
	end
\end{luacode*}
\title{
	Notation
}
\begin{document}
	
	\maketitle
	
	\section{Substitution}
	This works similarly to Mathematica's Replace function. Example:
	\begin{equation}
	\pdiff{3}{} 3^2 \ne \pdiff{x}{} x^2 \replace{x}{3} = 2x \replace{x}{3} = 6
	\end{equation}
	
	\url{https://reference.wolfram.com/language/ref/Replace.html}
	
	\section{Logic}
	
	And: Given statements \(A\) and \(B\). The statement \(A\land B\) means that both statements \(A\) and \(B\) are true.
	
	Or: Given statements \(A\) and \(B\). The statement \(A\lor B\) means that at least 1 of the statements \(A\) or \(B\) is true.
	
	Implies: The statement \(A\Rightarrow B\) is a conditional statement stating that if A is true, then B is true. 
	
	\(A\Leftrightarrow B\) means \(\bqty{\bqty{A\Rightarrow B}\land \bqty{B\Rightarrow A}}\).
	
	Brackets: Normally, logic is read from left to right, yet sometimes brackets are used to change the order or add clarification. Example: \(A\lor\bqty{B\land C}\).
	
	Exists: \(\exists a\) means that some a exists.
	\(\bqty{\exists a, P\pqty{a}} \iff \Bqty{}\neq\{a\mid P\pqty{a}\}\).
	
	\url{https://en.wikipedia.org/wiki/List_of_logic_symbols}
	
	Kronecker delta function:
	\begin{equation}
	\delta\pqty{0}=1
	\land
	\bqty{x\neq 0 \Rightarrow \delta\pqty{x}=0}
	\end{equation}
	\url{https://en.wikipedia.org/wiki/Kronecker_delta}
	
	\section{Order and Equality}
	Equals: \(a=b\) means that \(a\) is equal to \(b\).
	
	\url{https://en.wikipedia.org/wiki/Equality_(mathematics)}
	
	Greater than: \(a>b\) means that \(a\) is greater than \(b\).
	\(a\geq b\) means that \(a\) is greater than or equal to \(b\).
	
	Less than: \(a<b\) means that \(a\) is less than \(b\).
	\(a\leq b\) means that \(a\) is less than or equal to \(b\).
	
	\url{https://en.wikipedia.org/wiki/Inequality_(mathematics)}
	
	\section{Sets}
	Sets: Sets either have an element or they do not have the element; no element is listed more than 1 time. Example sets are: \(\Bqty{1,2,3}, \Bqty{1,\Bqty{1,a,b}}\).
	
	Element of: \(\in\) means "is an element of". Examples: \(1\in\Bqty{1,2,3}\) and \(\Bqty{1,a,b}\in\Bqty{1,\Bqty{1,a,b}}\).
	
	Subset: \(\subseteq\) means "is a subset of". Examples: \(\Bqty{1,2,3}\subseteq\Bqty{1,2,3}\) and \(\Bqty{3}\subseteq\Bqty{1,2,3}\) and \(\Bqty{}\subseteq\Bqty{1,2,3}\).
	
	Union: \(A\cup B\) is the minimal set which contains all elements either \(A\) or \(B\) contain.
	
	Intersection: \(A\cap B\) is the set which contains all elements both \(A\) and \(B\) contain.
	
	\url{https://en.wikipedia.org/wiki/Set_(mathematics)}
	
	Set builder notation:
	
	The set of all things \(a\) which satisfy the constraining statement \(P\pqty{a}\):
	\begin{equation}
	\Bqty{a\mid P\pqty{a}}
	\end{equation}
	
	The set of all things \(a\) in the set \(A\) which satisfy the statement \(P\pqty{a}\):
	\begin{equation}
	\Bqty{a\in A\mid P\pqty{a}}
	\end{equation}
	
	\url{https://en.wikipedia.org/wiki/Set-builder_notation}
	
	Integers:
	\begin{equation}
	0\in\mathbb{Z}\land\bqty{a\in\mathbb{Z}\Leftrightarrow \pqty{a+1}\in\mathbb{Z}}
	\end{equation}
	\begin{equation}
	\mathbb{Z}_{>0}=\Bqty{n\in\mathbb{Z}\mid n>0}
	\end{equation}
	\begin{equation}
	\mathbb{Z}_{\geq 0}=\Bqty{n\in\mathbb{Z}\mid n\geq 0}
	\end{equation}
	
	\url{https://en.wikipedia.org/wiki/Integer}
	
	\url{https://en.wikipedia.org/wiki/Natural_number#Notation}
	
	Rational numbers:
	\begin{equation}
	\mathbb{Q}=\Bqty{\frac{a}{b}\mid a\in\mathbb{Z}\land b\in\mathbb{Z}_{>0}}
	\end{equation}
	\begin{equation}
	\mathbb{Q}^+=\Bqty{q\in\mathbb{Q}\mid q>0}
	\end{equation}
	\begin{equation}
	\mathbb{Q}^{\geq 0}=\Bqty{q\in\mathbb{Q}\mid q\geq 0}
	\end{equation}
	
	\url{https://en.wikipedia.org/wiki/Rational_number}
	
	Real numbers:
	\begin{equation}
	\mathbb{R}^{\geq 0}=\Bqty{\inf A\mid A\subseteq\mathbb{Q}^+\land A\neq \Bqty{}}
	\end{equation}
	\begin{equation}
	\mathbb{R}=\Bqty{a-b\mid \Bqty{a,b}\subseteq\mathbb{R}^{\geq 0}}
	\end{equation}
	\begin{equation}
	\mathbb{R}^+=\Bqty{a\in\mathbb{R}^{\geq 0}\mid a>0}
	\end{equation}
	
	\url{https://en.wikipedia.org/wiki/Real_number}
	
	\section{Sigma summation}
	\begin{equation}
	\sum_{k\in\Bqty{}}f\pqty{k}=0
	\end{equation}
	\begin{equation}
	\sum_{k\in\Bqty{a}}f\pqty{k}=f\pqty{a}
	\end{equation}
	\begin{equation}
	\sum_{k\in\pqty{S_1\cup S_2}}f\pqty{k}
	=
	\pqty{\sum_{k\in{S_1}}f\pqty{k}}
	+
	\pqty{\sum_{k\in{S_2}}f\pqty{k}}
	-
	\pqty{\sum_{k\in\pqty{S_1\cap S_2}}f\pqty{k}}
	\end{equation}
	Shorthand
	\begin{equation}
	\sum_{k={a}}^{a-1} f\pqty{k}=0
	\end{equation}
	\begin{equation}
	\sum_{k={a}}^{b+1} f\pqty{k}=
	\pqty{\sum_{k={a}}^{b} f\pqty{k}}+f\pqty{b+1}=
	f\pqty{a}+{\sum_{k={a+1}}^{b+1} f\pqty{k}}
	\end{equation}
	
	\section{Calculus}
	A limit \(\lim\limits_{a\to b}f\pqty{a}\) is the value usually approaching \(f\pqty{b}\)
	
	\(n\)th derivative, not to be confused with \(f^{n}\pqty{x}\):
	\begin{equation}
	f^{\pqty{n}}\pqty{x}
	=
	\pdiff{x}{n}f\pqty{x}
	\end{equation}
	
	\url{https://en.wikipedia.org/wiki/Derivative#Leibniz's_notation}
	
	Integrals can be thought as the area under a curve. The integral from \(a\) to \(b\) of function \(f\pqty{x}\) with respect to \(x\) is:
	
	\begin{equation}
	\dint{a}{b}{f\pqty{x}}{x}
	\end{equation}
	
	Laplace transform:
	\begin{equation}
	\laplace{f\pqty{t}}=
	\dint{0}{\infty}{f\pqty{t}e^{-st}}{t}
	\end{equation}
	
	Laplace transform inverse:
	\begin{equation}
	\laplaceInv{\laplace{f\pqty{t}}}=f\pqty{t}
	\end{equation}
	
	\url{https://en.wikipedia.org/wiki/Laplace_transform#Formal_definition}
	
	Modular arithmetic function, \(\md:\pqty{\mathbb{C},\mathbb{C}}\to\mathbb{C}\)
	\begin{equation}
	0\leq\Re\pqty{\frac{\md\pqty{a,b}}{b}}<1
	\land
	\md\pqty{a,b}=\md\pqty{a+b,b}
	\end{equation}
	
\end{document}
