%!TEX engine = lualatex
\documentclass{article}

% --- CORE MATH & LOGIC ---
\usepackage{amsmath}
\usepackage{amsthm}
\usepackage{amssymb}
\usepackage{amsfonts}
\usepackage{physics}
\usepackage{mathtools}

% --- TYPOGRAPHY & LAYOUT ---
\usepackage[a4paper,margin=0.0931547in]{geometry}
\usepackage{fontspec}
\usepackage{microtype}
\usepackage[skip=0.5\baselineskip]{parskip}

% --- COLORS ---
\usepackage{xcolor}
% Define Palette
\definecolor{FavoriteColor1}{HTML}{607CB2} % Background (Medium Blue)
\definecolor{FavoriteColor2}{HTML}{303E59} % Text/Frame (Dark Blue)
\definecolor{LinkColor}{HTML}{FFFF00}	  % Links (Bright Yellow)

% Apply Background to Page
\pagecolor{FavoriteColor1}

% Apply Default Text Color (Global)
\makeatletter
\newcommand{\globalcolor}{\color{FavoriteColor2}\global\let\default@color\current@color}
\makeatother
\AtBeginDocument{\globalcolor}

% --- THEOREM STYLING ---
\usepackage[many]{tcolorbox}

% Define the box style
\tcbset{
	nazgandbox/.style={
		enhanced,
		% Background matches page color exactly (looks transparent)
		colback=FavoriteColor1,
		% Frame matches text color
		colframe=FavoriteColor2,
		% Text inside box matches global text color
		coltext=FavoriteColor2,
		% Title text matches background (high contrast against dark frame)
		coltitle=FavoriteColor1,
		fonttitle=\bfseries\sffamily,
		boxrule=1pt,
		arc=2pt,
		left=5pt, right=5pt, top=5pt, bottom=5pt,
		boxsep=1pt,
		sharp corners=downhill,
		breakable
	}
}

% Standard Definitions
\theoremstyle{definition}
\newtheorem{theorem}{Theorem}[section]
\newtheorem{lemma}[theorem]{Lemma}
\newtheorem{definition}[theorem]{Definition}
\newtheorem{corollary}[theorem]{Corollary}

% Wrap standard theorems in the custom box
\tcolorboxenvironment{theorem}{nazgandbox}
\tcolorboxenvironment{lemma}{nazgandbox}
\tcolorboxenvironment{definition}{nazgandbox}
\tcolorboxenvironment{corollary}{nazgandbox}

% --- UTILITIES ---
\usepackage{datetime2}
\usepackage{minted}

% --- FONTS ---
\directlua{luaotfload.add_fallback("EmojiFallback",{"NotoColorEmoji:mode=harf;"})}
\setmainfont{Noto Serif}[RawFeature={fallback=EmojiFallback}, Ligatures=TeX]
\setsansfont{Noto Sans}[RawFeature={fallback=EmojiFallback}]
\setmonofont{Noto Sans Mono}[RawFeature={fallback=EmojiFallback}]

% --- REFERENCES & LINKS ---
\usepackage{xr}
\usepackage{hyperref}
\hypersetup{
	colorlinks=true,
	linkcolor=LinkColor, % Internal links (equations, sections)
	urlcolor=LinkColor,  % Web URLs
	citecolor=LinkColor, % Bibliography citations
	filecolor=LinkColor, % Links to local files
	pdfnewwindow=true
}
\usepackage{cleveref}

% --- CUSTOM MACROS ---
\pdfstringdefDisableCommands{\def\eqref#1{(\ref{#1})}}
\newcommand{\inlineeqnum}{\refstepcounter{equation}~~\mbox{(\theequation)}}
\numberwithin{equation}{section}

% Operators
\DeclareMathOperator{\arsinh}{arsinh}
\DeclareMathOperator{\arcosh}{arcosh}
\DeclareMathOperator{\artanh}{artanh}
\DeclareMathOperator{\rues}{Rues}
\DeclareMathOperator{\md}{Mod}
\DeclareMathOperator{\pow}{Pow}

% Helpers
\newcommand{\floor}[1]{{\left\lfloor#1\right\rfloor}}
\newcommand{\ceil}[1]{{\left\lceil#1\right\rceil}}
\newcommand{\transpose}{^\intercal}
\newcommand{\laplace}[1]{\mathcal{L}\Bqty{#1}\pqty{s}}
\newcommand{\laplaceInv}[1]{\mathcal{L}^{-1}\Bqty{#1}\pqty{t}}
\newcommand{\pdiff}[2]{\frac{\partial^{#2}}{\partial #1^{#2}}}
\newcommand{\dint}[4]{\int_{#1}^{#2}#3\,\mathrm{d}#4}
\newcommand{\ketten}[4]{\underset{#1}{\overset{#2}{\mathop{\vcenter{\hbox{\huge\(\mathrm{K}\)}}}}}\frac{#3}{#4}}
\newcommand{\replace}[2]{\Big\vert_{#1\to{#2}}}

% --- METADATA & TITLE ---
\author{Mark Andrew Gerads \(<\)\href{MailTo:Nazgand@Gmail.Com}{Nazgand@Gmail.Com}\(>\)}
\let\oldauthor\author
\renewcommand{\author}[1]{
	\oldauthor{
		Author: #1
		\\
		Editor: Mark Andrew Gerads \(<\)\href{MailTo:Nazgand@Gmail.Com}{Nazgand@Gmail.Com}\(>\)
	}
}
\date{\DTMnow}
\let\oldtitle\title
\renewcommand{\title}[1]{
	\oldtitle{
		\vspace{-1.5cm}
		\url{https://GitHub.Com/Nazgand/NazgandMathBook}
		\\
		#1
	}
}

% --- LUA AUTO-REFERENCER ---
\usepackage{luacode}
\begin{luacode*}
	local lfs = require("lfs")

	local function split_path(path)
		local t = {}
		for part in path:gmatch("[^/]+") do table.insert(t, part) end
		return t
	end

	local function get_relative_path(base, target)
		local b_parts = split_path(base)
		local t_parts = split_path(target)
		local i = 1
		while b_parts[i] and t_parts[i] and b_parts[i] == t_parts[i] do
			i = i + 1
		end
		local rel = {}
		for j = i, #b_parts do table.insert(rel, "..") end
		for j = i, #t_parts do table.insert(rel, t_parts[j]) end
		return table.concat(rel, "/")
	end

	local handle = io.popen("git rev-parse --show-toplevel 2>/dev/null")
	local git_root = handle:read("*a"):gsub("%s+$", "")
	handle:close()

	if git_root ~= "" then
		local project_root = git_root .. "/LaTeX"
		local cwd = lfs.currentdir()
		
		local function scan_recursive(path)
			if not lfs.attributes(path) then return end
			for file in lfs.dir(path) do
				if file ~= "." and file ~= ".." then
					local full_path = path .. "/" .. file
					local f_attr = lfs.attributes(full_path)
					
					if f_attr and f_attr.mode == "directory" then
						scan_recursive(full_path)
					elseif file:match("%.tex$") then
						local job = file:gsub("%.tex$", "")
						if job ~= tex.jobname and job ~= "NazgandStyle" then
							local target_base = full_path:gsub("%.tex$", "")
							local rel_path = get_relative_path(cwd, target_base)
							
							tex.sprint("\\externaldocument{" .. rel_path .. "}[" .. rel_path .. ".pdf]")
						end
					end
				end
			end
		end

		texio.write_nl("--- [XR] Auto-Linking ---")
		scan_recursive(project_root)
	end
\end{luacode*}
\title{
	Trigonometry
}
\begin{document}
	
	\maketitle
	
	\section{Geometric Foundations}
	In [ElementaryGeometryAndTrigonometry.kig] by moving the angle control point to the second quadrant and comparing, we can see
	\begin{equation}
	\label{SinCosPhaseShift}
	\sin\pqty{\theta+\frac{\pi}{2}}=\cos\pqty{\theta}
	\land
	\cos\pqty{\theta+\frac{\pi}{2}}=-\sin\pqty{\theta}
	\end{equation}
	Moreover, we see as \(\theta\to 0\)
	\begin{equation}
	\label{SinXOverXLimit}
	\lim\limits_{\theta\to 0}\frac{\sin\pqty{\theta}}{\theta}=1
	\end{equation}
	Using the Pythagorean Theorem, we find
	\begin{equation}
	\label{SinCosPythagoras}
	1=\sin\pqty{\theta}^2+\cos\pqty{\theta}^2
	\end{equation}
	From [SumOfAnglesSinCos.kig], we get
	\begin{equation}
		\label{SinSumOfAngle}
		\sin\pqty{\alpha+\beta}=\sin\pqty{\alpha}\cos\pqty{\beta}+\cos\pqty{\alpha}\sin\pqty{\beta}
	\end{equation}
	And
	\begin{equation}
	\label{CosSumOfAngle}
	\cos\pqty{\alpha+\beta}=\cos\pqty{\alpha}\cos\pqty{\beta}-\sin\pqty{\alpha}\sin\pqty{\beta}
	\end{equation}
	
	\section{Derivatives}
	\begin{theorem}
	\begin{equation}
	\pdiff{x}{}\sin\pqty{x}=\cos\pqty{x}
	\end{equation}
	\begin{proof}
		\begin{equation}
			\pdiff{x}{}\sin\pqty{x}=
			\lim\limits_{h\to 0}\frac{\sin\pqty{x+h}-\sin\pqty{x}}{h}
		\end{equation}
		Use \eqref{SinSumOfAngle}
		\begin{equation}
		\pdiff{x}{}\sin\pqty{x}=
		\lim\limits_{h\to 0}\frac{\sin\pqty{x}\cos\pqty{h}+\cos\pqty{x}\sin\pqty{h}-\sin\pqty{x}}{h}
		\end{equation}
		Split the limit:
		\begin{equation}
		\pdiff{x}{}\sin\pqty{x}=
		\lim\limits_{h\to 0}\frac{\cos\pqty{x}\sin\pqty{h}}{h}
		+
		\lim\limits_{h\to 0}\frac{\sin\pqty{x}\pqty{\cos\pqty{h}-1}}{h}
		\end{equation}
		Simplify
		\begin{equation}
		\pdiff{x}{}\sin\pqty{x}=
		\cos\pqty{x}\lim\limits_{h\to 0}\frac{\sin\pqty{h}}{h}
		+
		\sin\pqty{x}\lim\limits_{h\to 0}\frac{{\cos\pqty{h}-1}}{h}
		\end{equation}
		Use \eqref{SinXOverXLimit} on left limit and multiply right limit by \(\frac{\cos\pqty{h}+1}{\cos\pqty{h}+1}\)
		\begin{equation}
		\pdiff{x}{}\sin\pqty{x}=
		\cos\pqty{x}
		+
		\sin\pqty{x}\lim\limits_{h\to 0}\frac{{\cos\pqty{h}^2-1}}{h\pqty{\cos\pqty{h}+1}}
		\end{equation}
		Use \eqref{SinCosPythagoras}
		\begin{equation}
		\pdiff{x}{}\sin\pqty{x}=
		\cos\pqty{x}
		+
		\sin\pqty{x}\lim\limits_{h\to 0}\frac{{\sin\pqty{h}^2}}{h\pqty{\cos\pqty{h}+1}}
		\end{equation}
		Split the limit:
		\begin{equation}
		\pdiff{x}{}\sin\pqty{x}=
		\cos\pqty{x}
		+
		\sin\pqty{x}\lim\limits_{a\to 0}\frac{{\sin\pqty{a}}}{a}
		\lim\limits_{b\to 0}{{\sin\pqty{b}}}
		\lim\limits_{c\to 0}\frac{1}{\pqty{\cos\pqty{c}+1}}
		\end{equation}
		Evaluate using \eqref{SinXOverXLimit} again
		\begin{equation}
		\pdiff{x}{}\sin\pqty{x}=
		\cos\pqty{x}
		+
		\sin\pqty{x}*
		1
		*0
		*\frac{1}{2}=
		\cos\pqty{x}
		\end{equation}
	\end{proof}
	\end{theorem}
	
	\begin{theorem}
		\begin{equation}
		\label{key}
		\pdiff{x}{}\cos\pqty{x}=-\sin\pqty{x}
		\end{equation}
		\begin{proof}
			Use \eqref{SinCosPhaseShift}
			\begin{equation}
			\pdiff{x}{}\cos\pqty{x}=\pdiff{x}{}\sin\pqty{x+\frac{\pi}{2}}
			\end{equation}
			Use the chain rule:
			\begin{equation}
			\pdiff{x}{}\cos\pqty{x}=
			\pqty{\pdiff{z}{}\sin\pqty{z} \replace{z}{x+\frac{\pi}{2}}}
			\pdiff{x}{}\pqty{x+\frac{\pi}{2}}
			\end{equation}
			Simplify
			\begin{equation}
			\pdiff{x}{}\cos\pqty{x}=
			\cos\pqty{x+\frac{\pi}{2}}
			\end{equation}
			Use 
			Use \eqref{SinCosPhaseShift}
			\begin{equation}
			\pdiff{x}{}\cos\pqty{x}=
			-\sin\pqty{x}
			\end{equation}
		\end{proof}
	\end{theorem}

	\section{Taylor Series}
	\begin{theorem}
		\begin{equation}
		\label{SinTaylor}
		\sin\pqty{x}=\sum_{k=0}^{\infty} \frac{\pqty{-1}^kx^{2k+1}}{\pqty{2k+1}!}
		\end{equation}
		\begin{proof}
			The multiple derivatives cycle through 4 functions. Let \(n\in\mathbb{Z}_{\ge 0}\):
			\begin{equation}
				\pdiff{x}{4n}\sin\pqty{x}=\sin\pqty{x}
			\end{equation}
			\begin{equation}
				\pdiff{x}{4n+1}\sin\pqty{x}=\cos\pqty{x}
			\end{equation}
			\begin{equation}
				\pdiff{x}{4n+2}\sin\pqty{x}=-\sin\pqty{x}
			\end{equation}
			\begin{equation}
				\pdiff{x}{4n+3}\sin\pqty{x}=-\cos\pqty{x}
			\end{equation}
			The derivatives at 0 thus cycle through \(0,1,0,-1\). Use [TaylorSeries\eqref{TaylorSeriesDef}] and simplify out the zeroes.
		\end{proof}
	\end{theorem}

	\begin{theorem}
		\begin{equation}
		\label{CosTaylor}
		\cos\pqty{x}=\sum_{k=0}^{\infty} \frac{\pqty{-1}^kx^{2k}}{\pqty{2k}!}
		\end{equation}
		\begin{proof}
			The multiple derivatives cycle through 4 functions. Let \(n\in\mathbb{Z}_{\ge 0}\):
			\begin{equation}
			\pdiff{x}{4n}\cos\pqty{x}=\cos\pqty{x}
			\end{equation}
			\begin{equation}
			\pdiff{x}{4n+1}\cos\pqty{x}=-\sin\pqty{x}
			\end{equation}
			\begin{equation}
			\pdiff{x}{4n+2}\cos\pqty{x}=-\cos\pqty{x}
			\end{equation}
			\begin{equation}
			\pdiff{x}{4n+3}\cos\pqty{x}=\sin\pqty{x}
			\end{equation}
			The derivatives at 0 thus cycle through \(1,0,-1,0\). Use [TaylorSeries\eqref{TaylorSeriesDef}] and simplify out the zeroes.
		\end{proof}
	\end{theorem}
\end{document}
