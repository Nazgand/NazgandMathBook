%!TEX engine = lualatex
\documentclass{article}

% --- CORE MATH & LOGIC ---
\usepackage{amsmath}
\usepackage{amsthm}
\usepackage{amssymb}
\usepackage{amsfonts}
\usepackage{physics}
\usepackage{mathtools}

% --- TYPOGRAPHY & LAYOUT ---
\usepackage[a4paper,margin=0.0931547in]{geometry}
\usepackage{fontspec}
\usepackage{microtype}
\usepackage[skip=0.5\baselineskip]{parskip}

% --- COLORS ---
\usepackage{xcolor}
% Define Palette
\definecolor{FavoriteColor1}{HTML}{607CB2} % Background (Medium Blue)
\definecolor{FavoriteColor2}{HTML}{303E59} % Text/Frame (Dark Blue)
\definecolor{LinkColor}{HTML}{FFFF00}	  % Links (Bright Yellow)

% Apply Background to Page
\pagecolor{FavoriteColor1}

% Apply Default Text Color (Global)
\makeatletter
\newcommand{\globalcolor}{\color{FavoriteColor2}\global\let\default@color\current@color}
\makeatother
\AtBeginDocument{\globalcolor}

% --- THEOREM STYLING ---
\usepackage[many]{tcolorbox}

% Define the box style
\tcbset{
	nazgandbox/.style={
		enhanced,
		% Background matches page color exactly (looks transparent)
		colback=FavoriteColor1,
		% Frame matches text color
		colframe=FavoriteColor2,
		% Text inside box matches global text color
		coltext=FavoriteColor2,
		% Title text matches background (high contrast against dark frame)
		coltitle=FavoriteColor1,
		fonttitle=\bfseries\sffamily,
		boxrule=1pt,
		arc=2pt,
		left=5pt, right=5pt, top=5pt, bottom=5pt,
		boxsep=1pt,
		sharp corners=downhill,
		breakable
	}
}

% Standard Definitions
\theoremstyle{definition}
\newtheorem{theorem}{Theorem}[section]
\newtheorem{lemma}[theorem]{Lemma}
\newtheorem{definition}[theorem]{Definition}
\newtheorem{corollary}[theorem]{Corollary}

% Wrap standard theorems in the custom box
\tcolorboxenvironment{theorem}{nazgandbox}
\tcolorboxenvironment{lemma}{nazgandbox}
\tcolorboxenvironment{definition}{nazgandbox}
\tcolorboxenvironment{corollary}{nazgandbox}

% --- UTILITIES ---
\usepackage{datetime2}
\usepackage{minted}

% --- FONTS ---
\directlua{luaotfload.add_fallback("EmojiFallback",{"NotoColorEmoji:mode=harf;"})}
\setmainfont{Noto Serif}[RawFeature={fallback=EmojiFallback}, Ligatures=TeX]
\setsansfont{Noto Sans}[RawFeature={fallback=EmojiFallback}]
\setmonofont{Noto Sans Mono}[RawFeature={fallback=EmojiFallback}]

% --- REFERENCES & LINKS ---
\usepackage{xr}
\usepackage{hyperref}
\hypersetup{
	colorlinks=true,
	linkcolor=LinkColor, % Internal links (equations, sections)
	urlcolor=LinkColor,  % Web URLs
	citecolor=LinkColor, % Bibliography citations
	filecolor=LinkColor, % Links to local files
	pdfnewwindow=true
}
\usepackage{cleveref}

% --- CUSTOM MACROS ---
\pdfstringdefDisableCommands{\def\eqref#1{(\ref{#1})}}
\newcommand{\inlineeqnum}{\refstepcounter{equation}~~\mbox{(\theequation)}}
\numberwithin{equation}{section}

% Operators
\DeclareMathOperator{\arsinh}{arsinh}
\DeclareMathOperator{\arcosh}{arcosh}
\DeclareMathOperator{\artanh}{artanh}
\DeclareMathOperator{\rues}{Rues}
\DeclareMathOperator{\md}{Mod}
\DeclareMathOperator{\pow}{Pow}

% Helpers
\newcommand{\floor}[1]{{\left\lfloor#1\right\rfloor}}
\newcommand{\ceil}[1]{{\left\lceil#1\right\rceil}}
\newcommand{\transpose}{^\intercal}
\newcommand{\laplace}[1]{\mathcal{L}\Bqty{#1}\pqty{s}}
\newcommand{\laplaceInv}[1]{\mathcal{L}^{-1}\Bqty{#1}\pqty{t}}
\newcommand{\pdiff}[2]{\frac{\partial^{#2}}{\partial #1^{#2}}}
\newcommand{\dint}[4]{\int_{#1}^{#2}#3\,\mathrm{d}#4}
\newcommand{\ketten}[4]{\underset{#1}{\overset{#2}{\mathop{\vcenter{\hbox{\huge\(\mathrm{K}\)}}}}}\frac{#3}{#4}}
\newcommand{\replace}[2]{\Big\vert_{#1\to{#2}}}

% --- METADATA & TITLE ---
\author{Mark Andrew Gerads \(<\)\href{MailTo:Nazgand@Gmail.Com}{Nazgand@Gmail.Com}\(>\)}
\let\oldauthor\author
\renewcommand{\author}[1]{
	\oldauthor{
		Author: #1
		\\
		Editor: Mark Andrew Gerads \(<\)\href{MailTo:Nazgand@Gmail.Com}{Nazgand@Gmail.Com}\(>\)
	}
}
\date{\DTMnow}
\let\oldtitle\title
\renewcommand{\title}[1]{
	\oldtitle{
		\vspace{-1.5cm}
		\url{https://GitHub.Com/Nazgand/NazgandMathBook}
		\\
		#1
	}
}

% --- LUA AUTO-REFERENCER ---
\usepackage{luacode}
\begin{luacode*}
	local lfs = require("lfs")

	local function split_path(path)
		local t = {}
		for part in path:gmatch("[^/]+") do table.insert(t, part) end
		return t
	end

	local function get_relative_path(base, target)
		local b_parts = split_path(base)
		local t_parts = split_path(target)
		local i = 1
		while b_parts[i] and t_parts[i] and b_parts[i] == t_parts[i] do
			i = i + 1
		end
		local rel = {}
		for j = i, #b_parts do table.insert(rel, "..") end
		for j = i, #t_parts do table.insert(rel, t_parts[j]) end
		return table.concat(rel, "/")
	end

	local handle = io.popen("git rev-parse --show-toplevel 2>/dev/null")
	local git_root = handle:read("*a"):gsub("%s+$", "")
	handle:close()

	if git_root ~= "" then
		local project_root = git_root .. "/LaTeX"
		local cwd = lfs.currentdir()
		
		local function scan_recursive(path)
			if not lfs.attributes(path) then return end
			for file in lfs.dir(path) do
				if file ~= "." and file ~= ".." then
					local full_path = path .. "/" .. file
					local f_attr = lfs.attributes(full_path)
					
					if f_attr and f_attr.mode == "directory" then
						scan_recursive(full_path)
					elseif file:match("%.tex$") then
						local job = file:gsub("%.tex$", "")
						if job ~= tex.jobname and job ~= "NazgandStyle" then
							local target_base = full_path:gsub("%.tex$", "")
							local rel_path = get_relative_path(cwd, target_base)
							
							tex.sprint("\\externaldocument{" .. rel_path .. "}[" .. rel_path .. ".pdf]")
						end
					end
				end
			end
		end

		texio.write_nl("--- [XR] Auto-Linking ---")
		scan_recursive(project_root)
	end
\end{luacode*}
\title{
	Exponential Function
}
\begin{document}
	
	\maketitle
	
	\section{Power function definition}
	\begin{definition}
			Here I formally define the power function.
		\begin{equation}
			\pow\pqty{x,y}=x^y
		\end{equation}
		\begin{equation}
			\pow\pqty{x,0}=1
		\end{equation}
		\begin{equation}
			\exists\pow\pqty{x,y-1}\Rightarrow\pow\pqty{x,y}=\pow\pqty{x,y-1}*x
		\end{equation}
		\begin{equation}
			{\exists\pow\pqty{x,b}^{-1}}\Rightarrow\pow\pqty{x,-b}=\pow\pqty{x,b}^{-1}
		\end{equation}
		\begin{equation}
			\bqty{\exists\pow\pqty{x,a}\land\exists\pow\pqty{x,b}}\Rightarrow\pow\pqty{x,a+b}=\pow\pqty{x,a}\pow\pqty{x,b}
		\end{equation}
		\begin{equation}
			\bqty{x\in\mathbb{R}^+ \land y\in\mathbb{R}}\Rightarrow\exists\pow\pqty{x,y}\in\mathbb{R}
		\end{equation}
		\begin{equation}
			\bqty{x\in\mathbb{R}^+\land y\in\mathbb{R}}
			\Rightarrow\exists\pqty{\pdiff{z}{}\pow\pqty{x,z}\replace{z}{y}}
		\end{equation}
	\end{definition}
	
	\section{Basic results and definition of \(e\)}
	\begin{theorem}
		The derivative of an exponential function is a multiple of the same exponential function.
		\begin{equation}
		\pdiff{y}{}x^y=\pqty{\pdiff{z}{}x^z\replace{z}{0}}x^y
		\end{equation}
		\begin{proof}
		Substitute \(y\to z+a\) in the left side of the equation to get the right side
		\begin{equation}
			\pdiff{y}{}x^y=\pqty{\pdiff{z}{}x^{z+a}\replace{z}{\pqty{y-a}}}
		\end{equation}
		Substitute \(x^{z+a}\to x^zx^a\)
		\begin{equation}
			\pdiff{y}{}x^y=\pqty{\pdiff{z}{}x^zx^a\replace{z}{\pqty{y-a}}}
		\end{equation}
		Bring \(x^a\) out:
		\begin{equation}
			\pdiff{y}{}x^y=x^a\pqty{\pdiff{z}{}x^z\replace{z}{\pqty{y-a}}}
		\end{equation}
		Substitute \(a\to y\)
		\begin{equation}
			\label{expDiffMultOfExp}
			\pdiff{y}{}x^y=x^y\pqty{\pdiff{z}{}x^z\replace{z}{0}}
		\end{equation}
		\end{proof}
	\end{theorem}
	
	\begin{definition}
		Define \(e\) to be the base of the exponential function which has a derivative of \(1\) at \(0\).
		\begin{equation}
		\label{defineE}
			1=\pqty{\pdiff{z}{}e^z\replace{z}{0}}
		\end{equation}
	\end{definition}
	
	\begin{lemma}
		\begin{equation}
		\label{expOwnDeriv}
		\pdiff{y}{}e^y=e^y
		\end{equation}
		\begin{proof}
		Substitute \(x\to e\) in \eqref{expDiffMultOfExp} and simplify with \eqref{defineE}.
		\end{proof}
	\end{lemma}
	
	\begin{corollary}
		The derivative of an exponential function is the natural logarithm of the base of the exponential function times the exponential function.
		\begin{equation}
			\pdiff{y}{}x^y
			={\ln\pqty{x}}x^y
		\end{equation}
		\begin{proof}
		\begin{equation}
			\pqty{\pdiff{z}{}x^z\replace{z}{0}}=\ln\pqty{x}
		\end{equation}
		\begin{equation}
			\pdiff{y}{}x^y=x^y\pqty{\pdiff{z}{}x^z\replace{z}{0}}
		\end{equation}
		Substitute \(x\to e^{\ln\pqty{x}}\) in the left to get the right
		\begin{equation}
			\pdiff{y}{}x^y=\pdiff{y}{}e^{\ln\pqty{x}y}
		\end{equation}
		Apply the chain rule:
		\begin{equation}
			\pdiff{y}{}x^y
			=\pqty{\pdiff{z}{}e^{z}\replace{z}{\ln\pqty{x}y}}*\pdiff{y}{}{\ln\pqty{x}y}
		\end{equation}
		Simplify
		\begin{equation}
			\pdiff{y}{}x^y
			=\pqty{e^{z}\replace{z}{\ln\pqty{x}y}}*{\ln\pqty{x}}
			=e^{\ln\pqty{x}y}*{\ln\pqty{x}}
			={\ln\pqty{x}}x^y
		\end{equation}
		\end{proof}
	\end{corollary}
	
	\section{Exponential Function Derivative}
	\begin{theorem}
		\begin{equation}
			e^x=\sum_{k=0}^{\infty}\frac{x^k}{k!}
		\end{equation}
		\begin{proof}
		We have a formula which is it's own derivative \eqref{expOwnDeriv}.
		Another formula which is it's own derivative is
		\begin{equation}
			\exp\pqty{x}
			=\sum_{k=0}^{\infty}\frac{x^k}{k!}
			=\pdiff{x}{}\exp\pqty{x}
		\end{equation}
		The differential equation \(f\pqty{x}=f'\pqty{x}\) has 1 degree of freedom which is filled by \(f\pqty{0}=1\) by both formulae. Thus both formulae express the same function; \(e^x=\exp\pqty{x}\).
		\end{proof}
	\end{theorem}
	
	\section{Convergence of \(\exp\pqty{x}\)}
	\begin{theorem}
		\(\exp\pqty{x}\) converges for all \(x\in\mathbb{C}\)
		\begin{proof}
		By the triangle inequality, an upper bound and a lower bound exist for all complex numbers.
		\begin{equation}
			-\frac{\abs{x}^k}{k!}\leq
			\frac{x^k}{k!}\leq
			\frac{\abs{x}^k}{k!}
		\end{equation}
		\begin{equation}
			-\exp\pqty{\abs{x}}\leq
			\exp\pqty{x}\leq
			\exp\pqty{\abs{x}}
		\end{equation}
		Thus convergence for \(x\in\mathbb{R}^+\) implies convergence for \(x\in\mathbb{C}\). Let \(x\in\mathbb{R}^+\). Bound part of the sum by a geometric series:
		\begin{equation}
			\exp\pqty{x}
			=\sum_{k=0}^{n-1}\frac{x^k}{k!}
			+\sum_{k=n}^{\infty}\frac{x^k}{k!}
			<\sum_{k=0}^{n-1}\frac{x^k}{k!}
			+\sum_{k=n}^{\infty}\frac{x^k}{n^{k-n}\pqty{n}!}
		\end{equation}
		Simplify
		\begin{equation}
			\sum_{k=n}^{\infty}\frac{x^k}{n^{k-n}\pqty{n}!}
			=\frac{n^{n}}{\pqty{n}!}\sum_{k=n}^{\infty}\pqty{\frac{x}{n}}^k
			=\frac{x^{n}}{\pqty{n}!}\sum_{m=0}^{\infty}\pqty{\frac{x}{n}}^{m}
		\end{equation}
		Find where the bounding geometric series converges. [GeometricSeries\eqref{GeometricSeries}]
		\begin{equation}
			\frac{x}{n}<1
			\Rightarrow
			\sum_{m=0}^{\infty}\pqty{\frac{x}{n}}^{m}
			=\frac{1}{1-\frac{x}{n}}
		\end{equation}
		
		Every specific \(x\) has an integer larger than it and is bounded by a circle of convergence from a corresponding geometric series. Let \(n\to\infty\) and \(\exp\pqty{x}\) converges for all \(x\in\mathbb{C}\).
		\end{proof}
	\end{theorem}
	
	\section{Limit Form of E}
	\begin{theorem}
		\begin{equation}
		\label{limitE}
		e=\lim\limits_{n\to\infty}\pqty{1+\frac{1}{n}}^n
		=\lim\limits_{n\to\infty}\pqty{\frac{1+n}{n}}^n
		\end{equation}
		\begin{proof}
		Proof from \url{https://mathcs.clarku.edu/~djoyce/ma122/elimit.pdf}.
		Note
		\begin{equation}
			\bqty{\Bqty{n,t}\subset\mathbb{R}^+\land 1\leq t\leq \frac{1+n}{n}}
			\Rightarrow
			1\geq \frac{1}{t}\geq \frac{n}{1+n}
		\end{equation}
		Integrate over the inequality:
		\begin{equation}
			\dint{1}{\frac{1+n}{n}}{1}{t}
			\geq \dint{1}{\frac{1+n}{n}}{\frac{1}{t}}{t}
			\geq \dint{1}{\frac{1+n}{n}}{\frac{n}{1+n}}{t}
		\end{equation}
		Simplify using [Logarithms\eqref{LnIntegralForm}]
		\begin{equation}
			\frac{1}{n}
			\geq \ln\pqty{\frac{1+n}{n}}
			\geq \frac{1}{n+1}
		\end{equation}
		Apply the exponential function:
		\begin{equation}
			e^{\frac{1}{n}}
			\geq \frac{1+n}{n}
			\geq e^{\frac{1}{n+1}}
		\end{equation}
		Raise to the power of \(n\) and \(n+1\)
		\begin{equation}
			e
			\geq \pqty{\frac{1+n}{n}}^n
			\land
			\pqty{\frac{1+n}{n}}^{n+1}
			\geq e
		\end{equation}
		Divide
		\begin{equation}
			e
			\geq \pqty{\frac{1+n}{n}}^n
			\land
			\pqty{\frac{1+n}{n}}^n
			\geq \frac{en}{1+n}
		\end{equation}
		Let \(n\to\infty\) using the squeeze theorem, simplify.
		\begin{equation}
			e
			\geq \lim\limits_{n\to\infty}\pqty{\frac{1+n}{n}}^n
			\geq e
		\end{equation}
		\end{proof}
	\end{theorem}
	
	\section{Exponential Function Limit Form}
	\begin{theorem}
		\begin{equation}
			\label{expLim2}
			e^x=\lim\limits_{m\to\infty}\pqty{1+\frac{x}{m}}^m
		\end{equation}
		\begin{proof}
		Raise \eqref{limitE} to the power of \(x\)
		\begin{equation}
			e^x=\lim\limits_{n\to\infty}\pqty{1+\frac{1}{n}}^{xn}
		\end{equation}
		For \(x\in\mathbb{R}^+\), a substitution \(n\to \frac{m}{x}\) can be made to obtain a limit known to exist.
		\begin{equation}
			e^x=\lim\limits_{m\to\infty}\pqty{1+\frac{x}{m}}^m
		\end{equation}
		The existence of the limit for \(x\in\mathbb{R}^+\) extends analytically to \(x\in\mathbb{C}\) because the new formula fulfills the differential equation \(f\pqty{x}=f'\pqty{x}\) and has \(f\pqty{0}=1\).
		Chain rule used:
		\begin{equation}
			\pdiff{x}{}\lim\limits_{m\to\infty}\pqty{1+\frac{x}{m}}^m
			=
			\lim\limits_{m\to\infty}\pqty{\pdiff{z}{}z^m\replace{z}{\pqty{1+\frac{x}{m}}}}*\pdiff{x}{}\pqty{1+\frac{x}{m}}
		\end{equation}
		Simplify:
		\begin{equation}
			\pdiff{x}{}\lim\limits_{m\to\infty}\pqty{1+\frac{x}{m}}^m
			=
			\lim\limits_{m\to\infty}m\pqty{1+\frac{x}{m}}^{m-1}*\frac{1}{m}
		\end{equation}
		Split the limit
		\begin{equation}
			\pdiff{x}{}\lim\limits_{m\to\infty}\pqty{1+\frac{x}{m}}^m
			=
			\lim\limits_{m\to\infty}\pqty{1+\frac{x}{m}}^{m}*\pqty{1+\frac{x}{m}}^{-1}
		\end{equation}
		\begin{equation}
			\pdiff{x}{}\lim\limits_{m\to\infty}\pqty{1+\frac{x}{m}}^m
			=
			\lim\limits_{m\to\infty}\pqty{1+\frac{x}{m}}^{m}*
			\lim\limits_{p\to\infty}\pqty{1+\frac{x}{p}}^{-1}
		\end{equation}
		Simplify to see \(f\pqty{x}=f'\pqty{x}\).
		\begin{equation}
			\pdiff{x}{}\lim\limits_{m\to\infty}\pqty{1+\frac{x}{m}}^m
			=
			\lim\limits_{m\to\infty}\pqty{1+\frac{x}{m}}^{m}
		\end{equation}
		\end{proof}
	\end{theorem}
	
	\section{Euler's Identity}
	\begin{theorem}
		\begin{equation}
		e^{ix}=\cos\pqty{x}+i\sin\pqty{x}
		\end{equation}
		\begin{proof}
		\begin{equation}
			\pdiff{x}{n}e^{ix}=i^ne^{ix}
		\end{equation}
		The derivatives at 0 thus cycle through \(1,i,-1,-i\).
		Use [TaylorSeries\eqref{TaylorSeriesDef}] and compare to [Trigonometry\eqref{SinTaylor}] and [Trigonometry\eqref{CosTaylor}]
		\end{proof}
	\end{theorem}
	
	\section{Bibliography}
	\url{https://en.wikipedia.org/wiki/Exponential_function}
	
\end{document}
