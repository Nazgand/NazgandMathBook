%!TEX engine = lualatex
\documentclass{article}

% --- CORE MATH & LOGIC ---
\usepackage{amsmath}
\usepackage{amsthm}
\usepackage{amssymb}
\usepackage{amsfonts}
\usepackage{physics}
\usepackage{mathtools}

% --- TYPOGRAPHY & LAYOUT ---
\usepackage[a4paper,margin=0.0931547in]{geometry}
\usepackage{fontspec}
\usepackage{microtype}
\usepackage[skip=0.5\baselineskip]{parskip}

% --- COLORS ---
\usepackage{xcolor}
% Define Palette
\definecolor{FavoriteColor1}{HTML}{607CB2} % Background (Medium Blue)
\definecolor{FavoriteColor2}{HTML}{303E59} % Text/Frame (Dark Blue)
\definecolor{LinkColor}{HTML}{FFFF00}	  % Links (Bright Yellow)

% Apply Background to Page
\pagecolor{FavoriteColor1}

% Apply Default Text Color (Global)
\makeatletter
\newcommand{\globalcolor}{\color{FavoriteColor2}\global\let\default@color\current@color}
\makeatother
\AtBeginDocument{\globalcolor}

% --- THEOREM STYLING ---
\usepackage[many]{tcolorbox}

% Define the box style
\tcbset{
	nazgandbox/.style={
		enhanced,
		% Background matches page color exactly (looks transparent)
		colback=FavoriteColor1,
		% Frame matches text color
		colframe=FavoriteColor2,
		% Text inside box matches global text color
		coltext=FavoriteColor2,
		% Title text matches background (high contrast against dark frame)
		coltitle=FavoriteColor1,
		fonttitle=\bfseries\sffamily,
		boxrule=1pt,
		arc=2pt,
		left=5pt, right=5pt, top=5pt, bottom=5pt,
		boxsep=1pt,
		sharp corners=downhill,
		breakable
	}
}

% Standard Definitions
\theoremstyle{definition}
\newtheorem{theorem}{Theorem}[section]
\newtheorem{lemma}[theorem]{Lemma}
\newtheorem{definition}[theorem]{Definition}
\newtheorem{corollary}[theorem]{Corollary}

% Wrap standard theorems in the custom box
\tcolorboxenvironment{theorem}{nazgandbox}
\tcolorboxenvironment{lemma}{nazgandbox}
\tcolorboxenvironment{definition}{nazgandbox}
\tcolorboxenvironment{corollary}{nazgandbox}

% --- UTILITIES ---
\usepackage{datetime2}
\usepackage{minted}

% --- FONTS ---
\directlua{luaotfload.add_fallback("EmojiFallback",{"NotoColorEmoji:mode=harf;"})}
\setmainfont{Noto Serif}[RawFeature={fallback=EmojiFallback}, Ligatures=TeX]
\setsansfont{Noto Sans}[RawFeature={fallback=EmojiFallback}]
\setmonofont{Noto Sans Mono}[RawFeature={fallback=EmojiFallback}]

% --- REFERENCES & LINKS ---
\usepackage{xr}
\usepackage{hyperref}
\hypersetup{
	colorlinks=true,
	linkcolor=LinkColor, % Internal links (equations, sections)
	urlcolor=LinkColor,  % Web URLs
	citecolor=LinkColor, % Bibliography citations
	filecolor=LinkColor, % Links to local files
	pdfnewwindow=true
}
\usepackage{cleveref}

% --- CUSTOM MACROS ---
\pdfstringdefDisableCommands{\def\eqref#1{(\ref{#1})}}
\newcommand{\inlineeqnum}{\refstepcounter{equation}~~\mbox{(\theequation)}}
\numberwithin{equation}{section}

% Operators
\DeclareMathOperator{\arsinh}{arsinh}
\DeclareMathOperator{\arcosh}{arcosh}
\DeclareMathOperator{\artanh}{artanh}
\DeclareMathOperator{\rues}{Rues}
\DeclareMathOperator{\md}{Mod}
\DeclareMathOperator{\pow}{Pow}

% Helpers
\newcommand{\floor}[1]{{\left\lfloor#1\right\rfloor}}
\newcommand{\ceil}[1]{{\left\lceil#1\right\rceil}}
\newcommand{\transpose}{^\intercal}
\newcommand{\laplace}[1]{\mathcal{L}\Bqty{#1}\pqty{s}}
\newcommand{\laplaceInv}[1]{\mathcal{L}^{-1}\Bqty{#1}\pqty{t}}
\newcommand{\pdiff}[2]{\frac{\partial^{#2}}{\partial #1^{#2}}}
\newcommand{\dint}[4]{\int_{#1}^{#2}#3\,\mathrm{d}#4}
\newcommand{\ketten}[4]{\underset{#1}{\overset{#2}{\mathop{\vcenter{\hbox{\huge\(\mathrm{K}\)}}}}}\frac{#3}{#4}}
\newcommand{\replace}[2]{\Big\vert_{#1\to{#2}}}

% --- METADATA & TITLE ---
\author{Mark Andrew Gerads \(<\)\href{MailTo:Nazgand@Gmail.Com}{Nazgand@Gmail.Com}\(>\)}
\let\oldauthor\author
\renewcommand{\author}[1]{
	\oldauthor{
		Author: #1
		\\
		Editor: Mark Andrew Gerads \(<\)\href{MailTo:Nazgand@Gmail.Com}{Nazgand@Gmail.Com}\(>\)
	}
}
\date{\DTMnow}
\let\oldtitle\title
\renewcommand{\title}[1]{
	\oldtitle{
		\vspace{-1.5cm}
		\url{https://GitHub.Com/Nazgand/NazgandMathBook}
		\\
		#1
	}
}

% --- LUA AUTO-REFERENCER ---
\usepackage{luacode}
\begin{luacode*}
	local lfs = require("lfs")

	local function split_path(path)
		local t = {}
		for part in path:gmatch("[^/]+") do table.insert(t, part) end
		return t
	end

	local function get_relative_path(base, target)
		local b_parts = split_path(base)
		local t_parts = split_path(target)
		local i = 1
		while b_parts[i] and t_parts[i] and b_parts[i] == t_parts[i] do
			i = i + 1
		end
		local rel = {}
		for j = i, #b_parts do table.insert(rel, "..") end
		for j = i, #t_parts do table.insert(rel, t_parts[j]) end
		return table.concat(rel, "/")
	end

	local handle = io.popen("git rev-parse --show-toplevel 2>/dev/null")
	local git_root = handle:read("*a"):gsub("%s+$", "")
	handle:close()

	if git_root ~= "" then
		local project_root = git_root .. "/LaTeX"
		local cwd = lfs.currentdir()
		
		local function scan_recursive(path)
			if not lfs.attributes(path) then return end
			for file in lfs.dir(path) do
				if file ~= "." and file ~= ".." then
					local full_path = path .. "/" .. file
					local f_attr = lfs.attributes(full_path)
					
					if f_attr and f_attr.mode == "directory" then
						scan_recursive(full_path)
					elseif file:match("%.tex$") then
						local job = file:gsub("%.tex$", "")
						if job ~= tex.jobname and job ~= "NazgandStyle" then
							local target_base = full_path:gsub("%.tex$", "")
							local rel_path = get_relative_path(cwd, target_base)
							
							tex.sprint("\\externaldocument{" .. rel_path .. "}[" .. rel_path .. ".pdf]")
						end
					end
				end
			end
		end

		texio.write_nl("--- [XR] Auto-Linking ---")
		scan_recursive(project_root)
	end
\end{luacode*}
% Title and author
\title{
	Proofs of the Argument Sum Conjectures for
	
	Homogeneous Linear Differential Equations
}
\author{Grok 3}

\begin{document}
	
	\maketitle
	
	\section{Introduction}
	
	We consider homogeneous linear differential equations with constant coefficients of the form:
	
	\begin{equation}
		0 = \sum_{k=0}^n a_k f^{\pqty{k}}\pqty{z}, \quad a_k \in \mathbb{C}, \quad a_n \neq 0,
	\end{equation}
	
	where \(f^{\pqty{k}}\pqty{z}\) denotes the \(k\)-th derivative of \(f\pqty{z}\).
	
	The solution space of this equation is \(n\)-dimensional, and we let \(\{g_0\pqty{z}, g_1\pqty{z}, \ldots, g_{n-1}\pqty{z}\}\) be a basis for this space.
	
	Define the vector:
	
	\begin{equation}
		v\pqty{z} = \begin{bmatrix} g_0\pqty{z} \\ g_1\pqty{z} \\ \vdots \\ g_{n-1}\pqty{z} \end{bmatrix}.
	\end{equation}
	
	The conjectures \(\text{ArgSumCon}\pqty{m}\) for positive integers \(m \in \mathbb{Z}_{>0}\) assert that for any solution \(f\), there exist constants \(c\pqty{k_0, \ldots, k_{m-1}}\) such that:
	
	\begin{equation}
		f\pqty{\sum_{j=0}^{m-1} z_j} = \sum_{k_0=0}^{n-1} \cdots \sum_{k_{m-1}=0}^{n-1} c\pqty{k_0, \ldots, k_{m-1}} \prod_{j=0}^{m-1} g_{k_j}\pqty{z_j}.
	\end{equation}
	
	In this document, we prove \(\text{ArgSumCon}\pqty{2}\) explicitly, generalize to \(\text{ArgSumCon}\pqty{m}\) using induction, and justify the invertibility of any matrix inverted during the proofs.
	
	\section{Proof of \(\text{ArgSumCon}\pqty{2}\)}
	
	\begin{theorem}
		For any solution \(f\) to the differential equation, there exists a unique symmetric matrix \(A \in \mathbb{C}^{n \times n}\) such that:
		
		\begin{equation}
			f\pqty{z_0 + z_1} = v\pqty{z_1}\transpose A v\pqty{z_0}.
		\end{equation}
		\begin{proof}
			We proceed through a series of steps, using lemmas and claims to build the result.
			
			\begin{lemma}
				For any solution \(f\) and fixed \(z_1 \in \mathbb{C}\), the function \(h\pqty{z_0} := f\pqty{z_0 + z_1}\) is also a solution to the differential equation.
				\begin{proof}
					Since the differential equation has constant coefficients, compute the derivatives:
					
					\begin{equation}
						h^{\pqty{k}}\pqty{z_0} = f^{\pqty{k}}\pqty{z_0 + z_1}.
					\end{equation}
					
					Applying the differential operator:
					
					\begin{equation}
						\sum_{k=0}^n a_k h^{\pqty{k}}\pqty{z_0} = \sum_{k=0}^n a_k f^{\pqty{k}}\pqty{z_0 + z_1} = 0,
					\end{equation}
					
					since \(f\) satisfies the original equation. Thus, \(h\pqty{z_0}\) is a solution.
				\end{proof}
			\end{lemma}
			
			\begin{lemma}
				There exist infinitely differentiable functions \(c_k: \mathbb{C} \to \mathbb{C}\) such that:
				
				\begin{equation}
					f\pqty{z_0 + z_1} = \sum_{k=0}^{n-1} c_k\pqty{z_1} g_k\pqty{z_0}.
				\end{equation}
				\begin{proof}
					Fix \(z_1\). Since \(h\pqty{z_0} = f\pqty{z_0 + z_1}\) is a solution, and \(\{g_0\pqty{z_0}, \ldots, g_{n-1}\pqty{z_0}\}\) is a basis, we can express:
					
					\begin{equation}
						h\pqty{z_0} = \sum_{k=0}^{n-1} c_k\pqty{z_1} g_k\pqty{z_0},
					\end{equation}
					
					where the coefficients \(c_k\pqty{z_1}\) depend on \(z_1\). To find these coefficients, consider the Wronskian matrix of the basis at \(z_0 = 0\):
					
					\begin{equation}
						W\pqty{0} = \begin{bmatrix}
							g_0\pqty{0} & g_1\pqty{0} & \cdots & g_{n-1}\pqty{0} \\
							g_0'\pqty{0} & g_1'\pqty{0} & \cdots & g_{n-1}'\pqty{0} \\
							\vdots & \vdots & \ddots & \vdots \\
							g_0^{\pqty{n-1}}\pqty{0} & g_1^{\pqty{n-1}}\pqty{0} & \cdots & g_{n-1}^{\pqty{n-1}}\pqty{0}
						\end{bmatrix}.
					\end{equation}
					
					Since \(\{g_0, \ldots, g_{n-1}\}\) are linearly independent solutions, \(W\pqty{0}\) is invertible (justified in Section 4). Solve for the coefficients using initial conditions at \(z_0 = 0\):
					
					\begin{equation}
						\begin{bmatrix} h\pqty{0} \\ h'\pqty{0} \\ \vdots \\ h^{\pqty{n-1}}\pqty{0} \end{bmatrix} = \begin{bmatrix} f\pqty{z_1} \\ f'\pqty{z_1} \\ \vdots \\ f^{\pqty{n-1}}\pqty{z_1} \end{bmatrix} = W\pqty{0} \begin{bmatrix} c_0\pqty{z_1} \\ c_1\pqty{z_1} \\ \vdots \\ c_{n-1}\pqty{z_1} \end{bmatrix}.
					\end{equation}
					
					Thus:
					
					\begin{equation}
						\begin{bmatrix} c_0\pqty{z_1} \\ c_1\pqty{z_1} \\ \vdots \\ c_{n-1}\pqty{z_1} \end{bmatrix} = W\pqty{0}^{-1} \begin{bmatrix} f\pqty{z_1} \\ f'\pqty{z_1} \\ \vdots \\ f^{\pqty{n-1}}\pqty{z_1} \end{bmatrix}.
					\end{equation}
					
					Since \(f\) is infinitely differentiable (as a solution to a linear ODE with constant coefficients), each \(c_k\pqty{z_1}\) is infinitely differentiable.
				\end{proof}
			\end{lemma}
			
			\begin{theorem}
				Each coefficient \(c_k\pqty{z_1}\) satisfies the differential equation:
				
				\begin{equation}
					\sum_{j=0}^n a_j c_k^{\pqty{j}}\pqty{z_1} = 0.
				\end{equation}
				\begin{proof}
					Differentiate \(f\pqty{z_0 + z_1} = \sum_{k=0}^{n-1} c_k\pqty{z_1} g_k\pqty{z_0}\) with respect to \(z_1\):
					
					\begin{equation}
						\pdiff{z_1}{j} f\pqty{z_0 + z_1} = \sum_{k=0}^{n-1} c_k^{\pqty{j}}\pqty{z_1} g_k\pqty{z_0}.
					\end{equation}
					
					Since \(f\pqty{z_0 + z_1}\) satisfies the differential equation in \(z_0 + z_1\), apply the operator with respect to \(z_1\) (noting \(\pdiff{z_1}{} f = f'\)):
					
					\begin{equation}
						0 = \sum_{j=0}^n a_j \pdiff{z_1}{j} f\pqty{z_0 + z_1} = \sum_{j=0}^n a_j \sum_{k=0}^{n-1} c_k^{\pqty{j}}\pqty{z_1} g_k\pqty{z_0} = \sum_{k=0}^{n-1} \pqty{\sum_{j=0}^n a_j c_k^{\pqty{j}}\pqty{z_1}} g_k\pqty{z_0}.
					\end{equation}
					
					Since the \(g_k\pqty{z_0}\) are linearly independent, each coefficient must vanish:
					
					\begin{equation}
						\sum_{j=0}^n a_j c_k^{\pqty{j}}\pqty{z_1} = 0,
					\end{equation}
					
					proving that each \(c_k\pqty{z_1}\) is a solution.
				\end{proof}
			\end{theorem}
			
			\begin{lemma}
				There exist constants \(d_{km} \in \mathbb{C}\) such that:
				
				\begin{equation}
					c_k\pqty{z_1} = \sum_{m=0}^{n-1} d_{km} g_m\pqty{z_1}.
				\end{equation}
				\begin{proof}
					Since \(c_k\pqty{z_1}\) satisfies the differential equation and \(\{g_0\pqty{z_1}, \ldots, g_{n-1}\pqty{z_1}\}\) is a basis for the solution space, we can write:
					
					\begin{equation}
						c_k\pqty{z_1} = \sum_{m=0}^{n-1} d_{km} g_m\pqty{z_1},
					\end{equation}
					
					where the \(d_{km}\) are constants because the differential equation has constant coefficients.
				\end{proof}
			\end{lemma}
			
			Now, substitute into the expression for \(f\):
			
			\begin{equation}
				f\pqty{z_0 + z_1} = \sum_{k=0}^{n-1} c_k\pqty{z_1} g_k\pqty{z_0} = \sum_{k=0}^{n-1} \pqty{\sum_{m=0}^{n-1} d_{km} g_m\pqty{z_1}} g_k\pqty{z_0} = \sum_{k=0}^{n-1} \sum_{m=0}^{n-1} d_{km} g_m\pqty{z_1} g_k\pqty{z_0}.
			\end{equation}
			
			Define the matrix \(D\) with entries \(D_{mk} = d_{km}\), so:
			
			\begin{equation}
				f\pqty{z_0 + z_1} = v\pqty{z_1}\transpose D v\pqty{z_0}.
			\end{equation}
			
			Since \(f\pqty{z_0 + z_1} = f\pqty{z_1 + z_0}\), we have:
			
			\begin{equation}
				v\pqty{z_1}\transpose D v\pqty{z_0} = v\pqty{z_0}\transpose D v\pqty{z_1} = v\pqty{z_1}\transpose D\transpose v\pqty{z_0}.
			\end{equation}
			
			This holds for all \(z_0, z_1\), so \(D = D\transpose\). Set \(A = D\), which is symmetric. Thus:
			
			\begin{equation}
				f\pqty{z_0 + z_1} = v\pqty{z_1}\transpose A v\pqty{z_0}.
			\end{equation}
			
			For uniqueness, suppose there exist two symmetric matrices \(A\) and \(B\) such that \(v\pqty{z_1}\transpose A v\pqty{z_0} = v\pqty{z_1}\transpose B v\pqty{z_0}\) for all \(z_0, z_1\). Then:
			
			\begin{equation}
				v\pqty{z_1}\transpose \pqty{A - B} v\pqty{z_0} = 0.
			\end{equation}
			
			Since the \(g_k\pqty{z}\) span the solution space, and thus \(v\pqty{z_0}, v\pqty{z_1}\) can take on independent values, \(A - B = 0\), so \(A = B\). Hence, \(A\) is unique.
		\end{proof}
	\end{theorem}
	
	\section{Proof of \(\text{ArgSumCon}\pqty{m}\)}
	
	\begin{theorem}
		For any positive integer \(m\), \(\text{ArgSumCon}\pqty{m}\) holds: there exist constants \(c\pqty{k_0, \ldots, k_{m-1}}\) such that:
		
		\begin{equation}
			f\pqty{\sum_{j=0}^{m-1} z_j} = \sum_{k_0=0}^{n-1} \cdots \sum_{k_{m-1}=0}^{n-1} c\pqty{k_0, \ldots, k_{m-1}} \prod_{j=0}^{m-1} g_{k_j}\pqty{z_j}.
		\end{equation}
		\begin{proof}
			We prove this by induction on \(m\).
			
			\textbf{Base Case (\(m = 1\)):} For any solution \(f\pqty{z_0}\), since \(\{g_0\pqty{z_0}, \ldots, g_{n-1}\pqty{z_0}\}\) is a basis:
			
			\begin{equation}
				f\pqty{z_0} = \sum_{k_0=0}^{n-1} c\pqty{k_0} g_{k_0}\pqty{z_0},
			\end{equation}
			
			where the \(c\pqty{k_0}\) are constants. Thus, \(\text{ArgSumCon}\pqty{1}\) holds.
			
			\textbf{Inductive Step:} Assume \(\text{ArgSumCon}\pqty{m}\) holds for some \(m \geq 1\). We show it holds for \(m+1\). Consider:
			
			\begin{equation}
				f\pqty{\sum_{j=0}^{m} z_j}.
			\end{equation}
			
			Define \(u = \sum_{j=0}^{m-1} z_j\), so:
			
			\begin{equation}
				f\pqty{\sum_{j=0}^{m} z_j} = f\pqty{u + z_m}.
			\end{equation}
			
			By \(\text{ArgSumCon}\pqty{2}\) (Theorem 1), there exists a symmetric matrix \(A\) such that:
			
			\begin{equation}
				f\pqty{u + z_m} = v\pqty{z_m}\transpose A v\pqty{u}.
			\end{equation}
			
			Now, \(v\pqty{u} = \begin{bmatrix} g_0\pqty{u} \\ \vdots \\ g_{n-1}\pqty{u} \end{bmatrix}\), where \(u = \sum_{j=0}^{m-1} z_j\). Since each \(g_l\pqty{u}\) is a solution, by the inductive hypothesis:
			
			\begin{equation}
				g_l\pqty{u} = g_l\pqty{\sum_{j=0}^{m-1} z_j} = \sum_{k_0=0}^{n-1} \cdots \sum_{k_{m-1}=0}^{n-1} d_l\pqty{k_0, \ldots, k_{m-1}} \prod_{j=0}^{m-1} g_{k_j}\pqty{z_j},
			\end{equation}
			
			for some constants \(d_l\pqty{k_0, \ldots, k_{m-1}}\). Substitute into the expression:
			
			\begin{align}
				f\pqty{u + z_m} &= v\pqty{z_m}\transpose A v\pqty{u} = \sum_{k=0}^{n-1} \sum_{l=0}^{n-1} A_{kl} g_k\pqty{z_m} g_l\pqty{u} \\
				&= \sum_{k=0}^{n-1} \sum_{l=0}^{n-1} A_{kl} g_k\pqty{z_m} \bqty{\sum_{k_0=0}^{n-1} \cdots \sum_{k_{m-1}=0}^{n-1} d_l\pqty{k_0, \ldots, k_{m-1}} \prod_{j=0}^{m-1} g_{k_j}\pqty{z_j}}.
			\end{align}
			
			Rearrange the summations:
			
			\begin{align}
				f\pqty{\sum_{j=0}^{m} z_j} = \sum_{k_0=0}^{n-1} \cdots \sum_{k_{m-1}=0}^{n-1} \sum_{k=0}^{n-1} \sum_{l=0}^{n-1} A_{kl} d_l\pqty{k_0, \ldots, k_{m-1}} g_k\pqty{z_m} \prod_{j=0}^{m-1} g_{k_j}\pqty{z_j}.
			\end{align}
			
			Let \(k_m = k\), and define:
			
			\begin{equation}
				c\pqty{k_0, \ldots, k_m} = \sum_{l=0}^{n-1} A_{kl_m} d_l\pqty{k_0, \ldots, k_{m-1}},
			\end{equation}
			
			which are constants. Then:
			
			\begin{align}
				f\pqty{\sum_{j=0}^{m} z_j} &= \sum_{k_0=0}^{n-1} \cdots \sum_{k_{m-1}=0}^{n-1} \sum_{k_m=0}^{n-1} \pqty{\sum_{l=0}^{n-1} A_{k_m l} d_l\pqty{k_0, \ldots, k_{m-1}}} g_{k_m}\pqty{z_m} \prod_{j=0}^{m-1} g_{k_j}\pqty{z_j} \\
				&= \sum_{k_0=0}^{n-1} \cdots \sum_{k_m=0}^{n-1} c\pqty{k_0, \ldots, k_m} \prod_{j=0}^{m} g_{k_j}\pqty{z_j},
			\end{align}
			
			proving \(\text{ArgSumCon}\pqty{m+1}\). By induction, \(\text{ArgSumCon}\pqty{m}\) holds for all \(m \geq 1\).
		\end{proof}
	\end{theorem}
		
	\section{Justification of Matrix Invertibility}
	
	In the proof of \(\text{ArgSumCon}\pqty{2}\), we inverted the Wronskian matrix \(W\pqty{0}\). Here, we justify its invertibility.
	
	\begin{lemma}
		The Wronskian matrix \(W\pqty{z}\) of the basis \(\{g_0\pqty{z}, g_1\pqty{z}, \ldots, g_{n-1}\pqty{z}\}\) is invertible for some \(z \in \mathbb{C}\), including \(z = 0\).
		\begin{proof}
			The Wronskian matrix at \(z\) is:
			
			\begin{equation}
				W\pqty{z} = \begin{bmatrix}
					g_0\pqty{z} & g_1\pqty{z} & \cdots & g_{n-1}\pqty{z} \\
					g_0'\pqty{z} & g_1'\pqty{z} & \cdots & g_{n-1}'\pqty{z} \\
					\vdots & \vdots & \ddots & \vdots \\
					g_0^{\pqty{n-1}}\pqty{z} & g_1^{\pqty{n-1}}\pqty{z} & \cdots & g_{n-1}^{\pqty{n-1}}\pqty{z}
				\end{bmatrix}.
			\end{equation}
			
			The determinant \(\det W\pqty{z}\) is the Wronskian of the functions \(g_0, \ldots, g_{n-1}\). Since these are linearly independent solutions to an \(n\)-th order linear differential equation, \(\det W\pqty{z} \neq 0\) for some \(z\). For constant-coefficient equations, if the characteristic equation has distinct roots, the solutions (e.g., \(e^{\lambda_i z}\)) ensure \(\det W\pqty{z} \neq 0\) for all \(z\). Even with repeated roots, using solutions like \(z^k e^{\lambda z}\), the Wronskian is non-zero at \(z = 0\). Thus, \(W\pqty{0}\) is invertible.
		\end{proof}
	\end{lemma}
		
	This completes the justification of all matrix inversions in the proofs.
	
\end{document}