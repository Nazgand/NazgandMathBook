%!TEX engine = lualatex
\documentclass{article}

% --- CORE MATH & LOGIC ---
\usepackage{amsmath}
\usepackage{amsthm}
\usepackage{amssymb}
\usepackage{amsfonts}
\usepackage{physics}
\usepackage{mathtools}

% --- TYPOGRAPHY & LAYOUT ---
\usepackage[a4paper,margin=0.0931547in]{geometry}
\usepackage{fontspec}
\usepackage{microtype}
\usepackage[skip=0.5\baselineskip]{parskip}

% --- COLORS ---
\usepackage{xcolor}
% Define Palette
\definecolor{FavoriteColor1}{HTML}{607CB2} % Background (Medium Blue)
\definecolor{FavoriteColor2}{HTML}{303E59} % Text/Frame (Dark Blue)
\definecolor{LinkColor}{HTML}{FFFF00}	  % Links (Bright Yellow)

% Apply Background to Page
\pagecolor{FavoriteColor1}

% Apply Default Text Color (Global)
\makeatletter
\newcommand{\globalcolor}{\color{FavoriteColor2}\global\let\default@color\current@color}
\makeatother
\AtBeginDocument{\globalcolor}

% --- THEOREM STYLING ---
\usepackage[many]{tcolorbox}

% Define the box style
\tcbset{
	nazgandbox/.style={
		enhanced,
		% Background matches page color exactly (looks transparent)
		colback=FavoriteColor1,
		% Frame matches text color
		colframe=FavoriteColor2,
		% Text inside box matches global text color
		coltext=FavoriteColor2,
		% Title text matches background (high contrast against dark frame)
		coltitle=FavoriteColor1,
		fonttitle=\bfseries\sffamily,
		boxrule=1pt,
		arc=2pt,
		left=5pt, right=5pt, top=5pt, bottom=5pt,
		boxsep=1pt,
		sharp corners=downhill,
		breakable
	}
}

% Standard Definitions
\theoremstyle{definition}
\newtheorem{theorem}{Theorem}[section]
\newtheorem{lemma}[theorem]{Lemma}
\newtheorem{definition}[theorem]{Definition}
\newtheorem{corollary}[theorem]{Corollary}

% Wrap standard theorems in the custom box
\tcolorboxenvironment{theorem}{nazgandbox}
\tcolorboxenvironment{lemma}{nazgandbox}
\tcolorboxenvironment{definition}{nazgandbox}
\tcolorboxenvironment{corollary}{nazgandbox}

% --- UTILITIES ---
\usepackage{datetime2}
\usepackage{minted}

% --- FONTS ---
\directlua{luaotfload.add_fallback("EmojiFallback",{"NotoColorEmoji:mode=harf;"})}
\setmainfont{Noto Serif}[RawFeature={fallback=EmojiFallback}, Ligatures=TeX]
\setsansfont{Noto Sans}[RawFeature={fallback=EmojiFallback}]
\setmonofont{Noto Sans Mono}[RawFeature={fallback=EmojiFallback}]

% --- REFERENCES & LINKS ---
\usepackage{xr}
\usepackage{hyperref}
\hypersetup{
	colorlinks=true,
	linkcolor=LinkColor, % Internal links (equations, sections)
	urlcolor=LinkColor,  % Web URLs
	citecolor=LinkColor, % Bibliography citations
	filecolor=LinkColor, % Links to local files
	pdfnewwindow=true
}
\usepackage{cleveref}

% --- CUSTOM MACROS ---
\pdfstringdefDisableCommands{\def\eqref#1{(\ref{#1})}}
\newcommand{\inlineeqnum}{\refstepcounter{equation}~~\mbox{(\theequation)}}
\numberwithin{equation}{section}

% Operators
\DeclareMathOperator{\arsinh}{arsinh}
\DeclareMathOperator{\arcosh}{arcosh}
\DeclareMathOperator{\artanh}{artanh}
\DeclareMathOperator{\rues}{Rues}
\DeclareMathOperator{\md}{Mod}
\DeclareMathOperator{\pow}{Pow}

% Helpers
\newcommand{\floor}[1]{{\left\lfloor#1\right\rfloor}}
\newcommand{\ceil}[1]{{\left\lceil#1\right\rceil}}
\newcommand{\transpose}{^\intercal}
\newcommand{\laplace}[1]{\mathcal{L}\Bqty{#1}\pqty{s}}
\newcommand{\laplaceInv}[1]{\mathcal{L}^{-1}\Bqty{#1}\pqty{t}}
\newcommand{\pdiff}[2]{\frac{\partial^{#2}}{\partial #1^{#2}}}
\newcommand{\dint}[4]{\int_{#1}^{#2}#3\,\mathrm{d}#4}
\newcommand{\ketten}[4]{\underset{#1}{\overset{#2}{\mathop{\vcenter{\hbox{\huge\(\mathrm{K}\)}}}}}\frac{#3}{#4}}
\newcommand{\replace}[2]{\Big\vert_{#1\to{#2}}}

% --- METADATA & TITLE ---
\author{Mark Andrew Gerads \(<\)\href{MailTo:Nazgand@Gmail.Com}{Nazgand@Gmail.Com}\(>\)}
\let\oldauthor\author
\renewcommand{\author}[1]{
	\oldauthor{
		Author: #1
		\\
		Editor: Mark Andrew Gerads \(<\)\href{MailTo:Nazgand@Gmail.Com}{Nazgand@Gmail.Com}\(>\)
	}
}
\date{\DTMnow}
\let\oldtitle\title
\renewcommand{\title}[1]{
	\oldtitle{
		\vspace{-1.5cm}
		\url{https://GitHub.Com/Nazgand/NazgandMathBook}
		\\
		#1
	}
}

% --- LUA AUTO-REFERENCER ---
\usepackage{luacode}
\begin{luacode*}
	local lfs = require("lfs")

	local function split_path(path)
		local t = {}
		for part in path:gmatch("[^/]+") do table.insert(t, part) end
		return t
	end

	local function get_relative_path(base, target)
		local b_parts = split_path(base)
		local t_parts = split_path(target)
		local i = 1
		while b_parts[i] and t_parts[i] and b_parts[i] == t_parts[i] do
			i = i + 1
		end
		local rel = {}
		for j = i, #b_parts do table.insert(rel, "..") end
		for j = i, #t_parts do table.insert(rel, t_parts[j]) end
		return table.concat(rel, "/")
	end

	local handle = io.popen("git rev-parse --show-toplevel 2>/dev/null")
	local git_root = handle:read("*a"):gsub("%s+$", "")
	handle:close()

	if git_root ~= "" then
		local project_root = git_root .. "/LaTeX"
		local cwd = lfs.currentdir()
		
		local function scan_recursive(path)
			if not lfs.attributes(path) then return end
			for file in lfs.dir(path) do
				if file ~= "." and file ~= ".." then
					local full_path = path .. "/" .. file
					local f_attr = lfs.attributes(full_path)
					
					if f_attr and f_attr.mode == "directory" then
						scan_recursive(full_path)
					elseif file:match("%.tex$") then
						local job = file:gsub("%.tex$", "")
						if job ~= tex.jobname and job ~= "NazgandStyle" then
							local target_base = full_path:gsub("%.tex$", "")
							local rel_path = get_relative_path(cwd, target_base)
							
							tex.sprint("\\externaldocument{" .. rel_path .. "}[" .. rel_path .. ".pdf]")
						end
					end
				end
			end
		end

		texio.write_nl("--- [XR] Auto-Linking ---")
		scan_recursive(project_root)
	end
\end{luacode*}
\title{Proof of \(x^{\frac{1}{x}} \leq e^{\frac{1}{e}}\) for \(x > 0\)}
\author{Gemini 2.0 Flash Experimental}

\begin{document}
	
	\maketitle
	
	\begin{abstract}
		We provide a complete proof that for any positive real number \(x\), the inequality \(x^{\frac{1}{x}} \leq e^{\frac{1}{e}}\) holds. This is achieved by analyzing the natural logarithm of the function \(g\pqty{x} = x^{\frac{1}{x}}\), demonstrating that its maximum occurs at \(x=e\).
	\end{abstract}
	
	\section{Introduction}
	
	The inequality \(x^{\frac{1}{x}} \leq e^{\frac{1}{e}}\) for \(x > 0\) is a well-known result. Our approach involves examining the function \(g\pqty{x} = x^{\frac{1}{x}}\). Since the natural logarithm is a strictly increasing function, finding the maximum of \(\ln(g\pqty{x})\) is equivalent to finding the maximum of \(g\pqty{x}\). This simplification makes the analysis considerably easier.
	
	\section{Proof}
	
	Let \(g\pqty{x} = x^{\frac{1}{x}}\) for \(x > 0\). We define \(f\pqty{x} = \ln(g\pqty{x})\):
	
	\[f\pqty{x} = \ln\pqty{x^{\frac{1}{x}}} = \frac{1}{x} \ln x = \frac{\ln x}{x}\]
	
	Our goal is to find the maximum of \(f\pqty{x}\). Since the natural logarithm is a monotonically increasing function, the \(x\) value at which \(f\pqty{x}\) attains its maximum will be the same \(x\) value at which \(g\pqty{x}\) attains its maximum.
	
	To find the maximum of \(f\pqty{x}\), we compute its first derivative:
	
	\[f'\pqty{x} = \frac{\frac{1}{x} \cdot x - \ln x \cdot 1}{x^2} = \frac{1 - \ln x}{x^2}\]
	
	We find the critical points by setting \(f'\pqty{x} = 0\):
	
	\[\frac{1 - \ln x}{x^2} = 0\]
	
	Since \(x^2 > 0\) for \(x > 0\), this is equivalent to:
	
	\[1 - \ln x = 0\]
	\[\ln x = 1\]
	\[x = e\]
	
	Now, we use the second derivative test to determine if this critical point corresponds to a maximum or minimum. We calculate the second derivative of \(f\pqty{x}\):
	
	\[f''\pqty{x} = \frac{-\frac{1}{x} \cdot x^2 - \pqty{1 - \ln x} \cdot 2x}{x^4} = \frac{-x - 2x + 2x \ln x}{x^4} = \frac{-3x + 2x \ln x}{x^4} = \frac{-3 + 2 \ln x}{x^3}\]
	
	We evaluate the second derivative at \(x = e\):
	
	\[f''\pqty{e} = \frac{-3 + 2 \ln e}{e^3} = \frac{-3 + 2}{e^3} = \frac{-1}{e^3}\]
	
	Since \(f''\pqty{e} < 0\), the function \(f\pqty{x}\) has a local maximum at \(x = e\).
	
	Because \(f'\pqty{x} > 0\) for \(0 < x < e\) and \(f'\pqty{x} < 0\) for \(x > e\), \(f\pqty{x}\) is increasing on \(\pqty{0, e}\) and decreasing on \(\pqty{e, \infty}\). Thus, the local maximum at \(x=e\) is a global maximum.
	
	The maximum value of \(f\pqty{x}\) is:
	
	\[f\pqty{e} = \frac{\ln e}{e} = \frac{1}{e}\]
	
	Since the maximum of \(f\pqty{x} = \ln(g\pqty{x})\) occurs at \(x=e\), the maximum of \(g\pqty{x} = x^{\frac{1}{x}}\) also occurs at \(x=e\). The maximum value of \(g\pqty{x}\) is:
	
	\[g\pqty{e} = e^{\frac{1}{e}}\]
	
	Therefore, for all \(x > 0\):
	
	\[g\pqty{x} = x^{\frac{1}{x}} \leq e^{\frac{1}{e}}\]
	
	This completes the proof.
	
\end{document}