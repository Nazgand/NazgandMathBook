%!TEX engine = lualatex
\documentclass{article}

% --- CORE MATH & LOGIC ---
\usepackage{amsmath}
\usepackage{amsthm}
\usepackage{amssymb}
\usepackage{amsfonts}
\usepackage{physics}
\usepackage{mathtools}

% --- TYPOGRAPHY & LAYOUT ---
\usepackage[a4paper,margin=0.0931547in]{geometry}
\usepackage{fontspec}
\usepackage{microtype}
\usepackage[skip=0.5\baselineskip]{parskip}

% --- COLORS ---
\usepackage{xcolor}
% Define Palette
\definecolor{FavoriteColor1}{HTML}{607CB2} % Background (Medium Blue)
\definecolor{FavoriteColor2}{HTML}{303E59} % Text/Frame (Dark Blue)
\definecolor{LinkColor}{HTML}{FFFF00}	  % Links (Bright Yellow)

% Apply Background to Page
\pagecolor{FavoriteColor1}

% Apply Default Text Color (Global)
\makeatletter
\newcommand{\globalcolor}{\color{FavoriteColor2}\global\let\default@color\current@color}
\makeatother
\AtBeginDocument{\globalcolor}

% --- THEOREM STYLING ---
\usepackage[many]{tcolorbox}

% Define the box style
\tcbset{
	nazgandbox/.style={
		enhanced,
		% Background matches page color exactly (looks transparent)
		colback=FavoriteColor1,
		% Frame matches text color
		colframe=FavoriteColor2,
		% Text inside box matches global text color
		coltext=FavoriteColor2,
		% Title text matches background (high contrast against dark frame)
		coltitle=FavoriteColor1,
		fonttitle=\bfseries\sffamily,
		boxrule=1pt,
		arc=2pt,
		left=5pt, right=5pt, top=5pt, bottom=5pt,
		boxsep=1pt,
		sharp corners=downhill,
		breakable
	}
}

% Standard Definitions
\theoremstyle{definition}
\newtheorem{theorem}{Theorem}[section]
\newtheorem{lemma}[theorem]{Lemma}
\newtheorem{definition}[theorem]{Definition}
\newtheorem{corollary}[theorem]{Corollary}

% Wrap standard theorems in the custom box
\tcolorboxenvironment{theorem}{nazgandbox}
\tcolorboxenvironment{lemma}{nazgandbox}
\tcolorboxenvironment{definition}{nazgandbox}
\tcolorboxenvironment{corollary}{nazgandbox}

% --- UTILITIES ---
\usepackage{datetime2}
\usepackage{minted}

% --- FONTS ---
\directlua{luaotfload.add_fallback("EmojiFallback",{"NotoColorEmoji:mode=harf;"})}
\setmainfont{Noto Serif}[RawFeature={fallback=EmojiFallback}, Ligatures=TeX]
\setsansfont{Noto Sans}[RawFeature={fallback=EmojiFallback}]
\setmonofont{Noto Sans Mono}[RawFeature={fallback=EmojiFallback}]

% --- REFERENCES & LINKS ---
\usepackage{xr}
\usepackage{hyperref}
\hypersetup{
	colorlinks=true,
	linkcolor=LinkColor, % Internal links (equations, sections)
	urlcolor=LinkColor,  % Web URLs
	citecolor=LinkColor, % Bibliography citations
	filecolor=LinkColor, % Links to local files
	pdfnewwindow=true
}
\usepackage{cleveref}

% --- CUSTOM MACROS ---
\pdfstringdefDisableCommands{\def\eqref#1{(\ref{#1})}}
\newcommand{\inlineeqnum}{\refstepcounter{equation}~~\mbox{(\theequation)}}
\numberwithin{equation}{section}

% Operators
\DeclareMathOperator{\arsinh}{arsinh}
\DeclareMathOperator{\arcosh}{arcosh}
\DeclareMathOperator{\artanh}{artanh}
\DeclareMathOperator{\rues}{Rues}
\DeclareMathOperator{\md}{Mod}
\DeclareMathOperator{\pow}{Pow}

% Helpers
\newcommand{\floor}[1]{{\left\lfloor#1\right\rfloor}}
\newcommand{\ceil}[1]{{\left\lceil#1\right\rceil}}
\newcommand{\transpose}{^\intercal}
\newcommand{\laplace}[1]{\mathcal{L}\Bqty{#1}\pqty{s}}
\newcommand{\laplaceInv}[1]{\mathcal{L}^{-1}\Bqty{#1}\pqty{t}}
\newcommand{\pdiff}[2]{\frac{\partial^{#2}}{\partial #1^{#2}}}
\newcommand{\dint}[4]{\int_{#1}^{#2}#3\,\mathrm{d}#4}
\newcommand{\ketten}[4]{\underset{#1}{\overset{#2}{\mathop{\vcenter{\hbox{\huge\(\mathrm{K}\)}}}}}\frac{#3}{#4}}
\newcommand{\replace}[2]{\Big\vert_{#1\to{#2}}}

% --- METADATA & TITLE ---
\author{Mark Andrew Gerads \(<\)\href{MailTo:Nazgand@Gmail.Com}{Nazgand@Gmail.Com}\(>\)}
\let\oldauthor\author
\renewcommand{\author}[1]{
	\oldauthor{
		Author: #1
		\\
		Editor: Mark Andrew Gerads \(<\)\href{MailTo:Nazgand@Gmail.Com}{Nazgand@Gmail.Com}\(>\)
	}
}
\date{\DTMnow}
\let\oldtitle\title
\renewcommand{\title}[1]{
	\oldtitle{
		\vspace{-1.5cm}
		\url{https://GitHub.Com/Nazgand/NazgandMathBook}
		\\
		#1
	}
}

% --- LUA AUTO-REFERENCER ---
\usepackage{luacode}
\begin{luacode*}
	local lfs = require("lfs")

	local function split_path(path)
		local t = {}
		for part in path:gmatch("[^/]+") do table.insert(t, part) end
		return t
	end

	local function get_relative_path(base, target)
		local b_parts = split_path(base)
		local t_parts = split_path(target)
		local i = 1
		while b_parts[i] and t_parts[i] and b_parts[i] == t_parts[i] do
			i = i + 1
		end
		local rel = {}
		for j = i, #b_parts do table.insert(rel, "..") end
		for j = i, #t_parts do table.insert(rel, t_parts[j]) end
		return table.concat(rel, "/")
	end

	local handle = io.popen("git rev-parse --show-toplevel 2>/dev/null")
	local git_root = handle:read("*a"):gsub("%s+$", "")
	handle:close()

	if git_root ~= "" then
		local project_root = git_root .. "/LaTeX"
		local cwd = lfs.currentdir()
		
		local function scan_recursive(path)
			if not lfs.attributes(path) then return end
			for file in lfs.dir(path) do
				if file ~= "." and file ~= ".." then
					local full_path = path .. "/" .. file
					local f_attr = lfs.attributes(full_path)
					
					if f_attr and f_attr.mode == "directory" then
						scan_recursive(full_path)
					elseif file:match("%.tex$") then
						local job = file:gsub("%.tex$", "")
						if job ~= tex.jobname and job ~= "NazgandStyle" then
							local target_base = full_path:gsub("%.tex$", "")
							local rel_path = get_relative_path(cwd, target_base)
							
							tex.sprint("\\externaldocument{" .. rel_path .. "}[" .. rel_path .. ".pdf]")
						end
					end
				end
			end
		end

		texio.write_nl("--- [XR] Auto-Linking ---")
		scan_recursive(project_root)
	end
\end{luacode*}
\title{
	Rigorous Proof of the 2-Argument Matrix Conjecture for
	
	Homogeneous Linear Differential Equations with Constant Coefficients
}
\author{Claude Sonnet}

\begin{document}

\maketitle

\section{Introduction}

We consider a homogeneous linear differential equation with constant coefficients:

\[
0 = \sum_{k=0}^n a_k f^{\pqty{k}}\pqty{z}, \quad a_k \in \mathbb{C}, a_n \neq 0
\]

Let \(\{g_0, \ldots, g_{n-1}\}\) be a basis for the solution space of this equation.

\section{The Conjecture}

For any solution \(f\) of the differential equation, there exists a unique symmetric matrix \(A \in \mathbb{C}^{n \times n}\) such that:

\[
f\pqty{z_0 + z_1} = v\pqty{z_1}\transpose A v\pqty{z_0}
\]

where \(v\pqty{z} = \bqty{g_0\pqty{z}, \ldots, g_{n-1}\pqty{z}}\transpose\), and \(z_0, z_1 \in \mathbb{C}\).

\section{Proof}

We will prove this conjecture through a series of lemmas and claims.

\begin{lemma}
For any solution \(f\) and any fixed \(z_1 \in \mathbb{C}\), the function \(h\pqty{z_0} := f\pqty{z_0 + z_1}\) is also a solution.
\end{lemma}

\begin{proof}
This follows from the linearity and constant coefficients of the differential equation:
\[
\sum_{k=0}^n a_k h^{\pqty{k}}\pqty{z_0} = \sum_{k=0}^n a_k f^{\pqty{k}}\pqty{z_0 + z_1} = 0
\]
\end{proof}

\begin{lemma}
For any solution \(f\), there exist continuous functions \(c_k: \mathbb{C} \to \mathbb{C}\) such that:
\[
f\pqty{z_0 + z_1} = \sum_{k=0}^{n-1} c_k\pqty{z_1} g_k\pqty{z_0}
\]
for all \(z_0, z_1 \in \mathbb{C}\).
\end{lemma}

\begin{proof}
Fix \(z_1\). By Lemma 1, \(h\pqty{z_0} = f\pqty{z_0 + z_1}\) is a solution. Since \(\{g_k\}\) is a basis for the solution space, we can write:
\[
h\pqty{z_0} = \sum_{k=0}^{n-1} c_k\pqty{z_1} g_k\pqty{z_0}
\]
where \(c_k\pqty{z_1}\) are constants with respect to \(z_0\). As \(z_1\) varies, these constants become functions of \(z_1\). The continuity of \(c_k\pqty{z_1}\) follows from the continuity of solutions to linear differential equations with constant coefficients.
\end{proof}

\begin{theorem}
The functions \(c_k\pqty{z_1}\) satisfy the same differential equation as \(f\), i.e.,
\[
\sum_{j=0}^n a_j c_k^{\pqty{j}}\pqty{z_1} = 0
\]
for all \(k = 0, \ldots, n-1\) and all \(z_1 \in \mathbb{C}\).
\end{theorem}

\begin{proof}
Apply the differential operator \(\sum_{j=0}^n a_j \pdiff{z_1}{j}\) to both sides of the equation in Lemma 2:
\[
0 = \sum_{j=0}^n a_j f^{\pqty{j}}\pqty{z_0 + z_1} = \sum_{k=0}^{n-1} \pqty{\sum_{j=0}^n a_j c_k^{\pqty{j}}\pqty{z_1}} g_k\pqty{z_0}
\]

Let \(h_k\pqty{z_1} = \sum_{j=0}^n a_j c_k^{\pqty{j}}\pqty{z_1}\). We have:

\[
\sum_{k=0}^{n-1} h_k\pqty{z_1} g_k\pqty{z_0} = 0 \quad \forall z_0, z_1 \in \mathbb{C}
\]

Since \(\{g_k\}\) form a basis for the solution space, they are linearly independent. Therefore, the only way for this equation to hold for all \(z_0\) is if all \(h_k\pqty{z_1}\) are identically zero. Thus, \(\sum_{j=0}^n a_j c_k^{\pqty{j}}\pqty{z_1} = 0\) for all \(z_1\) and all \(k\).
\end{proof}

\begin{lemma}
There exist constants \(d_{km} \in \mathbb{C}\) such that:
\[
c_k\pqty{z_1} = \sum_{m=0}^{n-1} d_{km} g_m\pqty{z_1}
\]
for all \(k = 0, \ldots, n-1\) and all \(z_1 \in \mathbb{C}\).
\end{lemma}

\begin{proof}
This follows directly from the fact that \(c_k\pqty{z_1}\) satisfies the same differential equation as \(f\) (proven in the previous claim) and that \(\{g_m\}\) is a basis for the solution space of this equation.
\end{proof}

\begin{theorem}
The 2-argument matrix conjecture holds.
\end{theorem}

\begin{proof}
1) From Lemma 2 and Lemma 3, we can write:
\[
f\pqty{z_0 + z_1} = \sum_{k=0}^{n-1} \sum_{m=0}^{n-1} d_{km} g_m\pqty{z_1} g_k\pqty{z_0}
\]

2) Define matrix \(A\) with entries \(A_{km} = d_{km}\). Then:
\[
f\pqty{z_0 + z_1} = v\pqty{z_1}\transpose A v\pqty{z_0}
\]

3) To prove \(A\) is symmetric, note that \(f\pqty{z_0 + z_1} = f\pqty{z_1 + z_0}\), which implies:
\[
v\pqty{z_1}\transpose A v\pqty{z_0} = v\pqty{z_0}\transpose A v\pqty{z_1} = v\pqty{z_1}\transpose A\transpose v\pqty{z_0}
\]
As this holds for all \(z_0, z_1\), and \(\{g_k\}\) form a basis, we must have \(A = A\transpose\).

4) Uniqueness follows from the linear independence of \(\{g_k\}\). If there were two such matrices \(A\) and \(B\), then:
\[
v\pqty{z_1}\transpose \pqty{A-B} v\pqty{z_0} = 0 \quad \forall z_0, z_1
\]
This implies \(A-B = 0\), hence \(A = B\).
\end{proof}

\section{Conclusion}

This proof rigorously establishes the 2-argument matrix conjecture for homogeneous linear differential equations with constant coefficients. The key steps involve showing that translated solutions and coefficient functions satisfy the original differential equation, leveraging the linear independence of the basis functions, and carefully constructing the symmetric matrix \(A\).

\section{Implications and Applications}

This result has several important implications:

1. It provides a compact representation of solutions to linear differential equations, potentially simplifying their analysis and manipulation.

2. The symmetric matrix \(A\) encodes all the information about a particular solution, which could be useful in classification or comparison of solutions.

3. This representation might lead to new numerical methods for solving or approximating solutions to linear differential equations.

4. In the context of linear systems theory, this result could provide insights into the structure of impulse responses and transfer functions.

5. The proof technique used here might be adaptable to other classes of differential equations or functional equations.

Further research could explore generalizations to non-homogeneous equations, partial differential equations, or equations with non-constant coefficients.

\end{document}
