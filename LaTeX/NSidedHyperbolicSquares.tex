%!TEX engine = lualatex
\documentclass{article}

% --- CORE MATH & LOGIC ---
\usepackage{amsmath}
\usepackage{amsthm}
\usepackage{amssymb}
\usepackage{amsfonts}
\usepackage{physics}
\usepackage{mathtools}

% --- TYPOGRAPHY & LAYOUT ---
\usepackage[a4paper,margin=0.0931547in]{geometry}
\usepackage{fontspec}
\usepackage{microtype}
\usepackage[skip=0.5\baselineskip]{parskip}

% --- COLORS ---
\usepackage{xcolor}
% Define Palette
\definecolor{FavoriteColor1}{HTML}{607CB2} % Background (Medium Blue)
\definecolor{FavoriteColor2}{HTML}{303E59} % Text/Frame (Dark Blue)
\definecolor{LinkColor}{HTML}{FFFF00}	  % Links (Bright Yellow)

% Apply Background to Page
\pagecolor{FavoriteColor1}

% Apply Default Text Color (Global)
\makeatletter
\newcommand{\globalcolor}{\color{FavoriteColor2}\global\let\default@color\current@color}
\makeatother
\AtBeginDocument{\globalcolor}

% --- THEOREM STYLING ---
\usepackage[many]{tcolorbox}

% Define the box style
\tcbset{
	nazgandbox/.style={
		enhanced,
		% Background matches page color exactly (looks transparent)
		colback=FavoriteColor1,
		% Frame matches text color
		colframe=FavoriteColor2,
		% Text inside box matches global text color
		coltext=FavoriteColor2,
		% Title text matches background (high contrast against dark frame)
		coltitle=FavoriteColor1,
		fonttitle=\bfseries\sffamily,
		boxrule=1pt,
		arc=2pt,
		left=5pt, right=5pt, top=5pt, bottom=5pt,
		boxsep=1pt,
		sharp corners=downhill,
		breakable
	}
}

% Standard Definitions
\theoremstyle{definition}
\newtheorem{theorem}{Theorem}[section]
\newtheorem{lemma}[theorem]{Lemma}
\newtheorem{definition}[theorem]{Definition}
\newtheorem{corollary}[theorem]{Corollary}

% Wrap standard theorems in the custom box
\tcolorboxenvironment{theorem}{nazgandbox}
\tcolorboxenvironment{lemma}{nazgandbox}
\tcolorboxenvironment{definition}{nazgandbox}
\tcolorboxenvironment{corollary}{nazgandbox}

% --- UTILITIES ---
\usepackage{datetime2}
\usepackage{minted}

% --- FONTS ---
\directlua{luaotfload.add_fallback("EmojiFallback",{"NotoColorEmoji:mode=harf;"})}
\setmainfont{Noto Serif}[RawFeature={fallback=EmojiFallback}, Ligatures=TeX]
\setsansfont{Noto Sans}[RawFeature={fallback=EmojiFallback}]
\setmonofont{Noto Sans Mono}[RawFeature={fallback=EmojiFallback}]

% --- REFERENCES & LINKS ---
\usepackage{xr}
\usepackage{hyperref}
\hypersetup{
	colorlinks=true,
	linkcolor=LinkColor, % Internal links (equations, sections)
	urlcolor=LinkColor,  % Web URLs
	citecolor=LinkColor, % Bibliography citations
	filecolor=LinkColor, % Links to local files
	pdfnewwindow=true
}
\usepackage{cleveref}

% --- CUSTOM MACROS ---
\pdfstringdefDisableCommands{\def\eqref#1{(\ref{#1})}}
\newcommand{\inlineeqnum}{\refstepcounter{equation}~~\mbox{(\theequation)}}
\numberwithin{equation}{section}

% Operators
\DeclareMathOperator{\arsinh}{arsinh}
\DeclareMathOperator{\arcosh}{arcosh}
\DeclareMathOperator{\artanh}{artanh}
\DeclareMathOperator{\rues}{Rues}
\DeclareMathOperator{\md}{Mod}
\DeclareMathOperator{\pow}{Pow}

% Helpers
\newcommand{\floor}[1]{{\left\lfloor#1\right\rfloor}}
\newcommand{\ceil}[1]{{\left\lceil#1\right\rceil}}
\newcommand{\transpose}{^\intercal}
\newcommand{\laplace}[1]{\mathcal{L}\Bqty{#1}\pqty{s}}
\newcommand{\laplaceInv}[1]{\mathcal{L}^{-1}\Bqty{#1}\pqty{t}}
\newcommand{\pdiff}[2]{\frac{\partial^{#2}}{\partial #1^{#2}}}
\newcommand{\dint}[4]{\int_{#1}^{#2}#3\,\mathrm{d}#4}
\newcommand{\ketten}[4]{\underset{#1}{\overset{#2}{\mathop{\vcenter{\hbox{\huge\(\mathrm{K}\)}}}}}\frac{#3}{#4}}
\newcommand{\replace}[2]{\Big\vert_{#1\to{#2}}}

% --- METADATA & TITLE ---
\author{Mark Andrew Gerads \(<\)\href{MailTo:Nazgand@Gmail.Com}{Nazgand@Gmail.Com}\(>\)}
\let\oldauthor\author
\renewcommand{\author}[1]{
	\oldauthor{
		Author: #1
		\\
		Editor: Mark Andrew Gerads \(<\)\href{MailTo:Nazgand@Gmail.Com}{Nazgand@Gmail.Com}\(>\)
	}
}
\date{\DTMnow}
\let\oldtitle\title
\renewcommand{\title}[1]{
	\oldtitle{
		\vspace{-1.5cm}
		\url{https://GitHub.Com/Nazgand/NazgandMathBook}
		\\
		#1
	}
}

% --- LUA AUTO-REFERENCER ---
\usepackage{luacode}
\begin{luacode*}
	local lfs = require("lfs")

	local function split_path(path)
		local t = {}
		for part in path:gmatch("[^/]+") do table.insert(t, part) end
		return t
	end

	local function get_relative_path(base, target)
		local b_parts = split_path(base)
		local t_parts = split_path(target)
		local i = 1
		while b_parts[i] and t_parts[i] and b_parts[i] == t_parts[i] do
			i = i + 1
		end
		local rel = {}
		for j = i, #b_parts do table.insert(rel, "..") end
		for j = i, #t_parts do table.insert(rel, t_parts[j]) end
		return table.concat(rel, "/")
	end

	local handle = io.popen("git rev-parse --show-toplevel 2>/dev/null")
	local git_root = handle:read("*a"):gsub("%s+$", "")
	handle:close()

	if git_root ~= "" then
		local project_root = git_root .. "/LaTeX"
		local cwd = lfs.currentdir()
		
		local function scan_recursive(path)
			if not lfs.attributes(path) then return end
			for file in lfs.dir(path) do
				if file ~= "." and file ~= ".." then
					local full_path = path .. "/" .. file
					local f_attr = lfs.attributes(full_path)
					
					if f_attr and f_attr.mode == "directory" then
						scan_recursive(full_path)
					elseif file:match("%.tex$") then
						local job = file:gsub("%.tex$", "")
						if job ~= tex.jobname and job ~= "NazgandStyle" then
							local target_base = full_path:gsub("%.tex$", "")
							local rel_path = get_relative_path(cwd, target_base)
							
							tex.sprint("\\externaldocument{" .. rel_path .. "}[" .. rel_path .. ".pdf]")
						end
					end
				end
			end
		end

		texio.write_nl("--- [XR] Auto-Linking ---")
		scan_recursive(project_root)
	end
\end{luacode*}
\title{
	n-sided Hyperbolic Squares
}
\begin{document}
	
	\maketitle
	
	\begin{abstract}
		The goal of this paper is to analyze the properties (such as edge length, incircle radius, circumcircle radius) of an n-sided hyperbolic square, meaning a regular polygon with \(n\) right angles. These polygons can tile the hyperbolic plane, which looks nice plotted with the Cartesian Hyperbolic Plane Metric.
	\end{abstract}
	
	\section{Calculations}
	Let the Gaussian curvature of the plane be \(-1\).
	
	The regular polygon with \(n\) right angles can be partitioned into \(2n\) right triangles with angles \(\frac{\pi}{n},\frac{\pi}{2},\frac{\pi}{4}\).
	
	The radius of the circumcircle is \(\arcosh\pqty{\cot\pqty{\frac{\pi}{n}}\cot\pqty{\frac{\pi}{4}}}=\arcosh\pqty{\cot\pqty{\frac{\pi}{n}}}\).
	
	The radius of the incircle is
	\(\arcosh\pqty{\frac{\cos\pqty{\frac{\pi}{4}}}{\sin\pqty{\frac{\pi}{n}}}}\).
	
	The length of an edge is
	\(2\arcosh\pqty{\frac{\cos\pqty{\frac{\pi}{n}}}{\sin\pqty{\frac{\pi}{4}}}}\).
	
	The length of the perimeter is
	\(2n\arcosh\pqty{\frac{\cos\pqty{\frac{\pi}{n}}}{\sin\pqty{\frac{\pi}{4}}}}\).
	
	The area is
	\(2n\pqty{\pi-\frac{\pi}{n}-\frac{\pi}{2}-\frac{\pi}{4}}=
	2\pi n\pqty{\frac{1}{4}-\frac{1}{n}}=
	\frac{n\pi}{2}-2\pi\).
	
	It should be noted that as \(n\to\infty\), the edge length approaches
	\(2\arcosh\pqty{\sqrt{2}}\approx 1.76274717403908605046521864995958461806\), where both the incircle and circumcircle are horocycles, and above which any edge length is allowed.
	
	\begin{table}[H]
		\begin{tabular}{|l|l|}
			\hline
			n  & radius of circumcircle                    \\ \hline
			5  & 0.842482081462007459111380941149711215310 \\ \hline
			6  & 1.14621583478058884390039365567400771581  \\ \hline
			7  & 1.36005184973956765038539610902535387417  \\ \hline
			8  & 1.52857091948099816127245618479367339329  \\ \hline
			9  & 1.66893379511935191412136015769839107377  \\ \hline
			10 & 1.78982041007171516740591260177484529866  \\ \hline
			11 & 1.89630753915117229883786713543359764353  \\ \hline
			12 & 1.99165239104943682406899667528592695415  \\ \hline
			13 & 2.07808427065529112824674264723998571714  \\ \hline
			14 & 2.15720370625423547470560897610422250689  \\ \hline
			15 & 2.23020294245828385075785639634686399084  \\ \hline
		\end{tabular}
	\end{table}
	
	\begin{table}[H]
		\begin{tabular}{|l|l|}
			\hline
			n  & radius of incircle                        \\ \hline
			5  & 0.626869662906177814144463376211936401478 \\ \hline
			6  & 0.881373587019543025232609324979792309028 \\ \hline
			7  & 1.07040486155894418433137235164029245455  \\ \hline
			8  & 1.22422622383903789500265495681793457016  \\ \hline
			9  & 1.35504851879687183155393515348937116243  \\ \hline
			10 & 1.46935174436818527325584431736164761679  \\ \hline
			11 & 1.57108858003724067935632301765558416239  \\ \hline
			12 & 1.66288589105862107565248503907940605953  \\ \hline
			13 & 1.74659470452044937381132825499312321725  \\ \hline
			14 & 1.82357635695908839705212496034624551058  \\ \hline
			15 & 1.89486539086573102025718355160194478600  \\ \hline
		\end{tabular}
	\end{table}
	
	\begin{table}[H]
		\begin{tabular}{|l|l|}
			\hline
			n  & length of an edge                         \\ \hline
			5  & 1.06127506190503565203301891621357348581  \\ \hline
			6  & 1.31695789692481670862504634730796844403  \\ \hline
			7  & 1.44907472267758633503217314325772678247  \\ \hline
			8  & 1.52857091948099816127245618479367339329  \\ \hline
			9  & 1.58069813785651414369050575393040896418  \\ \hline
			10 & 1.61692166751188651380348666582235462921  \\ \hline
			11 & 1.64319223747457343054577404285923637279  \\ \hline
			12 & 1.66288589105862107565248503907940605953  \\ \hline
			13 & 1.67804671290710848454907520339943861187  \\ \hline
			14 & 1.68997616809339866171304910723600185759  \\ \hline
			15 & 1.69953703096172053759430146589327334224  \\ \hline
		\end{tabular}
	\end{table}
	
\end{document}
